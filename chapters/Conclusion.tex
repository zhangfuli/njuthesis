\chapter{总结与展望}

\section{总结}

区块链网络具有去中心化、不可伪造、不可篡改的特性, 其被称为下一代的新型生产关系。随着区块链概念的普及, 其为供应链、金融等传统领域带来了巨大变革。当前, 去中心化应用逐步从概念验证阶段转变为工程化实践阶段。在去中心化应用落地的过程中, 区块链网络的构建消耗了大量的时间成本。BaaS基于云原生技术体系可以帮助区块链开发人员构建、管理、托管区块链网络, 这降低了去中心化应用的落地门槛。

然而, 当前BaaS平台依旧存在着诸多挑战。首先是商业化应用工具不完善, 当前BaaS平台由云提供商巨头把控, 行业马太效应明显。不同云厂商拥有其独立的BaaS平台, 缺乏顶层规划导致多云网络及区块链数据孤岛。其次, 虽然各个云厂商拥有独立的BaaS平台及其相关生态, 但本质上还是将云原生平台作为部署环境, 仅提供一种类似于脚本化部署区块链网路的功能, 远未挖掘云原生底层特性。

因此, 本文针对上述存在的挑战, 本文提出了一种面向Hyperledger Fabric的区块链云化框架。由于Kubernetes Operator可以将领域知识有效的集成到云原生基础设施, 充分发挥云原生的特性。因此本文首先对Kubernetes Operator进行快速评审, 快速评审的范围涵盖了计算机与软件工程领域的4个权威全文数据库并将Scoups和谷歌学术作为补充。将得到的51篇论文, 经过筛选得到15篇。对这15篇论文进行数据抽取获得了基于Kubernetes Operator云化的策略集。针对策略集中的策略在区块链背景中进行适配获得了区块链云化的具体实施方案。

其次, 根据区块链云化的具体实施方案设计实现了面向Hyperledger Fabric的区块链云化框架及其原型平台。云化框架及其原型平台利用Operator分别管理HF网络中的Ca、Orderer、Orderer节点。CRD作为框架的输入, 根据官方功能以及配置对节点的属性进行可插拔设计, 如Ca节点的CRLSizeLimit, Orderer节点的Genesis, Peer节点的LevelDB/CouchDB; Manager作为中枢处理单元, 通过对应三个Controller对上述三个节点进行协调循环, 时刻保持节点在期望状态。同时, 本文结合实施方案中的策略来提升框架及HF网络的可移植性、可靠性、易用性、可扩展性以及安全性并支持通过命令行方式启停HF网络节点; HF网络作为输出, 包含并不限于维持HF网络节点稳定运行的Deployement、Service等Kubernetes配置。

最后, 本文对原型工具进行了全面的测试与评估。以典型案例的方式搭建了HF网络进行功能性测试, 在框架及原型工具的评估方面, 本文利用SAAM方法对工具进行架构评估, 结果表明原型工具利用云原生的特性满足工具的易用性、可扩展性、安全性、可靠性以及可移植性等质量属性要求; 本文采用定性分析的方式结合五层成熟度模型对原型工具进行评估, 采用定量分析的方式对比HF官方BaaS平台Cello在网络部署时间上进行了全面对比。结果表明, 原型工具基本满足五层成熟度模型的功能且在部署时间和部署时间的稳定性上优于Cello。

本文提出的面向Hyperledger Fabric的区块链云化框架及其原型工具依托于Kubernetes基础设施, 可以方便的迁移到支持Kubernetes的任何云, 包括公有云、专有云以及混合云, 打破云厂商的垄断格局; 利用输入本应由领域专家执行的参数与操作, 减轻HF网络管理员的负担。利用Kubernetes Operator更原生的管理HF网络, 复用Kubernertes API的公共功能, 如PVC存储、资源隔离、内置认证等方式更好的发挥云原生的潜能, 提升HF网络的易用性、可扩展性、安全性、可靠性以及可移植性。

\section{局限}

本文搭建了原型平台并与Cello进行对比验证, 虽然原型平台在网络部署时间上优于Cello, 但本文提出的面向Hyperledger Fabric的区块链云化框架及其原型工具依旧还有很多局限性:

\begin{itemize}[itemindent=2em]
    \item 为轻量级、快速的将已有知识转移到实践中, 在区块链云化框架调研过程中采用了快速评审的方式, 其相对于系统文献综述(Systematic literature reviews, 简称SLRs)而言缺乏更加深层次的知识掌握。同时Kubernetes Operator以及云原生技术体系在工业界应用广泛而本文的快速评审面对的是学术工作, 未对灰色文献进行调研。

    \item 本文的Manager中枢控制系统利用Controller对Helm进行进行生命周期监控, 并通过CRD的配置对Helm value进行参数传递。这种方式虽然有利于快速启动HF网络, 但增加了Helm作为中间传递的过程, 未能直接对HF网络节点本身的Deployment等配置进行关联。同时, 由于Kubernetes版本的所限, 本框架在Helm chart中配置的apiVersion均需要支持Kubernetes 1.18版本及其以上。

    \item 在框架评估方面, 虽然基本满足五层成熟度模型的能力要求, 但目前原型工具尚未具备数据备份和恢复的能力, 未对当前原型工具及所搭建的区块数据进行远端备份。在可扩展性方面, Kubernetes Operator云化策略集中包含支持HPA、VPA的的可伸缩处理, 由于区块链本身的特性, 本文未对Peer进行HPA。但区块链云化框架原则上可以支持Peer的VPA, 并且针对Ca以及Orderer节点, 区块链云化框架未提供任何可伸缩的处理方式。

    \item 本文的虽然与Cello进行了验证评估工作, 但定量评估方面目前仅针对于网络部署时间这一方面, 而且目前仅对Docker环境进行对比分析, 受到环境所限, 未能使用Cello在Kubernetes中进行部署对比。Cello部署的Hyperledger Fabric为1.4版本, 原型工具部署的版本为2.X版本, 在版本上有些许不同。

    \item 本文在测试评估阶段选取了官方链码asset进行案例研究, 但这并非企业的完整大型应用, 缺乏在大规模HF网络测试评估。并且, 原型工具能够通过命令行的方式进行HF网络及其链码的搭建, 这虽然对于软件工程师而言门槛较低, 但未提供图形化的方式。

\end{itemize}



\section{展望}

本文提出的面向Hyperledger Fabric的区块链云化框架及其原型工具能够有效的利用Kubernetes Operator对HF网络进行云化, 但依旧存在不足, 本文仍需在以下方面进一步优化提升。

\begin{itemize}[itemindent=2em]
    \item 深入调研学术界以及工业界关于Kubernetes Operator赋能其他领域提升质量属性的策略, 尤其增加对于灰色文献的调研, 最终对本文的策略集进行补充, 应用到区块链云化框架及其原型工具中。

    \item 依托于本文的原型工具, 进一步抽象封装提供图形化界面支持; 同时, 结合可扩展性策略, 对HF网络的Peer增加VPA配置, 对Ca、Orderer节点增加HPA以及VPA配置, 使其具备弹性伸缩的能力提供更加稳定的服务; 进一步研究针对不同账本存储单元的扩充, 并提供将原型工具和区块账本数据远程备份的策略, 增强数据备份与恢复机制; 有效利用Prometheus监控体系抓取出来的监控指标数据, 进一步挖掘数据深层次的价值。 

    \item 进一步增加对框架及原型工具的评估手段, 面向企业级大规模区块链业务场景, 整理更多专家及HF开发人员对框架及其原型工具的评价, 不断汲取、筛选适合区块链场景的云化策略妒对本框架升级改造。 

\end{itemize}
