\chapter{总结与展望}

\section{总结}

领域驱动设计作为一种针对复杂业务流程的分析与建模方法,
正在成为大型分布式系统设计与实现的最佳解决方案,
其中的战术建模层次可以帮助更快速地进行建模设计与实现代码的落地。
然而,战术建模在应用时仍面临着许多挑战,
由于对战术模式的规范和约束不明确,往往会导致战术建模结果不够标准和规范;
开发人员和架构师对战术建模理解程度不同,无法统一和复用建模结果;
领域建模过程缺少相应的支持平台和工具。
为了解决这些挑战,
对该领域的研究进展与工业界实践经验进行了调查与总结,
提出了一套战术建模支持方法及工具。

具体来说,本文所提出的战术建模支持方法及工具包括以下三个具体贡献。
其一,通过对领域驱动设计战术建模过程的理论调研和对工业界从业人员的访谈,
本文总结得出一套战术建模指南,
包括八种战术模式、这些模式的重要属性、使用时机以及实现技术,
该建模指南经过工业界实践经验认证,可以作为战术建模实践时的指导。
其二,通过开展焦点小组讨论,本文构建了一种战术建模语言,
该战术建模语言作为使用战术建模支持方法的支撑,
提升了建模的效率与准确性。
以上两点构成了战术建模支持方法的主要内容。
其三,以所提出的战术建模语言为基础,
本文还实现了一个建模支持工具,
实现了灵活易用的可视化建模过程,
保证了扩展性与通用性,并避免了对特定平台的依赖。

本文提出的战术建模支持方法和工具,支持简单了解领域驱动设计战术建模基本概念,
通过灵活的可视化建模来实现建模过程,并校验建模结果的标准性与规范性,
为战术建模提供了一套可靠的流程。此外,还支持将建模结果以多种格式导出和存储,
还能为实现阶段构建框架项目,提高了战术建模结果的复用程度。
总体而言,
本文提出的工具也降低了使用战术建模的最低要求,
对领域驱动设计在实践中发展具有积极意义。

\section{展望}

本文提出的领域驱动设计战术建模支持方法及工具可以支持战术建模实践,
但仍有很多方面可以进一步提升。首先,
战术建模的属性和实现技术在实践时还应该做到根据业务特性进行变通,
对不同开发团队的不同业务,战术建模工具应该有不同的个性化配置;
其次,战术建模工具对于战术建模的结果利用程度还不够高,
除了生成框架项目之外,
未来该工具还将支持更多种形式与更细粒度的扩展结果;
最后,战术建模工具在多人协同建模方面支持度不够,仅能通过共享建模结果来进行协同,
未来该工具还将支持通过在线协作实现即时协同建模,提升建模过程的体验并提高建模效率。




最后部署operator的时候
在Kubernetes集群中部署«操作员»时,使用了Helm工具[17]。该工具允许使用一个命令部署应用程序(图6)。我们可以为Helm开发特殊的清单(图表),以表示Kubernetes中的«操作员»应用程序。这些清单包含有关应用程序的所有必要信息,因此Kubernetes可以正确地部署它。例如,我们可以指定应该在Kubernetes集群上部署多少个应用程序副本

计算资源管理层支持公有云、专有云以及混合云,为区块链服务及上层应用提供所需要的云基础资源。

区块链底层平台是BaaS平台的核心,构建于云容器服务集群之上,支持超级账本、以太坊等不同区块链底层架构。
区块链服务层依托底层区块链的支持,抽象封装一系列服务模块,简化开发工作,帮助企业快速部署区块链应用,降低区块链开发门槛。

管理服务为用户提供基本管理功能,包括平台用户权限管理功能、服务使用计费管理功能、通知功能等。
运维服务提供图形化的区块链管理运维服务能力,实时监控区块链网络运行数据,帮助运维人员及时发现并解决问题。

区块链云化降低开发门槛。区块链技术与其他技术不同之处在于它是融合密码学、P2P网络、分布式存储等多种技术的组合体,技术门槛高致使其开发成本高。而区块链云化产品BaaS平台将区块链技术封装在底层,使功能模块化,开发人员直接调用封装后的API接口即可完成一键部署,降低中小企业用区块链技术的门槛,从而推动区块链应用的落地。

区块链云化实现个性化定制。区块链云化产品BaaS平台可依托云服务商强大的业务能力,在提供标准服务基础上再根据开发者业务需求提供不同的配置,扩展开发者自定义的功能,满足其个性化需求,提高灵活性。同时通过BaaS平台可以沉淀出一层标准的区块链应用解决方案模板,为用户快速匹配建链场景。

市场前景广阔,发展迅速。区块链技术的不断成熟加速了区块链行业应用落地,不断扩大BaaS市场规模,对全球云计算的服务市场促进作用明显。特别是云服务开放性和资源可扩展性使其成为区块链应用落地的最佳载体,区块链与云计算结合愈发紧密。云链协同在加快区块链产业发展的同时,也成为云计算产业发展的关键性新动能。

负责实现云资源的管理调度,该模块会调用云资源管理适配模块的统一接口,所以底层不同云平台接口的差异性对该模块是透明的。该模块的主要功能有创建及删除虚拟机(Docker容器) 和网络资源、进行初始化配置、对已有资源进行扩容或缩容等操作。

对区块链节点的跨云部署支持,需要由该模块来实现对不同公有云、私有云的虚拟机、Docker容器等资源调度API的封装,屏蔽各种云平台API的差异性,对上层调用模块提供统一的资源管理接口。

华为云区块链服务基于可信、开放、服务全球的华为云上运行,华为云产品和服务具有华为独有的新技术,以降低成本、弹性灵活、电信级安全、高效自助管理等优势惠及用户,BCS 可以和华为云技术产品和行业解决方案无缝对接,帮助企业在安全、高效、不可篡改等基础上轻松跨入云时代,快速部署新解决方案和应用。


