\chapter{总结与展望}

\section{总结}

领域驱动设计作为一种针对复杂业务流程的分析与建模方法,
正在成为大型分布式系统设计与实现的最佳解决方案,
其中的战术建模层次可以帮助更快速地进行建模设计与实现代码的落地。
然而,战术建模在应用时仍面临着许多挑战,
由于对战术模式的规范和约束不明确,往往会导致战术建模结果不够标准和规范;
开发人员和架构师对战术建模理解程度不同,无法统一和复用建模结果;
领域建模过程缺少相应的支持平台和工具。
为了解决这些挑战,
对该领域的研究进展与工业界实践经验进行了调查与总结,
提出了一套战术建模支持方法及工具。

具体来说,本文所提出的战术建模支持方法及工具包括以下三个具体贡献。
其一,通过对领域驱动设计战术建模过程的理论调研和对工业界从业人员的访谈,
本文总结得出一套战术建模指南,
包括八种战术模式、这些模式的重要属性、使用时机以及实现技术,
该建模指南经过工业界实践经验认证,可以作为战术建模实践时的指导。
其二,通过开展焦点小组讨论,本文构建了一种战术建模语言,
该战术建模语言作为使用战术建模支持方法的支撑,
提升了建模的效率与准确性。
以上两点构成了战术建模支持方法的主要内容。
其三,以所提出的战术建模语言为基础,
本文还实现了一个建模支持工具,
实现了灵活易用的可视化建模过程,
保证了扩展性与通用性,并避免了对特定平台的依赖。

本文提出的战术建模支持方法和工具,支持简单了解领域驱动设计战术建模基本概念,
通过灵活的可视化建模来实现建模过程,并校验建模结果的标准性与规范性,
为战术建模提供了一套可靠的流程。此外,还支持将建模结果以多种格式导出和存储,
还能为实现阶段构建框架项目,提高了战术建模结果的复用程度。
总体而言,
本文提出的工具也降低了使用战术建模的最低要求,
对领域驱动设计在实践中发展具有积极意义。

\section{展望}

本文提出的领域驱动设计战术建模支持方法及工具可以支持战术建模实践,
但仍有很多方面可以进一步提升。首先,
战术建模的属性和实现技术在实践时还应该做到根据业务特性进行变通,
对不同开发团队的不同业务,战术建模工具应该有不同的个性化配置;
其次,战术建模工具对于战术建模的结果利用程度还不够高,
除了生成框架项目之外,
未来该工具还将支持更多种形式与更细粒度的扩展结果;
最后,战术建模工具在多人协同建模方面支持度不够,仅能通过共享建模结果来进行协同,
未来该工具还将支持通过在线协作实现即时协同建模,提升建模过程的体验并提高建模效率。

