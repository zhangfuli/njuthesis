% !Mode:: "TeX:UTF-8"
%# -*- coding:utf-8 -*-

%% 南京大学学位论文的示例文档
%% 作者:njuhan: https://github.com/njuHan
%% 源模版repo: https://github.com/njuHan/njuthesis-nju-thesis-template

\documentclass[macfonts,master,oneside]{njuthesis}
%% 审阅模式很重要,命令是下方使用的/blind
%% njuthesis 文档类的可选参数有:
%%   winfonts, linuxfonts, macfonts, adobefonts winfonts b选项使得文档使用Windows 系统提供的字体;linuxfonts 选项使得文档使用Linux 系统提供的字体;macfonts 选项使得文档使用Mac 系统提供的字体;adobefonts 选项使得文档使用Adobe提供的OTF中文字体(需自行下载安转)
%%   phd/master/bachelor 选择博士/硕士/学士论文
%%   twoside 或 oneside 指定排版的文档为双面打印或单面打印格式(twoside会使得chapter 章节从奇数页开始,即纸张的正面开始,因此会出现一些空白的页面)
%%   nobackinfo 取消封二页导师签名信息。注意,按照南大的规定,是需要签名页的。



%%%%%%%%%%%%%%%%%%%%%%%%%%%%%%%%%%%%%%%%%%%%%%%%%%%%%%%%%%%%%%%%%%%%%%%%%%%%%%%
% set up labelformat and labelsep for subfigure 详见: http://www.latexstudio.net/archives/8652.html
\captionsetup[subfigure]{labelformat=simple, labelsep=space}

%%%%%%%%%%%%%%%%%%%%%%%%%%%%%%%%%%%%%%%%%%%%%%%%%%%%%%%%%%%%%%%%%%%%%%%%%%%%%%%
% 设置《国家图书馆封面》的内容,仅博士论文才需要填写

% 设置论文按照《中国图书资料分类法》的分类编号
%\classification{0175.2}
% 设置论文按照《国际十进分类法UDC》的分类编号
% 该编号可在下述网址查询:http://www.udcc.org/udcsummary/php/index.php?lang=chi
%\udc{004.72}
% 国家图书馆封面上的论文标题第一行,不可换行。此属性可选,默认值为通过\title设置的标题。
%\nlctitlea{论文标题第一行}
% 国家图书馆封面上的论文标题第二行,不可换行。此属性可选,默认值为空白。
%\nlctitleb{论文标题第二行}
% 国家图书馆封面上的论文标题第三行,不可换行。此属性可选,默认值为空白。
%\nlctitlec{}
% 导师的单位名称及地址
%\supervisorinfo{南京大学计算机科学与技术系~~南京市汉口路22号~~210093}
% 答辩委员会主席
%\chairman{张三丰~~教授}
% 第一位评阅人
%\reviewera{阳顶天~~教授}
% 第二位评阅人
%\reviewerb{张无忌~~副教授}
% 第三位评阅人
%\reviewerc{黄裳~~教授}
% 第四位评阅人
%\reviewerd{郭靖~~研究员}


%%%%%%%%%%%%%%%%%%%%%%%%%%%%%%%%%%%%%%%%%%%%%%%%%%%%%%%%%%%%%%%%%%%%%%%%%%%%%%%
% 设置论文的中文封面


% % 单行论文标题,不可换行
\title{}
\titlea{基于Hyperledger Fabric的}
\titleb{区块链云化框架的研究与实现}

% 如果论文标题过长,可以分两行,第一行用\titlea{}定义,第二行用\titleb{}定义,
% 使用以下3行:
%\title{} %用于覆盖单行标题内容为空
%\titlea{长标题第一行}  %第一行标题写这里
%\titleb{长标题第二行用于长标题换行} %第二行标题写这里
% 注意: \title 不能都注释,它用于控制标题选择双行还是单行。\title{}如果内容为空,则编译\titlea{},titleb{}双行标题,否则编译单行标题


% 论文作者姓名
\author{张富利}
% 论文作者联系电话
\telphone{18851751795}
% 论文作者电子邮件地址
\email{MG1932016@smail.nju.edu.cn}
% 论文作者学生证号
\studentnum{MG1932016}
% 论文作者入学年份(年级)
\grade{2019}
% 论文作者毕业年份(届), 出版授权书的学位年度
\graduateyear{2022}
% 导师姓名职称
\supervisor{张贺~~教授}
% 导师的联系电话
\supervisortelphone{}
% 论文作者的学科与专业方向
\major{软件工程}
% 论文作者的研究方向
\researchfield{DevOps、区块链}
% 论文作者所在院系的中文名称
\department{软件学院}
% 论文作者所在学校或机构的名称。此属性可选,默认值为``南京大学''。
\institute{南京大学}
% 论文的提交日期,需设置年、月、日。
\submitdate{xxxx年 xx 月 xx 日}
% 论文的答辩日期,需设置年、月、日。
\defenddate{xxxx年 xx 月 xx 日}
% 论文的定稿日期,需设置年、月、日。
% 此属性可选,若注释\date{},则默认值为最后一次编译时的日期,精确到日。
% \date{2019年5月20日}

%%%%%%%%%%%%%%%%%%%%%%%%%%%%%%%%%%%%%%%%%%%%%%%%%%%%%%%%%%%%%%%%%%%%%%%%%%%%%%%
% 设置论文的英文封面

% 论文的英文标题,不可换行
\englishtitle{}
\englishabstracttitlea{Modeling Support Method and Tool for}
\englishabstracttitleb{ Domain-Driven Design}
% 论文作者姓名的拼音
\englishauthor{Fuli Zhang}
% 导师姓名职称的英文
\englishsupervisor{Professor~~He Zhang}
% 论文作者学科与专业的英文名
\englishmajor{Software Engineering}
% 论文作者所在院系的英文名称
\englishdepartment{Software Institute}
% 论文作者所在学校或机构的英文名称。此属性可选,默认值为``Nanjing University''。
\englishinstitute{Nanjing University}
% 论文完成日期的英文形式,它将出现在英文封面下方。需设置年、月、日。日期格式使用美国的日期
% 格式,即``Month day, year'',其中``Month''为月份的英文名全称,首字母大写;``day''为
% 该月中日期的阿拉伯数字表示;``year''为年份的四位阿拉伯数字表示。
% 此属性可选,若注释掉\englishdate{},则默认值为最后一次编译时的日期。
% \englishdate{May 20, 2019}

%%%%%%%%%%%%%%%%%%%%%%%%%%%%%%%%%%%%%%%%%%%%%%%%%%%%%%%%%%%%%%%%%%%%%%%%%%%%%%%
% 设置论文的中文摘要

% 设置中文摘要页面的论文标题及副标题的第一行。
% 此属性可选,其默认值为使用|\title|命令所设置的论文标题
%\abstracttitlea{标题第一行}
% 设置中文摘要页面的论文标题及副标题的第二行。
% 此属性可选,其默认值为空白
%\abstracttitleb{标题第二行用于长标题换行}

%%%%%%%%%%%%%%%%%%%%%%%%%%%%%%%%%%%%%%%%%%%%%%%%%%%%%%%%%%%%%%%%%%%%%%%%%%%%%%%
% 设置论文的英文摘要

% 设置英文摘要页面的论文标题及副标题的第一行。
% 此属性可选,其默认值为使用|\englishtitle|命令所设置的论文标题
%\englishabstracttitlea{englishabstracttitlea}
% 设置英文摘要页面的论文标题及副标题的第二行。
% 此属性可选,其默认值为空白
%\englishabstracttitleb{nglishabstracttitleb}

%%%%%%%%%%%%%%%%%%%%%%%%%%%%%%%%%%%%%%%%%%%%%%%%%%%%%%%%%%%%%%%%%%%%%%%%%%%%%%
%% 盲审命令,空白字段设置请看 .cls文件 \newcommand*{\blind}
%% 此外,请按照盲审要求自行去掉个人简历、致谢等页面中的个人信息
%%**********************!!!非常重要的盲审命令,送审前必选!!!****************
% \blind
%%**********************!!!非常重要的盲审命令,送审前必选!!!****************

%%%%%%%%%%%%%%%%%%%%%%%%%%%%%%%%%%%%%%%%%%%%%%%%%%%%%%%%%%%%%%%%%%%%%%%%%%%%%%%
\begin{document}

%%%%%%%%%%%%%%%%%%%%%%%%%%%%%%%%%%%%%%%%%%%%%%%%%%%%%%%%%%%%%%%%%%%%%%%%%%%%%%%

% 制作国家图书馆封面(博士学位论文才需要)
%\makenlctitle
% 制作中文封面
\maketitle
% 制作英文封面
\makeenglishtitle

%%%%%%%%%%%%%%%%%%%%%%%%%%%%%%%%%%%%%%%%%%%%%%%%%%%%%%%%%%%%%%%%%%%%%%%%%%%%%%%
% 开始前言部分
\frontmatter

\begin{abstract}
  领域驱动设计是一种针对复杂业务流程的软件设计方法,
  它可以帮助架构师和软件开发人员提炼业务流程和构建复杂软件系统。
  近年来,
  越来越多的团队将领域驱动设计应用到大型分布式系统的设计与实现中。

  然而,作为领域驱动设计的核心内容,战术建模的应用却存在诸多问题与挑战。
  其一,战术建模模式的实践过程仍缺少标准且统一的规范,
  使得实践者无法准确地使用这些模式建模业务流程;
  其二,战术建模实践过程所缺少的规范,
  也使得开发人员和架构设计人员在使用这些建模模式时具有不同的理解,
  从而带来了一系列项目团队沟通问题;
  其三,现有的大多数建模平台对领域建模过程的支持仍不完善,
  极大地阻碍了领域战术建模的应用。

  为了解决上述问题与挑战,本文首先调研了现阶段有关领域驱动设计战术建模的理论文献,
  对具有相关实践经验的从业者进行访谈,
  提炼出一系列面向战术建模过程的实践规范作为战术建模指南。
  所提炼的建模指南包含一系列战术建模中使用到的战术模式、这些战术模式的属性、
  使用时机与实现技术。
  然后,基于UML扩展机制定义标准的领域驱动设计战术建模元模型作为战术建模语言。
  上述建模指南和战术建模语言共同组成本文所提出的战术建模支持方法。
  最后,基于所提出的战术建模支持方法,
  本文还实现了一个可视化战术建模工具DDDD(Draw for Domain-Driven Design),
  该工具支持标准化的战术建模、对建模结果的验证、复用以及扩展,
  帮助实践者更规范地进行战术建模。

  本文采用案例研究对提出的建模支持方法及工具进行验证。
  结果表明本文提出的战术建模支持方法及工具,
  能够帮助领域驱动设计的实践者开展更规范化的战术建模过程,
  从而解决战术建模实践过程缺少规范带来的一系列挑战,
  一定程度上促进了战术建模的应用。
	% 同时应该注意到,空白页是故意留白,以便章节开头能够出现在偶数页。
	% 中文关键词。关键词之间用中文全角分号隔开,末尾无标点符号。
	\keywords{领域驱动设计;战术建模;建模工具;特定领域建模语言}
\end{abstract}

%%%%%%%%%%%%%%%%%%%%%%%%%%%%%%%%%%%%%%%%%%%%%%%%%%%%%%%%%%%%%%%%%%%%%%%%%%%%%%%
% 论文的英文摘要
\begin{englishabstract}
  As a software design approach for complex business process, Domain-Driven Design (DDD) enables architects and developers building complex software system.
  In recent years, more and more teams have applied DDD to the design and implementation of large-scale distributed systems.

  However, as the core means of DDD, 
  tactical modeling are facing many challenges in its application.
  Firstly, the practicing process of tactical modeling patterns still lacks standard and unified specifications,
   making it hard for practitioners to use these patterns to model business processes accurately; 
  secondly, the lack of specifications also makes developers and architects have different understandings when using the modeling patterns,
   leading to a series of communication problems in project team; 
   thirdly, most of the existing modeling platforms have incomplete support for the tactical modeling process, 
   severely impeding the application of DDD.


  In order to solve the aforementioned challenges, 
  this thesis first reviews theoretical literature on tactical modeling,
   then interviews practitioners with relevant practical experience, 
  and finally extracts a series of practical specifications for the tactical modeling process.
   As a guide for tactical modeling, the specifications contain a series of tactical patterns used in tactical modeling, 
   their attributes, time of using, and implementation techniques. 
   Furthermore, the standard tactical modeling metamodel is defined based on UML profile mechanism, as a tactical modeling language. 
   The above modeling guide and tactical modeling language together constitute the tactical modeling support method proposed in this thesis. 
   Based on the proposed method, this thesis also implements a visual tactical modeling tool called Draw for Domain-Driven Design, or DDDD. 
   The tool supports standardized tactical modeling, verification, reuse and extension of modeling results to help practitioners develop tactical modeling in a more standardized way.


  This thesis uses case studies to verify the proposed modeling support method and tool. 
  Results show that the proposed method and tool can help DDD practitioners to conduct a more standardized tactical modeling process, 
  thus solving the challenges caused by lack of specification in the tactical modeling process, 
  and promoting the application of tactical modeling to a certain extent.
  
  \englishkeywords{Domain-driven design, Tactical modeling, Modeling tool, Domain-specific modeling language}
\end{englishabstract}

%%%%%%%%%%%%%%%%%%%%%%%%%%%%%%%%%%%%%%%%%%%%%%%%%%%%%%%%%%%%%%%%%%%%%%%%%%%%%%%
% 论文的前言,应放在目录之前,中英文摘要之后,一般不需要
%
%\begin{preface}
%
%在过去的40年中,手写中文文本领域识别(HCTR)取得了很大的进展[1,2]。
%
%\vspace{1cm}
%\begin{flushright}
%饶安逸\\
%2018年5月15日于南大仙林
%\end{flushright}
%
%\end{preface}

%%%%%%%%%%%%%%%%%%%%%%%%%%%%%%%%%%%%%%%%%%%%%%%%%%%%%%%%%%%%%%%%%%%%%%%%%%%%%%%
% 生成论文目录
\tableofcontents

%%%%%%%%%%%%%%%%%%%%%%%%%%%%%%%%%%%%%%%%%%%%%%%%%%%%%%%%%%%%%%%%%%%%%%%%%%%%%%%
% 生成插图清单。如无需插图清单则可注释掉下述语句。
\listoffigures

%%%%%%%%%%%%%%%%%%%%%%%%%%%%%%%%%%%%%%%%%%%%%%%%%%%%%%%%%%%%%%%%%%%%%%%%%%%%%%%
% 生成附表清单。如无需附表清单则可注释掉下述语句。
\listoftables

%%%%%%%%%%%%%%%%%%%%%%%%%%%%%%%%%%%%%%%%%%%%%%%%%%%%%%%%%%%%%%%%%%%%%%%%%%%%%%%
% 开始正文部分
\mainmatter

%%%%%%%%%%%%%%%%%%%%%%%%%%%%%%%%%%%%%%%%%%%%%%%%%%%%%%%%%%%%%%%%%%%%%%%%%%%%%%%
% 学位论文的正文应以《绪论》作为第一章,本模板是按照自身功能模块组织的,并非论文中的章节安排


\chapter{绪论}

\section{研究背景及意义}

区块链(Blockchain)是一种按照时间顺序将数据区块以链条的方式组成的特定数据结构,被视为一个分布式的共享账本和数据库。它能够使用户无需相互信任与可信第三方的条件下完成可信的价值传输\cite{SurveyofEnterpriseBlockchains}。作为区块链2.0 的以太坊\footnotemark[1]\footnotetext[1]{\href{http://github.com/ethereum/wiki/wiki/White-Paper/}{以太坊白皮书}}重新将智能合约描述为图灵完备的、部署于区块链网络中合同条款代码。这意味着传统的合同条款可以进入实体计算机中,且在区块链网络去中心化、不可伪造、不可篡改的特性下严格执行。区块链推进了人与企业之间和线上与线下之间的全方面互联,其被成为下一代的新型生产关系。随着区块链的快速发展, 智能合约极大地丰富和扩展了区块链应用场景。它们为供应链、金融等传统领域带来重大变革, 已经快速渗入到人们生活的方方面面。

当前, 区块链分为公有区块链、联盟区块链和私有区块链。如表\ref{blockchain_type}所示, 公有链没有准入限制, 没有监管方可以组织参与, 任何人都可以参与共识, 常见的两种共识协议为工作量证明机制(Proof of work, 简称PoW)和权益证明机制(Proof of stake, 简称PoS)。由于任何人都可以自由加入, 因此公有链网络具有高度分布式的拓扑结构。但是, 公有链在安全性和性能方面也进行了权衡。公有链上的许多服务器遇到了扩展瓶颈, 吞吐量相对较弱; 与公有区块链的无准入限制形成鲜明对比的是, 私有区块链建立了准入规则, 规定谁可以查看和写入区块链。因为在控制方面有明确的层次结构, 私有链也不是去中心化系统。在某些私有链中, 具备安全模型的背景下,共识协议是多余的。因此在私有区块链中,不使用PoW并不会造成很严重的威胁, 因为每个参与者的身份都是已知的, 是手动进行管理的; 联盟区块链是介于公有链和私有链之间的,结合了两者的特征要素。在共识方面, 联盟链将少数同等权力的参与方视为验证者,而不是像公有链那样开放的系统, 让任何人都可以验证区块, 也不是像私有链那样, 通过一个封闭的系统, 只允许某一个实体来任命区块的生产者。对于从事各类活动的个人和企业来说,存在大量的区块链选择。即使在公有链、私有链和联盟链中,根据复杂性的不同,也会出现许多不同的用户体验。根据实际使用情况,企业可以选择最适合实现自己目标的产品。供应链、电商、医疗等需要彼此之间需要相互沟通的场景下, 联盟链可减轻私有链中交易对手的风险, 并且较少的节点数通常可使它们能够比公共链更有效率的运行, 通常选择联盟链作为企业级区块链的底层。

{\footnotesize
\begin{longtable}[h]{m{70pt}|m{70pt}|m{70pt}|m{70pt}}
    \caption[区块链类型]{区块链类型} \label{blockchain_type} \\
        \hline  
        &公有区块链&私有区块链&联盟区块链\\
        \hline
        准入限制&无&有&有\\
        \hline
        读取者&任何人&仅限受邀用户&相关联用户\\
        \hline
        写入者&任何人&获批参与者&获批参与者\\
        \hline
        所属者&无&单一实体&多方实体\\
        \hline
        交易速度&慢&快&快\\
        \hline
    \end{longtable}
}

与此同时, 云原生(Cloud Native)作为一种基于云的基础之上的软件架构思想,以及基于云进行软件开发实践的一组方法论。因其弹性和分布式的优势成为当今流行的软件服务模式。区块链即服务(Blockchain as a Service, 简称BaaS)则是基于云的一种构建、管理、托管和运维区块链网络及其应用的云服务平台\cite{onik2019performance}。BaaS支持将任何企业级区块链实施到云环境,而无需任何IT专业知识。这大大降低了区块链技术的使用门槛, 是促使区块链技术更广泛、更深入地渗透到各个行业和企业的催化剂, 其市值预计从2018年的6.23亿美元猛增2023年的150亿美元\footnotemark[1]\footnotetext[1]{\href{https://www.reportbuyer.com/product/5486837/global-blockchain-as-a-service-market.html}{Global Blockchain-as-a-Service Market 2018-2022}}。其中, 云厂商提供了大多数的BaaS平台\cite{KuernetesbasedFabricChaincodeManagementAndHihgAvailabilityTechnology}, 如表\ref{major_BaaS_platforms}所示, 其底层区块链支撑技术大多数选择IBM开源的跨企业级联盟链Hyperledger Fabric, 这也是本文选择Hyperledger Fabric的原因。

{\footnotesize
\begin{longtable}[h]{m{150pt}|m{200pt}}
    \caption[主要公有云的BaaS平台]{主要公有云的BaaS平台} \label{major_BaaS_platforms} \\
        \hline  
        BaaS平台&区块链平台\\
        \hline
        AWS区块链服务&Hyperledger Fabric, Ethereum\\
        \hline
        Azure区块链服务&Ethereum\\
        \hline
        Google Cloud Platform&不支持\\
        \hline
        IBM区块链服务&Hyperledger Fabric\\
        \hline
        阿里云区块链服务&Hyperledger Fabric, 蚂蚁区块链, Ethereum\\
        \hline
        腾讯云区块链服务TBaaS&Hyperledger Fabric, FISCO BCOS, Tencent TrustSQL\\
        \hline
        华为云区块链服务BCS&Hyperledger Fabric\\
        \hline
    \end{longtable}
}

% 挑战
然而, 当前BaaS平台的发展尤其是底层的区块链基础设施的建设依旧存在诸多挑战。
第一, 基础商业化应用工具并不完善\footnotemark[1]\footnotetext[1]{\href{http://www.caict.ac.cn/kxyj/qwfb/ztbg/202107/P020210726503897354430.pdf}{区块链基础设施研究报告(2021年)}}。虽然市场上存在多种可选择的BaaS平台解决方案, 但是这些BaaS平台由商业巨头把控, 行业马太效应明显\cite{KuernetesbasedFabricChaincodeManagementAndHihgAvailabilityTechnology}。BaaS平台的构建需要专业的区块链以及云原生的技术能力, 只有实力雄厚的云厂商进行BaaS平台的研发, 这些BaaS服务往往与云计算节点捆绑, 用户租用BaaS服务计费高昂, 这并不利于中小型企业构建自身私有BaaS平台以及区块链应用。
第二, 现阶段BaaS平台利用云能力对区块链基础设施赋能乏力。对于BaaS平台市场规模的不断膨胀, 需要找到解决当前区块链网络部署、备份、升级以及数据存储以及可扩展性的方法。然而当前BaaS平台仅提供了一种基于开源区块链平台的一键化自动部署管理方案, 未深入到区块链与云基础设施的底层。这并不是有效的云化方式, 浪费了云原生技术的潜力。

% 意义、愿景
针对上述问题, 本文选择面向联盟链场景开源框架Hyperledger Fabric和Kubernetes构建BaaS平台的区块链基础设施。通过对Hyperledger Fabric的ca、peer、orderer组件进行抽象适配使其更原生的运行于Kubernetes。

本课题拟实现Hyperledger Fabric Operator能够对fabric网络进行主动的持续管理,包括故障转移、备份、升级和自动缩放,使用户通过专家提供的知识获得类似云的自我管理。具体一点就是,如果我们采用Blockchain Automation Framework发布了一个POD,然后我们在K8S中误删了POD,那么其是不会替我们自动重建POD的,这是该项目的的主要问题所在。

为解决上述挑战, 本文
Hyperledger Fabric Operator: 云原生时代下Fabric管理工具
利用Kubernetes,72\%的工程师在云原生生产中使用Kubernetes[3],可以方便的迁移到支持Kubernetes的任何云 
更简单、更原生的: 使用kubectl命令管理Fabric, 显示于k8s-dashboard;复用Kubernetes, API公共功能如CRUD、watch、内置认证
管理Fabric: 声明式自动化配置静态组件,如: CA、HF网络组件.命令式创建动态通道、链码

由linux基金会牵头,包括 IBM等30家初始企业成员共同成立的Hyperledger Fabric项目成为流行的面向联盟链场景开源框架之一。该项目定位是面向企业的分布式账本平台,引入权限管理,设计上支持可插拔、可扩展,自开源依赖广受欢迎,github上已经有12.7k star。然而云原生时代下,Fabric缺乏一个成熟的、一站式的解决方案来解决在云计算平台上构建区块链联盟链(私有链)。即区块链如何和云计算快速深度结合,利用云计算伸缩性、可移植性和高可用性价值来提供“高质量”的区块链服务,是当前仍然遗留的挑战。


为了解决上述挑战,本文提出了一种面向领域驱动设计的战术建模支持方法及工具,
具体工作内容如图\ref{workresult}所示。
战术建模方法包括一套标准化的\textbf{战术建模指南},
用于帮助理解不同模式的特征属性、
使用时机以及实现技术,作为领域建模的理论支撑指导建模过程;
根据战术建模指南,本文还基于UML的profile扩展机制实例化\textbf{战术建模语言},
用于规范化使用战术建模模式进行领域建模的流程;
基于所提出的战术建模方法,本文还实现了一种\textbf{战术建模支持工具},
支持可视化建模、验证建模结果以及存储和扩展建模结果,
为战术建模支持方法提供可视化方式展现。


\begin{figure}[h] %figure环境,h默认参数是可以浮动,不是固定在当前位置。如果要不浮动,你就可以使用大写float宏包的H参数,固定图片在当前位置,禁止浮动。
    \centering %使图片居中显示
    \includegraphics[width=0.8\textwidth]{FIGs/chapter1/workresult.pdf} %中括号中的参数是设置图片充满文档的大小,你也可以使用小数来缩小图片的尺寸。
    \caption{战术建方法及支持工具} %caption是用来给图片加上图题的
    \label{workresult} %这是添加标签,方便在文章中引用图片。
\end{figure}%figure环境



\section{国内外研究现状}

云原生背景下的区块链自诞生以来就受到了学术界与工业界的广泛关注。在区块链与云原生深度结合方面都进行了积极的探索。

% 学术
基于 Kubernetes 的 Fabric 链码管理及高可用技术[5]
(1)比较全面地研究了底层基础设施,尤其是生产环境下的高可用性 Kubernetes 平台
(2)设计并实现了Fabric在 Kubernetes上的云化部署,尤其是链码部分通过一个全新的容器控制插件实现了对 Kubernetes在代码级别上的支持,并完成了将链码纳入 Kubernetes环境管理的目标
云化的链码管理局限在链码层面, 无法实现对fabric网络进行主动的持续管理,包括故障转移、备份、升级和自动缩放

% 工业界
 Cello[6]: 一个区块链供应和操作系统,帮助人们以更高效的方式使用和管理区块链。Cello基于先进的区块链技术和现代PaaS工具,提供以下主要功能:
(1)管理区块链网络的生命周期
(2)支持自定义区块链网络配置,如网络大小、共识类型。
(3)支持多种底层基础设施,包括裸机、虚拟机、vSphere、本机Docker主机、swarm和Kubernetes
Cello当前阶段重点关注在Docker安装, 对Kubernetes operator支持方面的仍处在相对初级阶段,配置项简单灵活性不足且老旧,对CRD的最新更新在2年前;缺少对Operator的监控

Blockchain Automation Framework利用Ansible、Helm和Kubernetes部署生产DLT(分布式记账)网络。具体来说,使用helm charts作为Kubernetes部署模板、利用Ansible对网络进行配置。Blockchain Automation Framework框架目前支持Corda、Hyperledger Fabric、Hyperledger Indy和Quorum。然而,该项目采用的Ansible-Playbooks配置部署。本质上还是描述为一个需要希望远程主机执行命令的方案,或者一组IT程序运行的命令集合。虽然采用执行命令脚本的方式大大提升了自动化程度,但其远没有发挥k8s的潜力。

% 总结


领域驱动设计自提出以来,受到了学术界与工业界的广泛关注。
战略建模层次的实践与实现代码关联不紧密,更多关注的是顶层设计与架构搭建,
实施者一般也拥有较多的开发经验与较高的建模水平,所以在国内外都能较好地进行落地。
但战术建模层次更加侧重于建模的设计过程和具体实现,
对于建模实施者的水平有较高的要求,
还需要通过统一标准化的流程来达到建模结果的准确性,
规范化的建模语言也是支持战术建模成功实施的必要条件。所以,
一套完整的战术建模支持方法及工具就显得格外重要。

特定领域建模语言(Domain Specific Modeling Language)\cite{{frank2013domain}}
是一种专注于某个特定领域,结合了特定领域知识和概念的建模语言。
可以使用特定领域建模语言来进行战术建模,
从而获得统一标准化且具有特定领域特征的建模结果。
目前国内外研究人员对特定领域建模语言的定义方法了一些研究。
Hao Wu等人\cite{wu2018workflow}指出,
可以通过UML结合对象约束语言(Object Constraint Language,OCL)
的方式来扩充医疗系统领域现有元模型,形成一种更为完整的建模语言及方法。
Kühlwein等人\cite{kuhlwein2019firmware}通过一种分层架构的思想来打通物联网领域的多种元模型,
构建了一种平台型的特定领域建模语言,其中每层都有针对不同设备的元模型,
通过该平台进行组织和交互,这种定义方法需要依赖较为成熟的元模型;
Jumagaliyev等人\cite{jumagaliyev2019modelling}通过对云存储平台的通用存储类型进行分析,
抽象出数据类型的特征属性,创建元模型并使用EuGENia\cite{kolovos2010epsilon}进行注释,
最终借助Eclipse IDE将创建的元模型转换为具体的图形建模框架编辑器,
不仅定义了特定于云存储平台的元模型,还提供了建模工具编辑器。

国内外研究人员也对领域驱动设计建模语言的定义进行了一些研究。
Florian Rademacher等人\cite{rademacher2017towards}根据领域驱动设计实践经验提出使用UML profile
定义元模型,并将其应用到微服务领域中去。
Florian Rademacher等人调研了大量领域建模的UML图,
确定了可以描述领域模型的UML类图的构造方法。通过扩展元类的方式,
达到了使用领域驱动设计战术建模实现微服务系统建模的目的,
并为验证模型有效性和实现微服务代码奠定了基础。
同样的,Hippchen等人\cite{hippchen2019systematic}也强调了微服务化拆分中应用面向领域驱动设计建模语言的好处。
Florian Rademacher等人强调了领域驱动设计概念在UML中缺少正式的定义,阻碍了模型的验证和转化,
但同时UML又符合软件设计领域建模的要求,应用也十分广泛,故以UML为基础进行扩展,
定义出了一套新的针对领域驱动设计相关规则和约束的建模语言。
Andreas Diepenbrock等人\cite{diepenbrock2017ontology}提出使用本体论(Ontology)\cite{smith2003ontology}
来定义针对微服务应用的领域驱动设计元模型,该研究主要使用了本体论在特定领域表达知识语义的功能,
结合领域驱动设计实现了建模的目的。

以上工作说明任何特定领域建模语言的定义与提出都需要前期大量领域实践知识的总结,
并且在已有工作成果的基础上进行元模型的定义效率更高。
由于领域驱动设计战术建模理论较为成熟,
目前定义的战术模式特征明确,所以,
本文将基于领域驱动设计战术建模的现有研究成果,
结合实际应用情况,进行修改和优化,
定义一套新的领域驱动设计战术建模语言。


除了定义元模型来实现建模语言之外,提供支持建模语言的工具也十分重要。
刘辉等人\cite{刘辉2008元建模技术研究进展}提出元建模工具的实现主要分为两种途径。
第一种是使用配置文件扩展通用建模工具,让通用建模工具支持特定领域的元模型从而支持特定领域建模语言。
Guerriero等人\cite{guerriero2018streamgen}通过UML的profile扩展机制,
扩展了现有UML的语法元模型,达到创建新的元模型的目的,
扩展UML语法元模型的实现方式只依赖于配置文档,有利于多种建模方法的集成,
但需要借助统一建模语言的实现平台,如MetaEdit+\footnotemark[3]\footnotetext[3]{MetaEdit+首页:https://www.metacase.com/products.html}。
第二种是通过建模工具生成器根据元模型直接生成相应的建模工具,
La Fosse等人\cite{la2019towards}使用GEMOC(一种基于Eclipse的特定领域语言创建平台)
来创建扩展元模型,生成了新的建模语言及支持工具,
这种方式定制化程度更高,可以提供独立的工具,
但生成的建模工具依赖工具生成平台,
如EMF(Eclipse Modeling Framework)\footnotemark[4]\footnotetext[4]{Eclipse Modeling Framework主页:https://www.eclipse.org/modeling/emf/}
提供的一系列生成元模型的工具。

许多国内外的研究人员对领域驱动设计建模语言的支持工具开展了实践与探究工作。
Duc Minh Le等人\cite{le2018domain}定义了一种名为DCSL的基于注释的特定领域建模语言,
通过注释约束了领域模型的特征与行为,由于Java语言对注释的良好支持,
采用Java开发了一款建模语言支持工具,摆脱了建模工具的平台依赖性,
但注释对模型的表达不够直观,仅仅依靠注释来表达建模过程和结果远远不够;
Kapferer等人\cite{kapferer2020domain}定义了一种战略建模语言,
并提供了相应的编辑、验证和转换工具,重点关注上下文映射工作,
实现支持工具借助了PlantUML建模平台,
达到了可视化建模支持。


虽然目前已有一些关于领域驱动设计建模的探索与研究,但仍然存在诸多问题。
首先,许多建模语言的提出以元模型为基础,但元模型的定义缺少理论依据,
导致元模型的定义不够规范化和标准化;
其次,建模语言的提出即使调研过大量文献,
也缺少对学术界与工业界实际应用差别的思考,
导致最终定义的建模语言脱离实际应用场景;
最后,战术建模的支持工具易用性不够,或者依赖太多额外平台,学习和使用成本过高,
导致难以应用到实际生产中去。
总的来说,战术建模流程缺少一套标准化的建模支持方法及工具。

\section{本文主要研究工作}

本文主要的研究工作分为以下三个方面:

1.围绕领域驱动设计的战术建模过程展开了理论调研,
从《领域驱动设计:软件核心复杂性应对之道》\cite{DBLP:books/daglib/0013521}和
《实现领域驱动设计》\cite{vernon2013implementing}两本著作中抽取了八种战术建模模式及其重要特征。
具体地,针对八种战术建模模式,
设计了调查问卷,与工业界具有领域驱动设计实战经验的架构师和开发人员展开访谈;
根据访谈结果,通过多次焦点小组讨论,
对八种战术建模模式及其重要特征进行验证和完善,
克服了理论脱离实际的问题。
最终得出一套战术建模指南,该指南包括战术建模模式、模式属性、使用时机以及实现技术。


2.基于上述理论基础,通过UML profile机制扩展UML元类,实例化战术建模语言。
战术建模语言描述了战术模式的构造型、必要属性、关联关系以及重要约束。
以UML中元类为基础,更符合软件设计中面向对象(Object-Oriented)的思想,
也更易于软件从业者接受和学习。以该元模型为基础的建模语言,更关注战术建模,
包含最贴合实践的规则和约束,建模效率更高。

3.实现了一个战术建模支持工具,
该工具对建模过程中使用的战术模式进行约束与规范性校验,
对建模结果进行多种格式的转化与存储,
还包含生成框架项目代码包等扩展功能。
对战术建模支持工具进行了功能测试,并使用该工具进行了战术建模案例研究,
结果表明该工具支持开发人员快速理解各种战术模式的重要特征和规则约束,
降低了使用战术建模的学习成本;
可以对建模结果进行验证并提示开发人员进行修改,规范化建模过程;
还具有将建模结果转化为多种格式文件和框架项目代码的功能,使建模结果更具有通用性。


上述战术建模指南、战术建模语言以及战术建模支持工具共同组成了本文研究工作的战术建模支持方法及工具。

\section{本文组织结构}

本文组织结构如下:

第一章~绪论。介绍了本文的研究背景、国内外研究现状以及本文主要的研究工作;

第二章~理论与技术支持。介绍领域驱动设计尤其是战术建模设计相关理论和概念,对建模语言原理以及支持工具实现方法进行介绍;

第三章~面向领域驱动设计的战术建模支持方法。通过文献综述、访谈和焦点小组提出战术建模支持方法,
并详细介绍了该战术建模支持方法;

第四章~建模支持工具设计与实现。介绍本文如何基于第三章得到的战术建模方法实现一个建模支持工具,
介绍对该工具的需求分析,工具整体架构与各模块的设计和实现;

第五章~建模支持工具测试与案例研究。对建模支持工具进行功能测试、性能测试和案例分析研究,
介绍了使用建模支持工具建模的流程和效果;

第六章~总结与展望。总结本文所做的研究工作和贡献,分析工作不足之处,并对后续研究进行了展望。


\chapter{理论与技术支持}

本章将介绍领域驱动设计战略建模和战术建模相关概念,
为构建战术建模方法做理论支持;
还将介绍建模语言相关概念以及特定领域建模语言构建方法,
作为构建战术建模语言的参考;
最后,将介绍建模支持工具实现技术,
为实现支持工具做技术支撑。


\section{领域驱动设计}

领域驱动设计与传统的建模分析方法不同,领域驱动设计不再只关注业务数据,
而是从领域中的重要概念出发,将业务流程中关键对象提炼成领域模型与关键逻辑,
来解决复杂业务的本质问题。由于战术建模中的概念依赖于战略建模,
并存在于某个战略建模的限界上下文(Bounded Context)中,
所以战略建模是战术建模实施的支撑背景。
本节将从战略建模和战术建模两个层次进行介绍,
重点关注战术建模中的重要概念。

\subsection{战略建模相关概念}


战略建模是从宏观角度对领域业务进行建模的建模方法,强调领域内的业务特性。
战略建模可以划分出业务的边界、组织团队结构以及系统架构,
为后续的战术建模划定限界上下文。
使用战略建模可以很好地进行微服务化拆分\cite{DBLP:conf/icsa/MersonY20}与系统拆分。
下面将对战略建模相关重要概念进行介绍。


\textbf{通用语言(Ubiquitous Language)}

通用语言是Eric Evans在《领域驱动设计:软件核心复杂性应对之道》\cite{DBLP:books/daglib/0013521}
中提出的术语,用于开发人员和用户建立通用、无歧义的语言。
该语言的基础是软件设计中的领域模型,应严格按照领域模型进行定义,保证其严谨性。
本文提出的战术建模语言及支持工具都使用通用语言进行表达。

\textbf{子域(Subdomain)}

领域可以进一步划分为核心域(Core Domain)、通用域(General Domain)和支撑域(Support Domain)。
每一个子域对应更小的问题域或业务范围,根据其自身的功能属性进行不同划分。

核心域决定业务的核心竞争力,是最重要的子域,包含主要的业务流程。
通用域是被其他多个子域共同使用的部分,没有定制化需求,包含通用功能。
支撑域最关注业务,对应某个业务中的重要部分,具有业务特定性,在不同企业业务中不通用,但必不可少。
本文提出的战术建模支持方法目的在于解决某个子域的建模问题。

\textbf{限界上下文(Bounded Context)}

限界上下文是一种概念性边界,限定了领域模型的工作范围。每一个模型概念和其中的
属性与操作,在它所属的边界之内,都有特定的含义,通过之前约定的通用语言,
参与建模的团队成员应该可以明确领域模型的具体含义。
对于系统架构来说,限界上下文还确定了应用边界和技术边界,提供了解决特定领域问题的
建模明确边界,同时也充当了问题空间与解空间之间的桥梁。
本文提出的战术建模支持方法工作范围在某个限界上下文之内,
可以聚焦于特定的业务领域,使建模中的交流成本变低。

\textbf{上下文映射(Context Map)}

上下文映射在项目团队中共享,并确保被每个团队成员理解。通过上下文映射,可以从宏观上
看到每个上下文之间的关系,能够更好地指导后续的程序设计。
上下文映射不拘泥于任何形式的文档,
重点在于帮助团队成员理解不同上下文之间的关系。
在不同限界上下文使用本文提出的建模支持方法进行战术建模后,可以通过上下文映射进行建模结果的交流。

\textbf{防腐层(Anti-Corruption Layer)}

防腐层可以根据领域模型为自身所在的限界上下文提供服务。
该层通过与另一个系统进行通信,几乎不需要对自身限界上下文进行任何修改。
防腐层需要在两个模型之间进行必要的双向转换\cite{DBLP:books/daglib/0013521}。
防腐层不仅防止内部代码被外部逻辑侵入,还在于分离不同的领域并确保它们在将来保持分离。
通过防腐层可以在战术建模支持方法中体现战略建模的底层基础作用。

战略建模首先需要确定通用语言,将要解决的领域问题划分为更细粒度的子域,
并通过划定限界上下文隔离不同领域模型的工作范围;
然后通过通用语言进行建模,借助上下文映射来理解不同限界上下文之间的关系,
最后还可以通过防腐层来与外部系统进行沟通。


\subsection{战术建模相关概念}

本小节将描述领域驱动设计战术建模的相关概念,战术建模是本文研究工作的重点内容,
战术建模支持方法基于战术建模相关概念提出。
战术建模是在特定的限界上下文中进行的更细粒度的建模,
可以用来管理领域模型的复杂性并确保领域模型中行为的准确性。
通用语言是将团队沟通与软件实现紧密联系到一起的一种基于模型的语言。
战术模式是战术建模中常用的表达业务问题的一种方式,
包含领域中的抽象概念、关系和约束规则,
并将领域模型和通用语言中的概念映射到实现技术中。
战术建模依赖通用语言和战术模式,
如图\ref{DDD}所示为模型驱动设计战术模式关系图,
展示了主要的战术模式与它们之间的关系,下面将对战术模式相关概念进行详细介绍并举例说明。

\begin{figure}[h] %figure环境,h默认参数是可以浮动,不是固定在当前位置。如果要不浮动,你就可以使用大写float宏包的H参数,固定图片在当前位置,禁止浮动。
    \centering %使图片居中显示
    \includegraphics[width=0.8\textwidth]{FIGs/chapter2/DDD.pdf} %中括号中的参数是设置图片充满文档的大小,你也可以使用小数来缩小图片的尺寸。
    \caption{战术模式关系图\protect\footnotemark[1]} %caption是用来给图片加上图题的
    \label{DDD} %这是添加标签,方便在文章中引用图片。
\end{figure}%figure环境
\footnotetext[1]{图片来源于《领域驱动设计:软件核心复杂性应对之道》}

\textbf{实体(Entity)}

实体是一个由标识符定义的对象\cite{DBLP:books/daglib/0013521}。
当我们关注一个对象的个性特征,或者需要区分不同的对象时,我们使用实体这个概念,换句话说,
有必要将该对象与其他对象区别开来确保其个性特征时,我们希望将其建模成实体\cite{vernon2013implementing}。
例如,每辆汽车都通过车牌号与其他汽车进行区分,在这种条件下汽车应该被建模为实体。

\textbf{值对象(Value Object)}

值对象的作用是度量和描述事物,值对象可以很方便地进行创建、测试、使用、优化和维护。
值对象的使用比实体更加广泛,值对象一旦创建就不可变化,只可以进行替换,具有可比性\cite{vernon2013implementing}。
例如,一个人的家庭住址可以用来作为邮寄地址,当家庭住址改变时,可以直接将邮寄地址替换,
在这种条件下家庭住址应该被建模为值对象。

\textbf{领域服务(Domain Service)}

领域服务是一个独立接口,负责承担不属于实体或值对象职责的操作或转换过程。
领域服务可能关注一个显著的业务操作过程或领域对象的转换,
在单个原子操作中处理多个领域对象,并且命名要与通用语言保持一致\cite{vernon2013implementing}。
例如,保险公司在计算赔偿金额时需要进行复杂的计算流程,
而这种赔偿流程不是某个对象的职责,在这种条件下计算赔偿金额应该被建模为领域服务。

\textbf{领域事件(Domain Event)}

领域事件用来捕获发生在领域中的一些事情。领域事件记录了领域中已经发生的事情,无法改变。
需要维护业务一致性时,也需要用到领域事件,该事件往往是需要发布到外部系统(如外部限界上下文)的,
另外,领域事件还可以让远程依赖系统与本地系统保持一致\cite{vernon2013implementing}。
例如,在网购平台上订单已支付后,会触发商家发货等一系列流程,
在这种条件下订单已支付应该被建模为领域事件。

\textbf{聚合(Aggregate)}

聚合是一个将实体和值对象聚类到一致性边界内的容器。
聚合不仅仅是聚集了一些共享父类、密切关联的对象,更加重要的是其更关注内部的不变条件和整体的一致性边界\cite{vernon2013implementing}。
聚合根(Aggregate Root)是聚合用来标识自己的一个实体,需要追踪聚合变化时,就需要跟踪根实体,
根实体也是代表聚合与外界交互的对象。
例如,汽车是一种很复杂的机器,其中包含发动机、车轮、制动器等元件,
这些元件统一协同通信才组成了整个汽车,在这种条件下,
应该将这些元件组合到一起建模为聚合。

\textbf{资源库(Repository)}

资源库是一个用来存取领域对象的安全存储区域\cite{vernon2013implementing}。
可以通过资源库来安全地存储领域对象,并在需要时取出使用,资源库可以依据不同实现形式来
存取不同的领域对象。

\textbf{工厂(Factory)}

工厂是一种具有创建复杂对象和聚合职责的单独对象,该对象并不承担领域模型中的职责,
但是依然是领域驱动设计的一部分\cite{vernon2013implementing}。
创建复杂对象和聚合的逻辑经常封装在工厂中,与设计模式中的工厂相似,
但不应该为创建每个对象都提供一个工厂。
例如,组装汽车是一个复杂的过程,但最终产出仍然还是汽车,
在这种条件下,应该将创建汽车的工作建模成工厂。


\textbf{模块(Module)}

模块是一个命名的容器,用于存放内聚的类,
并可以对不在同一个模块的类解耦\cite{vernon2013implementing}。
主要应用在技术层面,如代码组织,构建目录与打包。
例如,汽车、飞机、轮船等交通工具在进行建模时,
可以统一放在交通工具模块中。

\section{建模语言}

战术建模与实现层次关联较强,甚至有些团队直接采用代码形式来进行建模,用代码来表达领域模型,
但这种方式不易于理解,在大型团队中无法很好地发挥领域模型的沟通作用。
建模语言是一种描述信息或者数据模型概念的语言,可以作为模型与实现之间沟通的桥梁,
UML就是一种统一建模语言。
构建元模型是实现建模语言的一种方式,
元模型可以作为交换和存储数据的介质或支持特定方法的语言,
UML也是通过元模型进行实现的建模语言。
通过对象约束语言对UML元模型的额外约束,
可以构建一种特定领域建模语言。
战术建模这一特定领域长期以来缺少标准、规范的建模语言来对建模过程进行规范化和约束,
随着战术建模应用越来越广泛,需要强大且合适的建模语言来进行支撑。

元模型(Metamodel)是描述模型的模型,在软件工程中使用模型来作为表达方式越来越普遍。
模型的建立背后应该对应着一个元模型,模型是真实世界中现象的抽象,
元模型是关注模型本身属性的一种抽象,所以可以把一个元模型看做对模型的抽象。
元模型可以充当交换或存储语义数据的中间介质,
可以作为支持特定方法或过程的语言,还可以作为扩展现有信息额外含义的语义语言\cite{emerson2006techniques}。
任何模型都应该服从其元模型的定义。
目前模型驱动工程(Model Driven Engineering,MDE)最活跃的分支是
Object Management Group(OMG)\footnotemark[2]\footnotetext[2]{OMG组织官网:https://www.omg.org/}
提出的模型驱动架构(Model Driven Architecture,MDA)解决方案\cite{soley2000model}。
该解决方案描述了被称为元对象机制(Meta-Object Facility,MOF)的元模型结构。
本文将采用元模型的方式来定义战术建模模型,战术建模语言也将以元模型为基础,
元模型是战术建模支持方法中建模语言的组成元素。


面向对象建模方法是一种依靠现实生活中常用思维来认识、理解和描述客观事物的建模方法,
强调最终建模的对象之间反映现实中的固有问题和关系。
面向对象分析与设计方法的发展在80年代末90年代初出现了一个高潮,由此出现了许多面向对象的建模方法。
其中具有代表性的如Yourdan等人\cite{coad1991object}提出的面向对象分析(Object-Oriented Analysis,OOA)方法,
Jacobson等人\cite{jacobson1995use}提出的面向对象软件工程(Object-Oriented Software Engineering,OOSE)方法等。
UML在此次高潮后,作为一种统一的建模语言产生。UML是一种建模语言,而不是方法,它不包含对过程的描述,
同时,UML也是OMG提出的典型元模型之一。

UML关注建模的普适性和可移植性,
直接使用UML进行战术建模粒度不够,
无法反映战术建模的一些模式及其规则和约束,会造成建模细节的损失。
特定领域建模语言(Domain Specific Modeling Language)\cite{{frank2013domain}}
是一种专注于某个特定领域,结合了特定领域知识和概念的建模语言。
分析领域的建模人员不必从头开始构建这些概念,
可以直接使用具有特定领域知识和概念的建模语言来进行建模,
所以,特定领域建模语言更加适合作为战术建模的基础语言。
本文构建特定于领域驱动设计战术建模的建模语言,包含战术建模相关概念,
使用该建模语言进行建模与战术建模目的相符。
同时UML可以适配任何适用于自己项目类型的过程,
并记录最终的分析和设计结果\cite{王文玲1999uml}。
因此,可以使用UML来作为建模结果的描述语言,
使建模结果具有通用性和普遍性,UML的profile扩展机制,
也提供了通过元模型(Metamodel)来确定构造型、约束和特定语义的方法,从而支持特定领域的建模过程。
针对于领域驱动设计战术建模,UML元类可以通过profile扩展机制,表示特定的模式,
如实体(Entity)、值对象(Value Object)等\cite{pahl2016microservices}。
本文将使用具有面向对象特征的UML profile扩展机制来作为元模型的实现方式,
既符合领域驱动设计的领域特征,又能扩充UML元模型构建新的建模语言。

对象约束语言是一种施加在指定模型元素上的约束语言。
对象约束语言最早是在1995年在IBM\footnotemark[3]\footnotetext[3]{IBM公司主页:https://www.ibm.com/}内部开发的,
1997年被集成到UML标准中,最初的作用是对UML表示法的补充,
OCL(Object Constraint Language)表达式能以附加在模型元素上的条件和限制来表现对该对象的约束,
克服了UML表示法甚至是任何图形表示法在系统详细设计时不够精确的局限性,
OCL充当了模型驱动工程(Model-Driven Engineering,MDE)技术的关键组成部分,
成为各种元模型查询,操作和制定规格要求的默认语言\cite{cabot2012object}。 
UML图表示法不够完善,无法规范化所有约束和属性,如果直接使用自然语言来表示,
会造成歧义,OCL也很好地填补了这一部分的空白。
本文提出的元模型经过OCL的附加约束,形成了完整的一套建模语言。

\section{特定领域建模语言构建方法}


本小节将介绍提出和验证特定领域建模语言的方法,
具体方法过程如图\ref{DSMLmethod}所示。
首先对特定的领域进行分析,得出基础理论知识;
然后根据知识设计元模型,进行细化;
将创建的元模型扩展为建模语言和工具,实现建模语言;
最后,对建模语言进行使用,验证效果\cite{sobernig2016extracting}。


\begin{figure}[h] %figure环境,h默认参数是可以浮动,不是固定在当前位置。如果要不浮动,你就可以使用大写float宏包的H参数,固定图片在当前位置,禁止浮动。
    \centering %使图片居中显示
    \includegraphics[width=0.8\textwidth]{FIGs/chapter3/DSMLmethod.pdf} %中括号中的参数是设置图片充满文档的大小,你也可以使用小数来缩小图片的尺寸。
    \caption{特定领域建模语言提出方法} %caption是用来给图片加上图题的
    \label{DSMLmethod} %这是添加标签,方便在文章中引用图片。
\end{figure}%figure环境

1)领域分析。这一阶段可以分为两种形式。第一种形式是由领域专家直接负责或参与,
由于领域专家拥有特定领域的丰富领域知识与实践经验,可以利用现有知识对特定领域重要概念进行提取,
抽象出一个初始的元模型,这种方法效率较高,适合于小范围领域,但最终建模语言质量依赖于领域专家的个人水平。
第二种形式是领域专家间接参与或不参与,这种形式需要通过阅读特定领域自然语言文献、对领域专家进行访谈或直接基于前人工作进行扩展,
得出基础理论知识或初始元模型,这种方法周期较长,适合比较成熟的领域。

2)设计元模型。这一阶段也可以在领域分析阶段直接完成。如果仅得到特定领域理论知识,需要使用理论知识构建元模型。
目前元模型的构建主要遵循的规则是元对象机制(Meta-Object Facility,MOF)定义的四层架构,UML也是MOF的一个实例\cite{bezivin2004search}。
如图\ref{MOF4}展示了MOF四层架构与UML类图之间的对应关系。M3元元模型层(Meta-Metamodel)包含了定义建模语言所需的元素,
作为整个框架的最高层,定义了最基本的元类(Metaclass),并且是自描述(Self-Descriptive)的;
M2元模型层(Metamodel)是M3元元模型的实例,定义了M1模型层需要使用的元素,如在类图中定义了“Operation”、“Class”、“Property”等元信息;
M1模型层(Model)是M2元模型层的具体实例,是用户根据M2元模型层定义所创建的具体模型实例,如类图中定义的具体类,以及该类包含的具体属性和操作;
M0实例层(Instance)是M1模型层的具体实例,是用户真正使用的实例对象,如根据类图创建的具体对象。

\begin{figure}[h] %figure环境,h默认参数是可以浮动,不是固定在当前位置。如果要不浮动,你就可以使用大写float宏包的H参数,固定图片在当前位置,禁止浮动。
    \centering %使图片居中显示
    \includegraphics[width=0.8\textwidth]{FIGs/chapter3/MOF4.pdf} %中括号中的参数是设置图片充满文档的大小,你也可以使用小数来缩小图片的尺寸。
    \caption{MOF四层架构与UML类图\protect\footnotemark[4]} %caption是用来给图片加上图题的
    \label{MOF4} %这是添加标签,方便在文章中引用图片。
\end{figure}%figure环境
\footnotetext[4]{图片来源于OMG Unified Modeling Language (OMG UML), Superstructure}

在MOF四层架构与UML类图对应的基础上,进一步细化元模型设计,将重要领域概念与元模型元素对应起来。
UML中使用profile对M2元模型层进行扩展定义,提供了三种扩展方式:
构造型(Stereotype)是一个配置文件,可以通过构造型从现有的模型元素中派生出新的模型,
新的模型通常具有更适合特定领域的属性;
标记值(Tagged Values)用于扩展UML属性,可以在模型元素的规则中添加额外信息,
允许使用键值对的方式以字符串形式呈现,标记在模型元素下方,通常强调模型的版本控制,著作权等信息;
约束(Constraints)用于指定模型中必须始终满足的语义或条件,约束可以显示为字符串,
并包含在关联元素附近的方括号中。通常来限制模型中属性的取值范围。如图\ref{UMLprofileexample}
展示了UML profile扩展类图的一个实例。
构造型<<Machine>>表示图中Car类为扩展的机器类型,
标记值表示了该Car类的版本号,约束保证了该Car类的速度数值限制在0到300之间。

\begin{figure}[!htbp] %figure环境,h默认参数是可以浮动,不是固定在当前位置。如果要不浮动,你就可以使用大写float宏包的H参数,固定图片在当前位置,禁止浮动。
    \centering %使图片居中显示
    \includegraphics[width=0.8\textwidth]{FIGs/chapter3/UMLprofileexample.pdf} %中括号中的参数是设置图片充满文档的大小,你也可以使用小数来缩小图片的尺寸。
    \caption{UML profile扩展类图实例} %caption是用来给图片加上图题的
    \label{UMLprofileexample} %这是添加标签,方便在文章中引用图片。
\end{figure}%figure环境

3)实现。将设计完成的元模型结合理论知识运用到生产中,是这一阶段的主要任务。
建模工具的实施方式分为两种,如图\ref{2kindsmodeling}展示了两种不同实施工具的方式。
第一种方式通过扩展通用建模工具实施,这种方式依赖现有的支持通用建模语言的工具或平台,
将特定领域的建模语言配置文件导入,来支持新的建模语言。该方式有利于多种建模语言的集成,
但需要统一建模语言平台提供扩展功能。第二种方式通过建模工具生成器根据元模型生成相应的建模工具,
这种方式可以给用户提供一个独立的工具,有利于对建模工具的定制、修改和优化,
但添加额外功能困难, 依赖于工具生成器的成熟度\cite{刘辉2008元建模技术研究进展}。


\begin{figure}[!htbp] %figure环境,h默认参数是可以浮动,不是固定在当前位置。如果要不浮动,你就可以使用大写float宏包的H参数,固定图片在当前位置,禁止浮动。
    \centering %使图片居中显示
    \includegraphics[width=0.8\textwidth]{FIGs/chapter3/2kindsmodeling.pdf} %中括号中的参数是设置图片充满文档的大小,你也可以使用小数来缩小图片的尺寸。
    \caption{两种建模工具实施方式} %caption是用来给图片加上图题的
    \label{2kindsmodeling} %这是添加标签,方便在文章中引用图片。
\end{figure}%figure环境

4)验证。针对最终形成的特定领域建模语言,还可以进行使用效果的验证。包含定性和定量的分析方法,
定性分析主要关注建模语言使用者的感受,从易用性、可靠性以及正确性等方面进行考察;
定量分析主要关注使用新建模语言的效率,从建模各阶段耗费时间进行统计分析\cite{kardas2018domain}。


\section{建模支持工具实现技术}

本小节将对建模支持工具实现技术进行介绍。
本文实现的建模支持工具所需技术包括可视化建模展示所用的mxGraph,
前端界面构建所用的Vue.js和ElementUI;
后端服务器所需技术包括Spring Boot和ElasticSearch。


\textbf{mxGraph}

mxGraph\footnotemark[5]\footnotetext[5]{mxGraph项目地址:https://github.com/jgraph/mxgraph}
是一个使用SVG和HTML进行渲染的JavaScript图表库客户端,
支持交互式图形和图表的快速创建,在主流的浏览器中都能良好运行。
支持在网页中设计和编辑工作流图、流程图、网络图和各种图表。
mxGraph库不使用任何第三方软件,不需要任何插件,并且可以集成在几乎任何框架中。
mxGraph拥有完善的绘图功能,许多大型产品底层都使用其进行绘图和渲染。
不仅拥有JavaScript源码版本,还有Java源码版本,拥有便于移植和适配性强等特点,
还支持以XML或图片形式进行展示等功能。
本文使用mxGraph作为可视化建模的支持技术,通过在网页中运行mxGraph,
来达到拖拽绘制模型、点击连线、双击修改文字以及导出建模结果等功能。

\textbf{Vue.js}

Vue.js\footnotemark[6]\footnotetext[6]{Vue 官网https://cn.vuejs.org/}是一套用于构建用户界面的渐进式框架。
与其它大型框架不同,Vue.js可以自底向上逐层应用。
Vue.js的核心库只关注视图层,能方便地获取数据更新,并通过组件内部特定的方法实现视图与模型的交互。
此外,Vue.js容易上手,与第三方库或现有项目进行整合时方便,
与主流的开发工具链以及各种支持类库结合使用时,Vue.js也完全能够为复杂的单页应用提供驱动。
得益于响应式设计,Vue.js获取数据更新十分方便。
响应式是指MVC模型中的视图随着模型变化而变化。
在Vue.js中,开发者只需将视图与对应的模型进行绑定,Vue.js便能自动观测模型的变动,并重绘视图,
这一特性使得Vue.js的状态管理变得相当简单直观。
Vue.js构建的界面简洁美观,交互方式易于理解,交互响应速度快,本文使用Vue.js来实现前端的用户界面展示与交互。

组件(Componet)也是Vue.js的一大基础功能。组件可以被复用,封装可多次重用的代码逻辑,
复用的组件能更加方便地进行管理、修改和优化。
组件还可以以不同形式(如数组、树)组织起来,几乎任意类型的应用的界面都可以抽象为一个组件树,
从而构建出大型复杂的应用。

如图\ref{vue}所示描述了响应式设计追踪变化的过程。
当一个普通的JavaScript对象传入Vue实例作为data选项时,
Vue将遍历此对象所有的property,并使用Object.defineProperty把这些property全部转为getter/setter。
这些getter/setter对用户来说是不可见的,但是在内部它们让Vue能够追踪依赖,在property被访问和修改时通知变更。
每个组件实例都对应一个watcher实例,它会在组件渲染的过程中把“接触”过的数据property记录为依赖。
之后当依赖项的setter触发时,会通知watcher,从而使它关联的组件重新渲染。

\begin{figure}[h] %figure环境,h默认参数是可以浮动,不是固定在当前位置。如果要不浮动,你就可以使用大写float宏包的H参数,固定图片在当前位置,禁止浮动。
    \centering %使图片居中显示
    \includegraphics[width=0.8\textwidth]{FIGs/chapter2/vue.pdf} %中括号中的参数是设置图片充满文档的大小,你也可以使用小数来缩小图片的尺寸。
    \caption{响应式设计追踪变化} %caption是用来给图片加上图题的
    \label{vue} %这是添加标签,方便在文章中引用图片。
\end{figure}%figure环境

\textbf{ElementUI}

ElementUI\footnotemark[7]\footnotetext[7]{ElementUI主页:https://element.eleme.io}
是一套基于Vue2.0的桌面端组件库,包含了大量的页面设计组件与资源,
秉承了一致、反馈、效率、可控的设计原则,帮助使用者快速搭建网站前端交互界面。
ElementUI强调界面元素一致及用户界面和生活场景一致,做到清晰的页面反馈和控制反馈,
提供的组件表述清晰直白,使用方式简单高效,用户可以自由准确地使用这些组件进行自主性操作。
本文使用ElementUI对前端组件进行统一管理,对数据和组件进行绑定和交互。

\textbf{Spring Boot}

Spring Boot\footnotemark[8]\footnotetext[8]{SpringBoot主页:https://spring.io/projects/spring-boot}
是为了简化Spring应用搭建以及开发过程的一种框架。
Spring Boot提供起步依赖和自动依赖管理,
以注解形式进行配置,
代替了不易阅读的XML格式配置文件。
集成的Web服务器不需要再次对网络请求进行复杂处理,可以快速构建后端服务器。
本文使用Spring Boot开发后端服务,以服务器形式为前端提供接口服务。

\textbf{ElasticSearch}

ElasticSearch\footnotemark[9]\footnotetext[9]{ElasticSearch主页:https://www.elastic.co/cn/}
是一个基于Lucene\cite{bialecki2012apache}库的搜索引擎。
ElasticSearch可以用于搜索各种文档,具有接近实时的搜索效果,可以通过JSON和Java API提供服务,
适合含有大量数据的文档搜索与存储。本文项目中的模型结果以文档形式存储在ElasticSearch中,
能够支持快速检索。

\section{本章小结}

本章主要介绍了与本文工作相关的理论与支持技术。
首先对领域驱动设计中的重要概念进行解释,主要介绍了战术建模包括的重要模式和一些相关概念;
然后对建模语言进行了介绍,
重点解释了元模型,UML profile机制以及OCL等支持构建建模语言的方式;
还介绍了构建特定领域建模语言的方法和具体过程;
最后,介绍了支持工具需要用到的技术,
包括可视化技术mxGraph,前端支持工具Vue.js和ElementUI,以及后端技术Spring Boot和ElasticSearch。







\chapter{区块链云化框架调研分析}

本章首先通过快速评审获得Kubernetes Operator赋能质量属性的策略集,  随后阐述区块链云化框架的设计原则, 最后结合区块链去中心化等特性以及设计原则分析筛选策略集形成面向Hyperledger Fabric的区块链云化框架的核心具体实施方案。

\section{快速评审}\label{section: rapid_reviews}

软件工程中, 快速评审(Rapid reviews)是一种轻量级的二级研究, 以实践为导向专注于及时的向研究人员提供证据\cite{cartaxo2020rapid}。本文对已发表的学术文章进行快速评审, 快速评审的目的是为了在较短时间内了解目前学术界使用Kubernetes Operator进行云化的现状, 以及如何使用Kubernetes对现有系统进行赋能。随后, 对应文献对快速评审所得到的结果进行整理归纳得到Kubernetes Operator如何为质量属性赋能的策略集。

本文的快速评审过程分为以下步骤: 首先进行自动化的全文数据库检索, 再筛选出与Kubernetes Operator及架构改造强相关的文献, 接下来提取出本文所关注的质量属性及相关策略, 最后归纳整理出策略集。

{\footnotesize
\begin{longtable}[h]{m{60pt} m{210pt} m{40pt}<{\centering}}
    \caption[每个全文数据库的搜索字符串]{每个全文数据库的搜索字符串} \label{search_string} \\
        \toprule  
        \textbf{全文数据库}&\textbf{搜索字符串}&\textbf{文献数}\\
        \hline
        IEEE Xplore &(“kubernetes AND operator”) OR (“k8s AND operator”) OR (“custom resource defination”) & 23 \\

        ACM & “kubernetes operator” OR “k8s operator” & 7 \\

        Springer &((“kubernetes AND operator”) OR (“k8s AND operator”))AND ((“custom resource defination”) OR "CRD") & 0 \\

        ScienceDirect &(“kubernetes AND operator”) OR (“k8s AND operator”) OR (“custom resource defination”) & 0 \\
        \hline
        Scoups\&Google &“kubernetes operator”& 21(去重) \\
        \bottomrule
    \end{longtable}
}

如表\ref{search_string}所示, 为全面获取学术届对基于Kubernetes Operator云化的策略, 确定了本次检索的全文数据库以及搜索字符串。本次检索主要针对计算机与软件工程领域的全文数据库\cite{lisboa2010systematic}(包含ACM、IEEE Xplore、Springer、ScienceDirect), 同时检索Scoups以及谷歌学术进行补充。最终, 围绕“kubernetes operator”为检索主题得到的共51篇论文。在文献筛选阶段, 本文根据筛选标准筛选出15篇与Kubernetes Operator云化强相关的文献, 入选文献如表\ref{rapid_reviews}所示。文献筛选标准主要有:

\begin{itemize}[itemindent=2em]
    \item 论文的主要目的是利用Kubernetes对原有系统进行云化;

    \item 论文针对于Kubernetes Operator方法;

    \item 论文介绍了具体的改造策略及相关的质量属性。
\end{itemize}

{\footnotesize
\begin{longtable}[h]{m{40pt} m{280pt} m{40pt}<{\centering}}
    \caption[快速评审入选文献列表]{快速评审入选文献列表} \label{rapid_reviews} \\
        \toprule  
        \textbf{文献编号}&\textbf{文献标题}&\textbf{文献引用}\\
        \hline
        [P1]&Enhancement of observability using Kubernetes operator&\cite{Shenoy2022496}\\
        
        [P2]&Designing a Kubernetes Operator for Machine Learning Applications&\cite{kanso2021designing}\\
        
        [P3]&Container orchestration on HPC systems through Kubernetes&\cite{zhou2021container}\\
        
        [P4]&Validation and Benchmarking of CNFs in OSM for pure Cloud Native applications in 5G and beyond&\cite{pino2021validation}\\
        
        [P5]&On-the-fly fusion of remotely-sensed big data using an elastic computing paradigm with a containerized spark engine on kubernetes&\cite{huang2021fly}\\
        
        [P6]&A Role-Based Orchestration Approach for Cloud Applications&\cite{yue2021role}\\
        
        [P7]&A Design of MANO System for Cloud Native Infrastructure&\cite{lee2021design}\\
        
        [P8]&Dynamic Updates of Virtual PLCs Deployed as Kubernetes Microservices&\cite{koziolek2021dynamic}\\

        [P9]&Suture: Stitching safety onto kubernetes operators&\cite{mahajan2020suture}\\
        
        [P10]&Automation of virtualized 5G infrastructure using mosaic 5G operator over kubernetes supporting network slicing&\cite{wiranata2020automation}\\
        
        [P11]&5G Cloud-Native: Network Management \& Automation&\cite{arouk20205g}\\
        
        [P12]&Proposed model for distributed storage automation system using kubernetes operators&\cite{sharma2020proposed}\\
      
        [P13]&Monitoring Resilience in a Rook-managed Containerized Cloud Storage System&\cite{baumann2019monitoring}\\
        
        [P14]&Reproducible Benchmarking of Cloud-Native Applications With the Kubernetes Operator Pattern&\cite{henning2021reproducible}\\
        
        [P15]&Pivotal Greenplum©for Kubernetes: Demonstration of Managing Greenplum Database on Kubernetes&\cite{patel2019pivotal}\\
        \bottomrule
    \end{longtable}
}

在数据提取阶段, 本文对5个提取项(文献标题、文献发表年份、质量属性、策略、所属领域)进行抽取, 共得到44条针对不同质量属性的相关策略。在数据归纳阶段, 由于不同的文献中对语义相同的质量属性用词存在明显差异, 本文根据国际软件质量评价标准ISO/IEC 25010:2011\footnotemark[1]\footnotetext[1]{\href{https://www.iso.org/standard/35733.html}{ISO/IEC 25010:2011 System and software quality models}}所定义的质量模型对收集到的文献中表述的质量属性进行映射, 同时对相同或相似的策略进行整理合并, 得到策略集如表所示。

{\footnotesize
\begin{longtable}[h]{m{40pt}|m{40pt}|m{20pt}|m{150pt}|m{80pt}}
    \caption[基于Kubernetes Operator云化策略集]{基于Kubernetes Operator云化策略集} \label{policy_set} \\  
        \hline
        \textbf{文献中质量属性}&\textbf{ISO质量属性}&\textbf{编号}&\textbf{策略}&\textbf{参考文献}\\
        \hline
        \multirow{4}*{\parbox[c]{40pt}{生产效率 \\ 效率}} & \multirow{4}*{易用性}
        &S1&容器化及Kubernetes能力 & [P3] \\\cline{3-5}
        & &S2&自动化配置复杂领域知识 & [P1, P2, P12-15] \\\cline{3-5}
        & &S3&自动化构建、部署应用程序 & [P6, P10, P11] \\\cline{3-5}
        & &S4&operator与helm结合 & [P1, P4, P13] \\\cline{3-5}

        \hline
        可迁移性 & 适应性
        &S1&容器化及Kubernetes能力 & [P1-P3, P8, P10-P12] \\\cline{3-5}

        \hline
        \multirow{2}*{可用性} & \multirow{2}*{可靠性}
        &S1&容器化及Kubernetes能力 & [P2, P15] \\\cline{3-5}
        & &S5&主备切换 & [P15] \\\cline{3-5}

        \hline
        \multirow{3}*{\parbox[c]{40pt}{可扩展性 \\ 可伸缩性 \\ 灵活性}} & \multirow{3}*{无}
        &S1&容器化及Kubernetes能力 & [P2, P3, P5, P7, P12, P15] \\\cline{3-5}
        & &S6&基于监控指标并进行伸缩处理 & [P2, P6, P13] \\\cline{3-5}
        & &S7&链外利用存储即服务 & [P15] \\\cline{3-5}

        \hline
        \multirow{4}*{安全性} & \multirow{4}*{安全性}
        &S8&RBAC & [P2] \\\cline{3-5}
        & &S9&自定义访问控制机制 & [P9] \\\cline{3-5}
        & &S10&多用户认证授权机制 & [P3, P13] \\\cline{3-5}
        & &S11&资源隔离 & [P13] \\\cline{3-5}

        \hline
        \multirow{1}*{可监控性} & \multirow{1}*{无}
        &S12&基于Prometheus的监控体系 & [P1, P5, P12, P13] \\\cline{3-5}
        \hline
    \end{longtable} 
}

\section{设计原则}\label{section: framework_characteristics}

区块链云化框架致力于在BaaS一站式构建、管理、托管和运维区块链网络及其应用的基础上更深入云原生底层基础设施Kubernetes的底层, 有效利用云能力对HF基础设施赋能。BaaS在设计上以简单易用、成熟可扩展、安全可靠、可视运维等为主要方向\footnotemark[1]\footnotetext[1]{\href{http://www.caict.ac.cn/kxyj/qwfb/bps/201901/P020190111409291959654.pdf}{2019区块链即服务平台BaaS白皮书}}。区块链云化框架需要在上述设计原则的基础上进行适配与拓展:

\begin{itemize}[itemindent=2em]
    \item 简单易用: BaaS平台需要具备帮助企业实现自动配置, 快速启动区块链的能力。区块链云化框架需要在Kubernetes上完成HF的全生命周期管理。在Kubernetes上启动管理HF网络并部署链码需要专业的领域知识, HF各节点的配置项繁多且与Kubernetes适配极易出错。区块链云化框架需屏蔽底层HF配置与Kubernetes逻辑, 帮助用户采用命令行配置的方式提供完整的HF各组件的全生命周期管理;

    \item 灵活扩展: BaaS平台设计应采用抽象架构以及可插拔的模块。在此原则上, 区块链云化框架需按照模块化配置, 将Kubernetes底层的计算资源、存储资源、网络资源等供给HF, HF的Ca、Peer、Orderer基本组件的证书认证、共识、TLS加密、存储等功能模块作为可配置项进行命令式配置; 该框架能够确保系统核心底层逻辑稳定运行的同时, 对外提供小而精的扩展边界, 实现系统的高内聚与低耦合;

    \item 安全可靠: BaaS平台应具备有效的安全机制保证区块链服务的安全性。 区块链云化框架为满足上述原则, 需要具备有效的认证、鉴权、准入机制来确保区块链系统的安全性; 具备可插拔的共识算法取保区块链的去中心化、可追溯等特点, 支持完善的用户密钥授权、保存、隔离处理以及提供可靠的故障恢复能力;

    \item 可视运维: BaaS平台要提供必要的运维接口和运维授权的能力。当前区块链系统缺乏一套涵盖不同层面的标准方法来监控区块链及其智能合约的运行\cite{zhangfuli2021smartcontract}。 区块链云化框架需提供基本运维能力, 有效复用云上监控方案为区块链系统提供7*24小时可视化资源监控能力;

    \item 云链结合: BaaS平台需要结合云平台提供各种区块链所需的丰富资源。在此原则之上, 区块链云化框架需要具备高可迁移性, 具备基础架构的云独立性。本框架依托Kubernetes, 可以方便的迁移到支持Kubernetes的任何云;

    \item 合作开放: BaaS平台应该和各个行业伙伴共同打造行业可信的区块链生态。为此, 区块链云化框架应该具备反馈机制的透明性。区块链云化框架需保证区块链上所有的交易记录是可追溯的、不可篡改的, 区块链交易过程以及获取交易产生的记录需要专业人员才能获取操作, 非专业人员难以理解, 框架需通过简单透明的方式获取交易记录。
\end{itemize}

\section{策略集应用}\label{section: policy_set_application}

通过快速评审的方式形成的基于Kubernetes Operator云化策略集作为可选指导方案, 需要贴合区块链领域自身约束才能有效纳入区块链云化框架中落地。本节将围“所选策略为何能提升质量属性”以及“所选策略是否适用于HF”进行阐述。

容器化在云原生应用程序中的拥有更高的部署效率\cite{zhou2021container}、可迁移性。容器化可以使每个软件应用程序在隔离的环境中运行, 其将应用程序及其库、配置文件和其他依赖项封装在一起, 确保了环境兼容性, 从而使用户能够轻松地在不同的环境中移动和部署程序。因此, 容器化使得应用程序具备极强的可迁移性。Docker\footnotemark[1]\footnotetext[1]{\href{https://www.docker.com/}{Docker官网}}是容器化的典型代表, 其可以将运行态的容器打包成镜像(image)并储存在在线存储仓库中。Docker容器不是虚拟机, 这意味着它可以使用宿主机现有的网络接口\cite{shah2019building}。一旦创建了容器, 就可以为容器提供一个专用环回接口, 用于内部通信, 完成部署。

Kubernetes作为容器管理调度平台, 其具备强大的调度、伸缩和自恢复能力。Kubernetes会自动打包应用程序, 并根据预设需求运行Pod并管理容器。利用自身机制, Kubernetes可以轻松的将Pod进行横向扩容(Horizontal Pod Autoscaler, 简称HPA)或纵向扩容(Vertical Pod Autoscaler, 简称VPA)。Kubernetes提供了强大的容器自恢复能力, 可以自动重启在执行过程中失败的容器, 并杀死那些没有响应用户定义的健康检查的容器。但如果节点本身死亡, 那么它会替换并重新安排发生故障的容器到其他可用节点上。

HF官方提供了Binaries和Docker镜像两种安装方式\footnotemark[1]\footnotetext[1]{\href{https://github.com/hyperledger/fabric/blob/main/scripts/bootstrap.sh}{fabric\_bootstrap.sh}}, 其天然的满足容器化构建。一旦Fabric以镜像的方式托管于Kubernetes, 即可原生的利用Kubernetes进行高可用的管理。

值得注意的是, 区块链网络是去中心化的网络, 所有的节点共同参与记账, 只需要网络中的大多数(超过51\%)节点的账本状态一致即可, 并不需要所有网络中的所有的节点时刻保持高可用状态。同时, 无论是基于Kubernetes的多副本伸缩还是基于监控指标进行伸缩仅保证的是无状态应用的高可用性, 并不解决数据一致性问题, 所以为某个Pod节点增加多副本进行可伸缩意义不大, 利用Kubernetes的自恢复能力保证单Pod节点可用即可。

Blockchain Automation Framework采用脚本方式将HF部署于Kubernetes, 仅仅成了一次性自动化构建、部署Fabric的任务。这虽然能在一定程度上提升HF网络的部署效率, 但这不仅无法将HF复杂的领域知识以插拔化的方式配置进入云基础设施内部而且与Helm的弊端一样, 不能对已经部署完成的HF网络各节点进行完整软件生命周期管理。Kubernetes Operator提供的CRD方式可以插拔式地领域知识注入进云基础设施内部, 降低了Kubernetes运维人员对HF网络的二次学习成本。通过Operator与HF Helm的结合不仅能够提升HF网络的部署效率, 也可以时刻监听Helm状态实现对整个网络进行完整生命周期管理。

虽然针对Peer节点的可伸缩性意义不大, 但从海量数据存储的角度来看, 使用云存储是加强区块链网络存储一种替代方法, 它减少了因区块存储空间有限而造成的限制\cite{gai2020blockchain}。云可以为区块链网络提供提供存储即服务(Storage as a Service, 简称SaaS)作为一种可扩展的链外解决方案。Kubernetes的SaaS通过服务的形式精细化使用存储资源, 实现了对存储定义(PVC)和存储申请(StorageClass)的灵活分离。HF网络的世界状态是一种脱链的链外附加缓存机制, 其最终被保存在链外的LevelDB或CouchDB中。同时, 世界状态的丢失并不会对区块链中的数据产生影响。这种链外存储机制可以使得HF网络能够通过Kubernetes的SaaS机制挂载外部存储单元不必关心存储资源的实际物理位置, 能够做到存储资源的即插即用。 

安全性方面, 区块链领域频繁的出现安全问题可能会动摇人们对去中心化应用的信任, 访问控制是提供云数据安全和隐私的一种基本方法, 可以防止未经授权的用户侵入云数据。不可靠的访问控制方法也会影响其他功能, 如身份验证、授权和数据审核。除自定义访问控制机制外, 云中的传统访问控制方法主要基于成熟的访问控制策略, 现有的传统访问控制策略分为四种: 自主访问控制(Discretionary Access Control, 简称DAC)、强制访问控制(Mandatory Access Control, 简称MAC)、基于角色的访问控制(Role-Based Access Control, 简称RBAC)和基于属性的访问控制(Attribute-Based Access Control, 简称ABAC), 其对比如表\ref{access_control}所示。

{\footnotesize
\begin{longtable}[h]{m{20pt} m{200pt} m{60pt} m{60pt}}
    \caption[访问控制机制]{访问控制机制} \label{access_control} \\
        \toprule  
        \textbf{名称}&\textbf{机制}&\textbf{优点}&\textbf{缺点}\\
        \hline
        DAC & 根据被操作对象的权限控制列表或者权限控制矩阵决定用户的是否能对其进行操作 & 操作简单 & 权限控制分散, 不利于管理 \\

        MAC & 每个对象含有权限标识, 每个用户也含有权限标识, 用户能够操作对象取决于双方权限标识的关系 & 适合机密或等级严格的场景 & 缺乏灵活性 \\

        RBAC & 一个用户关联一个或多个角色, 一个角色关联一个或多个权限, 根据权限需要创建角色并与用户绑定 & 灵活易扩展、职责分离 & 未提供操作顺序控制机制 \\

        ABAC & 根据规则动态计算一个或一组属性是否满足某种规则并授权判断 & 集中管理、按需实现不同粒度权限控制 & 不直观、规则复杂过多会出现性能问题 \\
        \bottomrule
    \end{longtable}
}

云化框架需要在Kubernetes上部署管理HF网络, 需要限制Kubernetes上其他用户对HF网络节点运行时的权限。在DAC中, 合法用户(如云服务提供商)根据自主访问控制策略来允许其他用户(如云用户)访问操作对象\cite{lopez2018access}, 同时阻止非授权的用户访问。DAC操作简单, 云用户能够进行灵活的访问控制, 然而DAC不需要严格的规则造成了权限管理的过于分散, 不利于综合管理; MAC则解决了DAC权限控制过于分散的问题, MAC是通过预定义的信任策略实现且该策略不能动态更改, 目标是限制访问的用户对目标对象的操作能力, MAC过于强调保密性, 管理缺乏灵活性; ABAC将访问规则写在配置文件里以实现多粒度的权限控制, 但是由于定义权限时不易看出用户和对象间的关系, 很容易给集群管理者带来运维上的麻烦; RBAC基于角色的权限控制体系, 能够实现“以岗定人”的目标。 虽然RBAC没有提供顺序操作的控制机制, 但HF中Orderer节点的功能本质上就是对无序的交易进行排序,所以这并不影响HF网络的正常交易流程。

云化框架除启用Kubernetes RBAC的授权机制外, 需要搭配命名空间(Namespace)进行资源隔离。并且应当在命名空间层面提供RBAC而不是集群。使用命名空间来隔离工作负载是一项基本的安全最佳实践, 云化框架可以主动使用命名空间所提供的分割功能可以将Fabric网络隔离在单独的命名空间内, 形成逻辑上的小组, 这可以实现与其他工作负载进行资源与安全性的隔离。

上述安全策略只是保护HF网络的第一道防线, 防止非法操作HF网络各节点, 而不是作为一项保护HF网络内部交易流程以及数据篡改的的策略。由于HF网络是基于MSP的认证性网络, 所以HF网络的参与者需要提供可认证的身份信息, 即HF网络含有多用户认证授权的机制。然而, 手动生成的HF网络密钥以文件的形式保存, 增加了密钥泄漏的风险, 云化框架需要结合HF的多用户授权机制提供安全的证书管理策略。

Prometheus\cite{sukhija2019towards}作为云原生领域的监控事实标准, 其是一个时序数据库又是一个监控系统, 有着强大的功能和良好的生态。
Grafana可以与各种其他类似于Prometheus的数据源进行交互并进行可视化。基于Prometheus的监控体系可以
收集开放的监控指标并进行多维度数据模型的灵活查询, 极大的增强了被监测系统的的可监控性。HF网络各节点的运行终态是Pod, 利用Prometheus监控体系可以无侵入式的收集HF网络各节点的监控指标, 并将指标数据以Grafana图表的方式展示, 真正做到7*24小时可视运维。

HF致力于构建透明、公开、去中心化的企业级分布式应用, 区块链云化框架需要在以下方面满足透明性原则。首先, 区块链云化框架构建HF网络过程的透明性, 区块链云化框架需要具备完备的日志体系, 日志能够提供记录区块链云化框架操作HF网络的行为以及网络状态, 并能够规范的表达出来; 其次, 对于链上交易而言, HF网络本就是公开透明的。 区块链云化框架不能屏蔽这一特性, 需要对HF网络账本具备更加简易、公开、透明的命令式查询渠道, 如命令式查询交易记录。

综上, 如表\ref{policy_set_application}所示, 在区块链HF网络场景下结合快速评审的结果以及云化框架应当满足的设计原则, 剔除了3条不适用于HF去中心化网络特点的策略(S5、S6、S9), 最终形成了面向Hyperledger Fabric的区块链云化框架的9个核心策略实施方案。

\begin{figure}[h] %figure环境,h默认参数是可以浮动,不是固定在当前位置。如果要不浮动,你就可以使用大写float宏包的H参数,固定图片在当前位置,禁止浮动。
    \centering %使图片居中显示
    \includegraphics[width=1.0\textwidth]{FIGs/chapter3/policy_characteristics.pdf} %中括号中的参数是设置图片充满文档的大小,你也可以使用小数来缩小图片的尺寸。
    \caption{策略集应用} %caption是用来给图片加上图题的
    \label{policy_set_application} %这是添加标签,方便在文章中引用图片。
\end{figure}%figure环境


\section{本章小结}

本章为了解Kubernetes Operator如何云化传统领域并对质量属性赋能, 首先详细介绍了快速评审的过程并得到基于Kubernetes Operator云化策略集;其次, 本章重点介绍了面向Hyperledger Fabric的区块链云化框架所应满足的6条设计原则; 最后结合HF网络特性以及区块链云化框架设计原则对上述策略集进行筛选, 形成了面向Hyperledger Fabric的区块链云化框架的9个核心策略实施方案。

\chapter{基于Hyperledger Fabric的区块链云化框架}

本章首先阐述全面阐述基于Hyperledger Fabric的区块链云化框架的原理、流程以及输入、处理单元和输出, 随后给出了基于该框架的原型工具的需求分析、设计以及相关实现。

\section{基于Hyperledger Fabric的区块链云化框架}\label{section: framework}

结合\ref{section: policy_set_application}所得的策略具体实施方案, 本文提出了基于Hyperledger Fabric的区块链云化框架。利用Kubernetes operator方法将领域知识集成到Kubernetes API编排过程中\cite{henning2021reproducible}。Kubernetes API是云原生容器管理系统的大脑, 它是一个复杂的API, 具有多个层与各种资源\cite{Yilmaz2021}。该框架重点整合\ref{section: BaaS}节所展示的BaaS区块链层Fabric的联盟链能力以及计算资源层Kubernetes的计算资源、存储资源等资源为上层的区块链服务层提供完备的去中心化应用开发的能力。同时该框架能够隐式管理密码文件、按需配置调度Kubernetes资源, 解决Fabric生产部署效率、安全性、数据弹性等运营难题, 节约开发人员以及运维人员时间成本, 使得其更加专注于去中心化应用的逻辑。

Operator应该管理单一类型的应用程序, 遵循UNIX原则:只做一件事, 并把它做好\cite{d2020design}。本框架必须解决的主要任务是屏蔽Fabric及Kubernetes底层细节, 简化Fabric网络的各节点的部署, 以及有效利用Kubernetes代码化及云化的管理基础设施, 所以本框架需要分别对Ca、Orderer、Peer三种不同的网络组建分别进行完整生命周期管理。

如图\ref{framework}所示, 本框架的整体工作流程将分为几个步骤。首先, 将Fabric领域知识注入CRD, 这些属性包含Fabric网络中Ca、Orderer、Peer各节点所具备的功能、性能、监控等可插拔的基础属性。将CRD作为本框架的输入, 通过自定义kubectl命令完成静态CRD相关属性的配置。其次, Manager是本框架的处理单元, 其被设计成一个黑盒\cite{yu2020system}, 用户无需关心Manager的内部逻辑设计。 Manager自动生成部署的配置文件, 使用生成的配置文件并结合Helm可以轻松的将Fabric网络各节点部署进目标Kubernetes集群。除了部署之外, Manager将持续监控这些节点及其存储资源的状态。根据持续监测结果并调用Kubernetes及Fabric相关API将各节点调整到期望状态以维持Fabric网络的稳定。最后, Fabric网络是本框架的输出, 通过自定义kubectl命令完成动态通道、链码的增删查改。除Fabric网络各节点基本Deployment、Service外, 本框架采用istio专用基础设施层完成高性能、适应性和可用性\cite{li2019service}\cite{larsson2020impact}的TLS通信负载均衡; 采用原生Role、Secret、PVC的方式管理Fabric网络的权限、密码及存储。

\begin{figure}[h] %figure环境,h默认参数是可以浮动,不是固定在当前位置。如果要不浮动,你就可以使用大写float宏包的H参数,固定图片在当前位置,禁止浮动。
    \centering %使图片居中显示
    \includegraphics[width=1.0\textwidth]{FIGs/chapter4/framework.pdf} %中括号中的参数是设置图片充满文档的大小,你也可以使用小数来缩小图片的尺寸。
    \caption{基于Hyperledger Fabric的区块链云化框架} %caption是用来给图片加上图题的
    \label{framework} %这是添加标签,方便在文章中引用图片。
\end{figure}%figure环境

\subsection{Custom Resource Definition}\label{section: Custom_Resource_Definition}

CRD通过扩展Kubernetes API将具备领域知识资源类型注入进Kubernetes集群中。Kubernetes提供了标准资源ConfigMap, 也可用于使配置数据项供应用程序使用, 但这两种针对不同的情况。ConfigMap擅长为在集群上的Pod中运行的程序提供配置, 应用程序通常希望从Pod中读取此类配置, 例如文件或环境变量的值, 而不是从Kubernetes API中读取。CRs由标准的Kubernetes客户端创建和访问, 遵守Kubernetes规范。通过自定义controller可以监控CRs运行, 这些代码可以反过来创建、更新或删除其他集群对象, 甚至集群外的任意资源。

Fabric作为一种去中心化的多方信息对接的网络, 具有一套标准化的数据结构与接口。本框架基于Fabric网络各节点自身功能及配置\footnotemark[1]\footnotetext[1]{\href{https://github.com/hyperledger/fabric-ca/blob/main/cmd/fabric-ca-server/config.go}{Ca Config}}\footnotemark[2]\footnotetext[2]{\href{https://github.com/hyperledger/fabric/blob/main/sampleconfig/orderer.yaml}{Orderer Config}}\footnotemark[3]\footnotetext[3]{\href{https://github.com/hyperledger/fabric/blob/main/sampleconfig/core.yaml}{Peer Config}}设计三种Fabric静态资源类型作为输入, 篇幅原因仅展示包括但不限于如表\ref{crd_description}所示的属性。额外的, 除上述针对不同网络节点的特殊属性外, 本框架需要为每个节点提供如副本数、镜像、Hosts、日志、ServiceMonitor等基本属性维持上述网络节点的基本运行状态。

{\footnotesize
\begin{longtable}[h]{m{60pt}|m{100pt}|m{210pt}}
    \caption[CRD描述]{CRD描述} \label{crd_description} \\
        \hline   
        \textbf{CRD名称}&\textbf{属性}&\textbf{描述}\\
        \hline
        \multirow{8}*{\parbox[c]{60pt}{Ca Resource \\ Definition}}
        & CRLSizeLimit & 可接受证书撤销列表(Certificate Revocation List, 简称CRL)的大小限制 \\\cline{2-3}
        & TLS & 服务器侦听TLS端口以及证书等信息 \\\cline{2-3}
        & CA & 包含与证书颁发机构相关的信息 \\\cline{2-3}
        & Database & 用作数据存储 \\\cline{2-3}
        & CFG & 配置身份允许的错误密码尝试次数 \\\cline{2-3}
        & CSR & 控制根CA证书的创建, 如根CA证书的过期时间配置 \\\cline{2-3}
        & Registry & 部分控制fabric-ca服务器执行验证包含用户名和密码的注册和检索标识的属性名称、值的方式 \\\cline{2-3}
        & BCCSP & 用于选择要使用的加密库实现 \\\cline{2-3}
        \hline  
        \multirow{4}*{\parbox[c]{60pt}{Orderer \\ Resource \\ Definition}}
        & Genesis & 初始区块相关配置 \\\cline{2-3}
        & BootstrapMethod & 指定了获取引导块系统通道的方法 \\\cline{2-3}
        & ChannelParticipation & 通道管理对系统链码的依赖 \\\cline{2-3}
        & Secret & 包含Orderer的数字签名以及与Ca通信所需的基本信息\\\cline{2-3}
        \hline 
        \multirow{7}*{\parbox[c]{60pt}{Peer Resource \\ Definition}}
        & Gossip & 确保Peer间通过Gossip协议来达到所有账本的最终一致性 \\\cline{2-3}
        & LevelDB/CouchDB & Fabric提供levelDB与CouchDB用以保存Fabric账本信息, 用以灵活适应Peer不同数据库之间的转换 \\\cline{2-3}
        & CouchDBExporter & 采集CouchDB的监控数据 \\\cline{2-3}
        & ExternalChaincodeBuilder & 提供外部链码构建的能力 \\\cline{2-3}
        & Secret & 包含Peer的数字签名以及与Ca通信所需的基本信息\\\cline{2-3}
        & MSP & 所属的组织信息\\\cline{2-3}
        \hline 
    \end{longtable} 
}

\subsection{Manager}

CRs本身仅为特定应用程序提供声明式API的数据项的集合, controller负责对CRs的不同事件做出反馈, 管理CRs的完整生命周期。

\begin{figure}[h] %figure环境,h默认参数是可以浮动,不是固定在当前位置。如果要不浮动,你就可以使用大写float宏包的H参数,固定图片在当前位置,禁止浮动。
    \centering %使图片居中显示
    \includegraphics[width=0.95\textwidth]{FIGs/chapter4/manager.pdf} %中括号中的参数是设置图片充满文档的大小,你也可以使用小数来缩小图片的尺寸。
    \caption{Manager监听CRs} %caption是用来给图片加上图题的
    \label{manager} %这是添加标签,方便在文章中引用图片。
\end{figure}%figure环境

在Manager内部管理了多个CRD, 就需要使用多个控制器进行协调循环。这有助于关注点分离和代码的可读性。如图\ref{manager}所展示的是Manager监听CRs的全过程。首先, 需要在CRs中指定Fabric网络各节点所期望的状态, CRD会注册进入Scheme, 其提供了ApiServer对应的集群中GVK(Group Version Kind, Kubernetes集群资源定义方式)与CRs资源类型的映射, 通过资源类型controllers就能获取CRs所定义的期望状态; 其次, cache通过List-Watch机制与ApiServer进行通信用以同步监听Fabric网络各节点在Kubernetes集群中的创建、删除、更新等操作, cache可以获取Fabric网络各节点的实际状态; 最后, controller循环监听期望状态与实际状态, 若期望状态与实际状态不一致, 则通过调用clients更新、缩放、扩展、备份等操作进行协调一致。

为提升区块链云化框架的生产效率, 本文设计了一套针对Fabric网络各节点helm的通用CRD与controller, 直接将提前配置好的helm随CRD以及controller一起部署进Kubernetes。Helm通过调用Kubernetes的ApiServer逐个将helm chart中的yaml推送给Kubernetes, 当且仅能进行安装。所以helm的弊端是缺乏对资源的全生命期监控, 只有CRD才能持续的监听Kubernetes资源对象的变化事件, 进行全生命期的监控响应, 高可靠的完成部署交付。一旦创建新的CR, controller根据对应的资源对象更新helm的模板参数并重新部署入Kubernetes集群。图\ref{controller}以spec.size为例展示了controller更新helm的流程。本框架不仅通过helm简化部署流程, 并且还能实现带全生命周期管理的helm效果。

\begin{figure}[h] %figure环境,h默认参数是可以浮动,不是固定在当前位置。如果要不浮动,你就可以使用大写float宏包的H参数,固定图片在当前位置,禁止浮动。
    \centering %使图片居中显示
    \includegraphics[width=0.75\textwidth]{FIGs/chapter4/controller.pdf} %中括号中的参数是设置图片充满文档的大小,你也可以使用小数来缩小图片的尺寸。
    \caption{controller循环监听} %caption是用来给图片加上图题的
    \label{controller} %这是添加标签,方便在文章中引用图片。
\end{figure}%figure环境

% 可迁移性
区块链云化框架遵循标准化原则, 复用Fabric网络各节点镜像并利用Kubernetes进行编排和管理底层的物理资源。这可以使用户能够轻松地在集群之间移动和部署云化框架及Fabric网络, 确保区块链系统基础架构的云独立性, 取消对云提供商的强依赖性, 提升本框架的的可迁移性与通用性。

% 存储扩展性
除\ref{section: Custom_Resource_Definition}所提到的利用CRD对Fabric网络配置进行模块化设计外, 本框架为每个运行中的Fabric网络节点选择配置的PVC, 并为每个PVC中预留出一定的额外存储资源。相较于对所有持久存储的服务使用一个PVC而言, 虽然分配存储远超必要范围增加额外的存储资源冗余, 但用拥有更对的PVC能够保障每个网络节点拥有足够的存储空间, 以便在不缺乏存储资源的情况下正确运行节点。拥有更多的PVC增加了首次部署难度及过度调配的风险, 但多PVC能够灵活针对不同节点运行情况利用Kubernetes进行有针对性的存储扩容, 增强框架对于存储的扩展性。尽管多PVC在管理方面存在一定的复杂性, 但在选择多PVC更加符合最佳实践, 并且效率更高\cite{d2020design}。
% 可插入图

% 安全性
在安全性方面区块链云化框架复用Kubernetes的原生安全保障体系, 主要涉及到两方面。 一方面是Kubernetes集群用户对框架操作的访问控制限制, 区块链云化框架会生成很多清单文件向Kubernetes集群中部署Fabric网络, 同时Kubernetes集群需要向已部署的区块链云化框架授予在Fabric网络中生命周期内执行各种任务的权限。Kubernetes没有以用户身份进行身份验证, 本框架将所有Fabric网络资源限制在同一命名空间下, 同时采用基于角色的权限控制(Role-Based Access Control, 简称RBAC)将对Kubernetes资源操作的最小权限映射到框架中的Manager及Fabric网络节点。值得注意的是, 区块链云化框架无需以root身份运行, 在确保在允许Fabric网络正常工作的同时, 应尽可能限制访问; 另一方面是Fabric网络运行环境均受Kubernetes安全容器保护。在密码管理方面, 框架避免使用直接向节点镜像中注入环境变量的方式管理密码信息。本框架采用Secret配合x509\cite{8249485}存储管理导出的敏感数据, 这种方式不但能提高灵活性而且增加了密码的传输、存储、访问安全, 增强隐私保护。


% 可视化运维
本框架采用非介入式的云上监控方案Prometheus以及Grafana进行可视运维, 在CRD中预留exporter、ServiceMonitor等属性, 对应的在helm chart中定制抓取周期的相关配置对Fabric网络中的Ca、Orderer、Peer、CouchDB等进行可视化监控。同时, Grafana开源特性能够创建自定义插件, 一定程度上能提升本框架的可视运维能力。


\subsection{Fabric网络}

Fabric网络一个复杂的分布式系统, 需要权衡速度、性能等条件对不同网络节点的部署状态进行合理设计。静态节点Ca、Orderer、Peer在首次启动网络时就需要部署在Kubernetes集群中并以Pod形式运行。Pod是Kubernetes中可以创建和部署的最小单位。在Kubernetes集群中, Pod有两种运行状态:

\begin{itemize}[itemindent=2em]
    \item Pod中运行单容器: 每个Pod一个容器是最常见的状态, 在这种状态下, 可以将Pod当作单容器进行封装, 但Kubernetes管理仍然是Pod而不是容器;

    \item Pod中运行个容器: 当容器间需要紧密协作时可以在同一Pod中运行多容器。
\end{itemize}

Ca是Fabric的证书授权中心, Orderer负责交易的排序, 这两个节点在配置上需要满足可插拔式设计。 但在网络运行时, 需要各自在一个Pod中运行即可, 同时 在一个Pod中运行可以更好的有效的利用Kubernetes的自动缩放功能进行弹性伸缩。

Peer是Fabric中被使用最多的模块, 是Fabric网络的基石, 其负责区块链数据的存储以及运行链码。由于Peer节点需要频繁的账本存储单元如CouchDB进行交互, 所以peer容器应当与couchdb容器存在于同一Pod中。在同一个Pod中, peer容器与couchdb紧密协作, 拥有相同的存活周期, 更优的, 相同Pod中的不同容器共享进程、IP地址和数据卷, 可以进行频繁的文件和数据交换。Peer支持外部链码部署, 即链码拥有自己独立的Pod运行环境与Peer解耦, 能够纳入智能合约微服务化流程\cite{zhangfuli2021smartcontract}进行快速响应与监控。 

\section{原型工具}

\subsection{工具概述}

云原生具有资源按需配置, 动态伸缩的特性。当前, BaaS虽然能够基于云平台构建区块链系统, 但仅提供脚本化的方式部署区块链网络及智能合约, 仍未深入云基础设施平台的底层有效利用云的特性管理区块链平台, 这导致了BaaS平台对云特性的严重浪费。

\begin{figure}[!htbp] %figure环境,h默认参数是可以浮动,不是固定在当前位置。如果要不浮动,你就可以使用大写float宏包的H参数,固定图片在当前位置,禁止浮动。
    \centering %使图片居中显示
    \includegraphics[width=1.0\textwidth]{FIGs/chapter4/tool.pdf} %中括号中的参数是设置图片充满文档的大小,你也可以使用小数来缩小图片的尺寸。
    \caption{原型工具总体功能} %caption是用来给图片加上图题的
    \label{toolstotal} %这是添加标签,方便在文章中引用图片。
\end{figure}%figure环境

因此, 一个支持区块链有效云化的工具十分重要。如图\ref{toolstotal}所示, 本文基于提出区块链云化框架提供配套的基于Hyperledger Fabric的区块链云化原型工具。原型工具建立在Fabric核心流程之上, 对外不仅完成以声明式的方式自动化配置Fabri网络中的Ca、Orderer、Peer实体组件而且能够以命令的方式动态管理整个Fabric网络, 包含静态Fabric组件以及通道、链码的全生命周期管理; 对内应当区块链领域知识注入进云平台Kubernetes中, 有效利用Kubernetes的安全性、数据扩展性、可监控等能力为Fabric网络赋能。



\subsection{需求分析} \label{section: requirement}

\subsubsection{功能性需求分析}

\begin{figure}[!htbp] %figure环境,h默认参数是可以浮动,不是固定在当前位置。如果要不浮动,你就可以使用大写float宏包的H参数,固定图片在当前位置,禁止浮动。
    \centering %使图片居中显示
    \includegraphics[width=1.0\textwidth]{FIGs/chapter4/fabric_use_case.pdf} %中括号中的参数是设置图片充满文档的大小,你也可以使用小数来缩小图片的尺寸。
    \caption{Fabric网络管理用例图} %caption是用来给图片加上图题的
    \label{fabric_use_case} %这是添加标签,方便在文章中引用图片。
\end{figure}%figure环境


如图\ref{fabric_use_case}所示, Fabric网络管理人员期望能够通过原型工具不影响原节点功能性的前提下对Fabric网络各节点分别进行命令式启停, 降低Fabric网络的启动时间成本。


% 注册->进数据库->签发证书->生成证书文件

表\ref{ca_use_case}展示了Ca管理的用例描述, Ca管理需为Fabric网络管理人员提供灵活方便的启停Ca节点的功能, 同时提供管理注册、签发证书的功能。在启动过程中应以kubectl内置命令行参数配置的方式设置Ca启动所可选的配置项, 如Database、Host、CLRSizeLimit等。同样, Ca管理需要覆盖Ca节点原有基本的用户注册、用户登记的功能。

{\footnotesize
\begin{longtable}[h]{m{60pt}|m{280pt}}
    \caption[Ca管理用例表]{Ca管理用例表} \label{ca_use_case} \\
        \hline  
        ID&UC-Ca\\
        \hline
        名称&可视化建模用例\\
        \hline
        描述&Fabric网络管理员可以通过kubectl的方式创建、删除Ca节点, 并能够以命令行参数的形式将Ca启动参数注入。 同时, Fabric网络管理员能够无障碍进行用户注册、登记。\\
        \hline
        触发条件&Fabric网络管理员输入对应Ca命令\\
        \hline
        前置条件&Kubernetes集群环境\\
        \hline
        后置条件&无\\
        \hline
        正常流程& (1)Fabric网络管理员输入“kubectl-hlf ca create”并以可选参数的形式输入其他如Ca名称、容量等配置项;
        \newline (2)Fabric网络管理员输入“kubectl-hlf ca register”并以可选参数的形式输入其他如用户名、密码等配置项;
        \newline (3)Fabric网络管理员输入“kubectl-hlf ca enroll”并输入已经注册过的用户名、密码等信息将注册的用户进行登记并导出证书信息;
        \newline (4)可选的, Fabric网络管理员输入“kubectl-hlf ca delete”并输入ca-name以删除ca节点;\\
        \hline
        异常流程&无\\
        \hline
    \end{longtable} 
}


表\ref{peer_use_case}展示了Peer管理的用例描述, Peer管理需为Fabric网络管理人员提供灵活方便的启停Peer节点的功能, 在启动过程中应以kubectl内置命令行参数配置的方式设置Peer启动所可选的配置项, 如Peer对应的Ca名称账本存储类型、Gossip协议、组织信息等。

{\footnotesize
\begin{longtable}[h]{m{60pt}|m{280pt}}
    \caption[Peer管理用例表]{Peer管理用例表} \label{peer_use_case} \\
        \hline  
        ID&UC-Peer\\
        \hline
        名称&可视化建模用例\\
        \hline
        描述&Fabric网络管理员可以通过kubectl的方式创建、删除Peer节点, 并能够以命令行参数的形式将Peer启动参数注入。\\
        \hline
        触发条件&Fabric网络管理员输入对应Peer命令\\
        \hline
        前置条件&已经启动对应组织级的Ca\\
        \hline
        后置条件&无\\
        \hline
        正常流程& (1)Fabric网络管理员输入“kubectl-hlf peer create”并以可选参数的形式输入其他如Peer名称、所属组织、账本存储类型等配置项;
        \newline (2)可选的, Fabric网络管理员输入“kubectl-hlf peer delete”并输入peer-name以删除Peer节点;\\
        \hline
        异常流程& Fabric网络管理员输入的对应Ca名称匹配不上, 报错提示\\
        \hline
    \end{longtable} 
}

表\ref{orderer_use_case}展示了Orderer管理的用例描述, Orderer管理需为Fabric网络管理人员提供灵活方便的启停Orderer节点的功能, 同时能够将Orderer加入到通道中,在启动过程中应以kubectl内置命令行参数配置的方式设置Orderer启动所可选的配置项, 如Orderer对应的Ca名称、自己的名称、组织、容量等信息。

{\footnotesize
\begin{longtable}[h]{m{60pt}|m{280pt}}
    \caption[Orderer管理用例表]{Orderer管理用例表} \label{orderer_use_case} \\
        \hline  
        ID&UC-Orderer\\
        \hline
        名称&可视化建模用例\\
        \hline
        描述&Fabric网络管理员可以通过kubectl的方式创建、删除Orderer节点, 并能够以命令行参数的形式将Orderer启动参数注入。\\
        \hline
        触发条件&Fabric网络管理员输入对应Orderer命令\\
        \hline
        前置条件& (1)已经启动对应组织级的Ca
        \newline (2)Orderer加入通道前确保通道建立\\
        \hline
        后置条件&无\\
        \hline
        正常流程& (1)Fabric网络管理员输入“kubectl-hlf ordnode create”并以可选参数的形式输入其他如Orderer名称、所属组织、所属组织的Ca名称等配置项;
        \newline (2)Fabric网络管理员输入“kubectl-hlf ordnode join” 并以参数的形式输入名称、命名空间、输入创世区块信息、自己的证书文件;
        \newline (3)可选的, Fabric网络管理员输入“kubectl-hlf ordnode delete”并输入orderer-name以删除Orderer节点;\\
        \hline 
        异常流程& (1) Fabric网络管理员输入的对应Ca名称匹配不上, 报错提示;
        \newline (2) Orderer节点证书文件验证不通过, 禁止加入; \\
        \hline
    \end{longtable} 
}

\begin{figure}[!htbp] %figure环境,h默认参数是可以浮动,不是固定在当前位置。如果要不浮动,你就可以使用大写float宏包的H参数,固定图片在当前位置,禁止浮动。
    \centering %使图片居中显示
    \includegraphics[width=1.0\textwidth]{FIGs/chapter4/chan_cc_use_case.pdf} %中括号中的参数是设置图片充满文档的大小,你也可以使用小数来缩小图片的尺寸。
    \caption{通道管理及链码管理用例图} %caption是用来给图片加上图题的
    \label{chan_cc_use_case} %这是添加标签,方便在文章中引用图片。
\end{figure}%figure环境


{\footnotesize
\begin{longtable}[h]{m{60pt}|m{280pt}}
    \caption[通道管理用例表]{通道管理用例表} \label{chan_use_case} \\
        \hline  
        ID&UC-Orderer\\
        \hline
        名称&可视化建模用例\\
        \hline
        描述&Fabric开发人员可以通过kubectl的方式创建、管理通道\\
        \hline
        触发条件&Fabric网络管理员输入对应通道命令\\
        \hline
        前置条件&已经启动好基础Fabric网络, 包含Ca、Orderer、Peer\\
        \hline
        后置条件&无\\
        \hline
        正常流程& (1)Fabric开发人员输入“kubectl-hlf channel generate”并以可选参数的形式输入所包含的组织名称以及输出的创世区块保存文件;
        \newline (2)Fabric开发人员输入“kubectl-hlf channel join” 并以参数的形式输入应当加入该通道的节点名称、组织信息和用户;
        \newline (3)Fabric开发人员员输入“kubectl-hlf channel inspect”并以参数的形式输入通道的名称导出该通道内配置与Orderer、Peer节点的信息;
        \newline (4)Fabric开发人员员输入“kubectl-hlf channel addanchorpeer”并以参数的形式输入通道的名称用以向通道中加入锚节点;
        \newline (5)Fabric开发人员员输入“kubectl-hlf channel top”并以参数的形式输入通道的名称用以向查询账本的高度;\\
        \hline 
        异常流程& 无; \\
        \hline
    \end{longtable} 
}

{\footnotesize
\begin{longtable}[h]{m{60pt}|m{280pt}}
    \caption[链码管理用例表]{链码管理用例表} \label{cc_use_case} \\
        \hline  
        ID&UC-Orderer\\
        \hline
        名称&可视化建模用例\\
        \hline
        描述&Fabric开发人员可以通过kubectl的方式安装、提交、查询链码;\\
        \hline
        触发条件&Fabric开发人员输入对应链码命令\\
        \hline
        前置条件&已经部署好通道\\
        \hline
        后置条件&无\\
        \hline
        正常流程& (1)Fabric开发人员输入“kubectl-hlf chaincode intall”并以可选参数的形式输入链码所在地址、链码语言、标签等参数用以安装链码;
        \newline (2)Fabric开发人员输入“kubectl-hlf chaincode queryinstalled" 并以参数的形式传入组织名称、用户、Peer信息等参数用以展示已经安装的链码;
         \newline (3)Fabric开发人员输入“kubectl-hlf chaincode approveformyorg" 并以参数的形式传入链码package-id、链码名称、组织信息、策略等参数用以所在组织审批链码;
        \newline (4)Fabric开发人员输入“kubectl-hlf chaincode commit”并以参数的形式输入链码名称、策略等信息用以提交链码
        \newline (5)Fabric开发人员输入“kubectl-hlf chaincode invoke”并以参数的形式输入链码名称、调用方法等信息用以调用链码
        \newline (6)Fabric开发人员输入“kubectl-hlf chaincode query”并以参数的形式输入链码名称、调用方法等信息用以查询链码\\
        \hline 
        异常流程& (1)因网络原因安装链码时间过长而导致, 提示并报错\\
        \hline
    \end{longtable} 
}

除管理Fabric网络节点外, 如图\ref{chan_cc_use_case}所示, Fabric开发人员期望能够通过原型工具
动态、灵活且低成本的完成通道以及链码的各项操作。

表\ref{chan_use_case}展示了通道管理的用例描述, 通道管理需为Fabric开发人员屏蔽大量证书文件提供灵活方便的开启通道的功能, 应以kubectl内置命令行参数配置的方式在配置通道, 包含对通道内部锚节点、创世区块、账本的管理。表\ref{cc_use_case}展示了链码管理的用例描述, 链码管理需为Fabric网络管理人员提供灵活方便的安装、提交链码等功能。


\subsubsection{非功能性需求分析}

本文涉及的区块链云化框架原型工具需要为Fabric网络及链码提供标准化的、便捷化启停构建方式, 同时还需要利用云的特性提升以下非功能性需求:

\begin{itemize}[itemindent=2em]
    \item 易用性: Fabric网络管理人员和开发人员在学习了基本的Fabric概念之后且Fabric网络组件镜像完备的情况下, 可以通过原型工具简化Fabric网络配置流程并应当在10分钟之内完成Fabric网络的构建;

    \item 可靠性: 在命令输入过程中, 原型工具应当提前自动校验命令行参数的准确性, 能够判断各种异常输入并快速做出提示响应, 防止出现异常;

    \item 可迁移性: 原型工具需要具备便捷的安装方式, 以便于能够在支持Kubernetes的云上自由迁移;

    \item 可扩展性: 原型工具需要为Fabric网络提供可插拔的标准化接口, 如共识算法、账本存储单元等; 同时, 随着交易数量的增加, 链外存储压力上升, 工具应能对存储进行动态的不重启扩容;

    \item 安全性: 原型工具需要具备严格的权限访问控制策略, Fabric网络管理员(开发人员)需要经过认证之后才能有权限操作Fabric网络各节点的启停; 只有经过认证的Fabric网络用户才能进行合法交易;
\end{itemize}


\subsection{设计与实现}

基于Hyperledger Fabric的区块链云化工具是一个基于Kubernetes operator的应用, 其整合结合Helm完成Fabric网络的快速部署和节点的全生命周期管理以及Cobra\footnotemark[1]\footnotetext[1]{\href{https://github.com/spf13/cobra}{cobra github地址}}完成命令的封装。由于原型工具在管理Ca、Orderer、Peer各组件时存在逻辑上的重复性, 故本节将以Ca为例介绍功能性需求、非功能性需求的设计与实现。

% create CRD
当Fabric网络管理员需要创建Ca并输入对应的命令及参数时, 如图\ref{create_crd}。原型工具首先会解析管理员输入参数的合法性, 然后将对应的输入参数填充到已经定义好的FabricCA模版中,最后调用Kubernetes client将Fabric Ca Resource依据CRD的规则部署进入集群中。伪代码\ref{code1}展示了创建Ca Resource的伪代码。

\begin{figure}[!htbp] %figure环境,h默认参数是可以浮动,不是固定在当前位置。如果要不浮动,你就可以使用大写float宏包的H参数,固定图片在当前位置,禁止浮动。
    \centering %使图片居中显示
    \includegraphics[width=0.7\textwidth]{FIGs/chapter4/create_crd.pdf} %中括号中的参数是设置图片充满文档的大小,你也可以使用小数来缩小图片的尺寸。
    \caption{创建Ca Resource时序图} %caption是用来给图片加上图题的
    \label{create_crd} %这是添加标签,方便在文章中引用图片。
\end{figure}%figure环境

\begin{algorithm}[!htbp]
    \floatname{algorithm}{\footnotesize 伪代码}
    \caption{\footnotesize 创建Ca Resource伪代码}
    \label{code1}
    {\footnotesize
    \begin{algorithmic}
        \renewcommand{\algorithmicrequire}{ \textbf{Input:}}
        \REQUIRE  
        “kubeclt ca create” with args

        \renewcommand{\algorithmicensure}{\textbf{Output:}}
        \ENSURE
        Success or Fail

        \STATE{err := ParseArgs()}
        \IF{err != nil}
            \STATE{return fail} 
        \ENDIF

        \STATE{client, err  := GetKubeOperatorClient()}
        \IF{err != nil}
            \STATE{return fail} 
        \ENDIF

        \STATE{fabricCa := \&v1alpha1.FabricCA\{}
        \STATE{\quad Initialization according to parameters}
        \STATE{\}}


        \IF{args.output}
            \STATE{out, err := yaml.Marshal(\&fabricCa)} 
        \ELSE
            \STATE{err := client.FabricCA(namespaces).Create(fabricCa)}
            \IF{err != nil}
                \STATE{return fail} 
            \ENDIF
        \ENDIF

        \STATE{return success}
    \end{algorithmic}
    }
\end{algorithm}

当Fabric Ca Resource一旦被部署到集群中就会被处理单元Manager中的Ca  controller探查到其存活。如图\ref{reconcile}所示, Ca controller通过Reconcile循环探听Ca Resource。由于资源被删除后再也无法获取到被删除资源的信息, 所以利用Finalizer字段进行标识, GetDeletetionTimeStamp()用于获取CR被删除时的时间戳。一旦探听到Ca Resource的存在就会先处理Finalizer字段。DeletionTimestamp不为空时, controller会轮询该CR的更新请求执行处理所有的Finalizer。随后, Ca controller会检查当前是否存在Ca helm release, 若不存在则将当前状态Ca Resource所定义的TLS、CFG等信息生成helm chart并将其部署进入集群中; 若存在则先获取当前Ca Resource的状态并对其进行更新之后再生成helm chart部署。helm chart中定义了Ca启动所需要的全部如Deployment、Service、Istio、ServiceMonitor等yaml, 通过启动helm就可以一次性将其全部启动部署。伪代码\ref{code2}展示了Ca Controller Reconcile的伪代码。


% Manager
\begin{figure}[!htbp] %figure环境,h默认参数是可以浮动,不是固定在当前位置。如果要不浮动,你就可以使用大写float宏包的H参数,固定图片在当前位置,禁止浮动。
    \centering %使图片居中显示
    \includegraphics[width=1.0\textwidth]{FIGs/chapter4/reconcile.pdf} %中括号中的参数是设置图片充满文档的大小,你也可以使用小数来缩小图片的尺寸。
    \caption{Ca Controller Reconcile逻辑时序图} %caption是用来给图片加上图题的
    \label{reconcile} %这是添加标签,方便在文章中引用图片。
\end{figure}%figure环境


\begin{algorithm}[!htbp]
    \floatname{algorithm}{\footnotesize 伪代码}
    \caption{\footnotesize Ca Controller Reconcile伪代码}
    \label{code2}
    {\footnotesize
    \begin{algorithmic}
        \renewcommand{\algorithmicrequire}{ \textbf{Input:}}
        \REQUIRE  
        nil

        \renewcommand{\algorithmicensure}{\textbf{Output:}}
        \ENSURE
        nil

        \STATE{fabricCa := \&v1alpha1.FabricCA\{\}}
        \IF{fabricCa.GetDeletetionTimestamp() != nil}
            \IF{fabricCa.GetFinalizers().contains(caFinalizer)}
                \STATE{RemoveFinalizer(fabricCa, caFinalizer)} 
            \ENDIF
        \ENDIF

        \IF{!fabricCa.GetFinalizers().contains(caFinalizer)}
            \STATE{AddFinalizer(fabricCa, caFinalizer)} 
        \ENDIF

        \STATE{exits, err  := status.Run(caReleaseName)}
        \IF{err != nil}
            \STATE{return err} 
        \ENDIF

        \IF{exits}
            \STATE{c, err := GetCurrentSpecConfig(caReleaseName)}
            \IF{err != nil}
                \STATE{return err} 
            \ENDIF
            \STATE{s, err := GetExistingStatus(caReleaseName)}
            \IF{err != nil}
                \STATE{return err} 
            \ENDIF
            \STATE{newCa := fabricCa.DeepCopy()}
            \STATE{newCa.Status = s.Status}

            \STATE{release, err  := cmd.Run(caReleaseName, c)}
            \IF{!reflect.DeepEqual(newCa.Status, release.Status)}
                \STATE{\_, err := status().Update(newCa)}
                \IF{err != nil}
                    \STATE{return err} 
                \ENDIF
            \ENDIF 
            
        \ELSE
            \STATE{c, err := GetCurrentSpecConfig(caReleaseName)}
            \IF{err != nil}
                \STATE{return err} 
            \ENDIF
            \STATE{release, err  := cmd.Run(caReleaseName, c)}
            \IF{err != nil}
                \STATE{return err} 
            \ENDIF

        \ENDIF
      
    \end{algorithmic}
    }
\end{algorithm}

当Ca成功在Kubernetes中启动后, 可以通过命令形式完成用户注册、用户登记的功能。如图\ref{enroll}所示, Fabric网络管理员输入Enroll命令以及相关参数, 原型工具会对其进行参数解析。当解析成功后, 原型工具会获取集群中Ca对外接口的相关信息, 包含URL、端口等。随后, 原型工具会生成Ca client, 并将管理员输入的用户信息通过接口传递给Ca Server。Ca Server就是原Fabric Ca在集群中服务的状态所以其具备完整的Ca的功能, 当Enroll完成后会返回key以及cert, 原型工具会利用X509对其编码并保存到yaml中返回给管理员。伪代码\ref{code3}展示了Ca Enroll User的伪代码。

% Manager
\begin{figure}[!htbp] %figure环境,h默认参数是可以浮动,不是固定在当前位置。如果要不浮动,你就可以使用大写float宏包的H参数,固定图片在当前位置,禁止浮动。
    \centering %使图片居中显示
    \includegraphics[width=1.0\textwidth]{FIGs/chapter4/enroll.pdf} %中括号中的参数是设置图片充满文档的大小,你也可以使用小数来缩小图片的尺寸。
    \caption{Ca Enroll User逻辑时序图} %caption是用来给图片加上图题的
    \label{enroll} %这是添加标签,方便在文章中引用图片。
\end{figure}%figure环境

\begin{algorithm}[!htbp]
    \floatname{algorithm}{\footnotesize 伪代码}
    \caption{\footnotesize Ca Enroll User伪代码}
    \label{code3}
    {\footnotesize
    \begin{algorithmic}
        \renewcommand{\algorithmicrequire}{ \textbf{Input:}}
        \REQUIRE  
        “kubeclt ca enroll” with args

        \renewcommand{\algorithmicensure}{\textbf{Output:}}
        \ENSURE
        User key\&cert yaml

        \STATE{err := ParseArgs()}
        \IF{err != nil}
            \STATE{return fail} 
        \ENDIF

        \STATE{client, err  := GetKubeOperatorClient()}
        \IF{err != nil}
            \STATE{return fail} 
        \ENDIF

        \STATE{url, err  := GetURLForCa()}
        \IF{err != nil}
            \STATE{return fail} 
        \ENDIF

        \STATE{crt, pk , err:= client.EnrollUser(args.Name, args.Secret, url)}

        \IF{err != nil}
            \STATE{return fail} 
        \ENDIF
 
        \STATE{crtPem := EncodeX509Certificate(crt)}
        \STATE{pkPem := EncodePrivateKey(pk)}

        \STATE{userYaml := yaml.Marshal(\{}
        \STATE{\quad "key": pkPem,} 
        \STATE{\quad "cert": crtPem} 
        \STATE{\})}

        \STATE{io.writeFile(args.output, userYaml)}

        \STATE{return nil}
    \end{algorithmic}
    }
\end{algorithm}

原型工具为Fabric网络的各组件提供了标准化的helm chart模板。除Deployment以及Service外, 在访问权限控制方面采用RBAC的方式。原型工具将所有资源放入同一命名空间下, 对于自定义生成的资源, 设置了两种类型的角色(ClusterRole)分别是editor以及viewer, 如图\ref{safety}-I展示了viewer角色的权限, editor相较于viewer则增加了create、delete、update、patch的权限, 使用ClusterRoleBinding将角色与用户(ServiceAccount)进行捆绑。如图\ref{safety}-II所示, 原型工具通过Secret保存Fabric网络生成的敏感密钥数据。

\begin{figure}[h] %figure环境,h默认参数是可以浮动,不是固定在当前位置。如果要不浮动,你就可以使用大写float宏包的H参数,固定图片在当前位置,禁止浮动。
    \centering %使图片居中显示
    \includegraphics[width=0.8\textwidth]{FIGs/chapter4/safety.pdf} %中括号中的参数是设置图片充满文档的大小,你也可以使用小数来缩小图片的尺寸。
    \caption{原型工具安全策略} %caption是用来给图片加上图题的
    \label{safety} %这是添加标签,方便在文章中引用图片。
\end{figure}%figure环境

在数据存储扩容方面, 原型工具采用链外存储的方式, 利用PVC及StorageClass动态资源供应模式将交易数据高效的写入可插拔式的持续久化介质里面。如图\ref{pvc_sc}所示, 当Fabric网络管理员创建Peer节点并输入需要多少容量的couchDB时, 原型工具会根据内置的PVC进行适配寻找合适的PV进行存储。PVC只是针对于存储的声明并不会进行真正的存储, 其服务于Peer Pod。原型工具在启动初期会定义StorageClass供PVC使用, 当新的Pod PVC被创建时, Kubernetes会寻找StorageClass并为PVC创建符合要求的PV以供存储。原型工具在定义StorageClass时会将“allowVolumeExpansion”字段设置为“true”, 当Fabric网络管理员设定的初始存储值不够时, 可以通过修改PVC的容量进行动态的灵活扩容。链外存储的使得账本数据与操作独立实现, “账本数据生于链却独立于链”, 能够更好的支持数据治理。

\begin{figure}[h] %figure环境,h默认参数是可以浮动,不是固定在当前位置。如果要不浮动,你就可以使用大写float宏包的H参数,固定图片在当前位置,禁止浮动。
    \centering %使图片居中显示
    \includegraphics[width=0.7\textwidth]{FIGs/chapter4/pvc_sc.pdf} %中括号中的参数是设置图片充满文档的大小,你也可以使用小数来缩小图片的尺寸。
    \caption{原型工具存储策略} %caption是用来给图片加上图题的
    \label{pvc_sc} %这是添加标签,方便在文章中引用图片。
\end{figure}%figure环境

最后, 原型工具利用Prometheus监控体系给所有Fabric网络组件以及工具本身提供监控能力。利用Prometheus并在Helm chart中配置ServiceMonitor、PodMonitor可以为监控对象实时采集数据并存储。 此外, 搭配Grafna提供可视化面板方便对Fabric网络各项指标进行监控。

\section{本章小节}

本章介绍了基于Hyperledger Fabric的区块链云化框架及其原型工具。首先, 根据\ref{section: policy_set_application}所得的策略具体实施方案给出了云化框架的云化框架的工作流程、输入单元CRD、处理单元Manager以及输出单元Fabric网络节点运行状态。然后结合云化框架介绍了搭配原型工具的需求分析、设计与实现。



% \begin{algorithm}[!htbp]
%     \floatname{algorithm}{\footnotesize 算法}
%     \caption{\footnotesize 模型校验算法}
%     \label{algorithm1}
%     {\footnotesize
%     \begin{algorithmic}
%         \renewcommand{\algorithmicrequire}{ \textbf{Input:}}
%         \REQUIRE  
%         mxCells:HTMLCollectionOf<Element>

%         \REQUIRE
%         patterns:PatternData

%         \renewcommand{\algorithmicensure}{\textbf{Output:}}
%         \ENSURE
%         models:Map<String, Pattern>

%         \STATE{Boolean success = true;}
%         \FOR{mxCell in mxCells}

%         \STATE{patterns.setSourceToTarget(mxCell.getId, mxCell.getTarget);}

%         \STATE{patterns.setTargetToSource(mxCell.getSource, mxCell.getId);}
        
%         \ENDFOR

%         \FOR{mxCell in mxCells}
%             \IF{patterns.isParentCell(mxCell)}
%                 \IF{patterns.validation(mxCell)}
%                     \STATE{newPattern = new Pattern(mxCell);}
%                     \STATE{patterns.models.add(mxCell.getId, newPattern);}
%                 \ELSE
%                     \STATE{success = false;}
%                     \STATE{break;}
%                 \ENDIF
%             \STATE{skip loopStep;}
%             \ELSE
%             \STATE{continue loop;}
%             \ENDIF
%         \ENDFOR
          
%         \IF{success}

%             \STATE{sendSuccessMessage();}

%             \STATE{return patterns.models;}

%         \ELSE
%             \STATE{sendErrorMessage();}
        
%         \ENDIF
        
%     \end{algorithmic}
%     }
% \end{algorithm}


\chapter{测试与评估}

本章将介绍面向Hyperledger Fabric的区块链云化框架的测试与评估工作。首先, 本章以典型案例以及SAAM方法对原型工具基本能力自证; 其次, 采用五层成熟度模型进行成熟度衡量; 最后, 与官方工具Cello\footnotemark[1]\footnotetext[1]{\href{https://github.com/hyperledger/cello}{Hyperledger Cello}}进行对比评估。



\section{测试环境}

如图\ref{assessment}所示, 本文从三个方面对本文提出的区块链云化框架与工具进行综合评估。首先是基本能力自证, 本文以典型案例研究的方式证明原型工具的功能完备性以及智能合约微服务改造的可行性, 结合软件体系结构评估方法Software Architecture Analysis Method(简称SAAM)验证区块链云化节点的质量属性; 其次是成熟度衡量, 本文以定性研究\cite{tashakkori1998mixed}的方式结合Operator五层成熟度模型对原型工具进行成熟度能力的评估; 最后是工具的对比分析, 本文将原型工具与Cello进行定量对比, 评估原型工具的网络节点部署时间以及链码交付效率。

\begin{figure}[h] %figure环境,h默认参数是可以浮动,不是固定在当前位置。如果要不浮动,你就可以使用大写float宏包的H参数,固定图片在当前位置,禁止浮动。
    \centering %使图片居中显示
    \includegraphics[width=0.4\textwidth]{FIGs/chapter6/assessment.pdf} %中括号中的参数是设置图片充满文档的大小,你也可以使用小数来缩小图片的尺寸。
    \caption{评估体系} %caption是用来给图片加上图题的
    \label{assessment} %这是添加标签,方便在文章中引用图片。
\end{figure}%figure环境

由于在本机环境下搭建Cello会出现诸多问题, 如文件挂载权限、编译等, 所以本文准备了两套测试环境。本文首先在本地环境进行功能方面的测试, 本地机器为MacBook Pro, 并采用Docker for Desktop搭建的单节点Kubernetes充当集群环境; 其次, 在第\ref{section: tool_comparison}节中, 为顺利运行对比工具Cello, 本文利用云主机Ubuntu 18.04搭建Minikube充当集群环境。上述两种环境具体配置如表\ref{computer}所示。

{\footnotesize
\begin{longtable}[h]{m{40pt} m{100pt} m{100pt}}
    \caption[配置详情]{配置详情} \label{computer} \\
        \toprule    
        \textbf{环境} & \textbf{配置项目} & \textbf{配置详情} \\
        \hline
        \multirow{2}*{\parbox[c]{40pt}{本机环境}}    
        & Docker for Desktop&Version 4.7.0\\     
        & Kubernetes&v1.22.5\\
        \hline
        \multirow{3}*{\parbox[c]{40pt}{云主机}} 
        & Docker & Community 20.10.13 \\
        & minikube & v1.25.2 \\
        & Kubernetes & v1.23.3 \\
        \bottomrule 
    \end{longtable} 
}


\section{基本能力自证}\label{section: tool_test}

\subsection{案例研究}

本小节主要以案例研究的方式对原型工具进行功能性测试。如图\ref{fabric_net}所示, 本节将以搭建最经典、简单的HF网络案例的方式针对\ref{section: requirement}节的用例进行完整的功能测试。测试重点针对于Ca、Peer、Orderer启动、创建通道、部署链码、调用的全部过程,验证HF网络启动及链码部署的正确性。

\begin{figure}[h] %figure环境,h默认参数是可以浮动,不是固定在当前位置。如果要不浮动,你就可以使用大写float宏包的H参数,固定图片在当前位置,禁止浮动。
    \centering %使图片居中显示
    \includegraphics[width=1.0\textwidth]{FIGs/chapter6/fabric_net.pdf} %中括号中的参数是设置图片充满文档的大小,你也可以使用小数来缩小图片的尺寸。
    \caption{测试网络} %caption是用来给图片加上图题的
    \label{fabric_net} %这是添加标签,方便在文章中引用图片。
\end{figure}%figure环境

\textbf{前置条件:}原型工具首先将编写好的HF网络各节点的CRD注入Kubernetes。注入的CRD可以通过原生kubectl命令进行管理。如图\ref{crdresult}所示为部署之后的结果, 已经将Ca、Peer、Orderer和链码的CRD部署入集群。

\begin{figure}[h] %figure环境,h默认参数是可以浮动,不是固定在当前位置。如果要不浮动,你就可以使用大写float宏包的H参数,固定图片在当前位置,禁止浮动。
    \centering %使图片居中显示
    \includegraphics[width=1.0\textwidth]{FIGs/chapter6/crds.png} %中括号中的参数是设置图片充满文档的大小,你也可以使用小数来缩小图片的尺寸。
    \caption{CRD部署结果} %caption是用来给图片加上图题的
    \label{crdresult} %这是添加标签,方便在文章中引用图片。
\end{figure}%figure环境

\textbf{步骤一:} 如表\ref{org1_test}所示, HF网络管理员首先为org1创建了名为“org1-ca”的Ca节点; 其次, 在org1中创建了名为“org1-peer0”的Peer节点,并利用链外存储CouchDB作为账本存储单元。如图\ref{testcase1result}所示为测试结果。

{\footnotesize
\begin{longtable}[h]{m{45pt} m{45pt} m{180pt} m{50pt} m{20pt}}
    \caption[创建Org1测试用例]{Org1测试用例} \label{org1_test}\\
        \toprule  
        \textbf{用例编号}&\textbf{测试编号}&\textbf{用户输入}&\textbf{预期结果}&\textbf{实际}\\
        \hline
        UC-Ca & TC1.1 & 创建名为org1-ca的Ca节点 & 成功启动Ca & 通过 \\
        \hline
        UC-Peer & TC1.2 & 创建名为org1-peer0的Peer节点, 其拥有外部存储CouchDB, 所属于Org1MSP & 成功启动Peer节点 & 通过 \\
        \bottomrule
    \end{longtable} 
}

\begin{figure}[h] %figure环境,h默认参数是可以浮动,不是固定在当前位置。如果要不浮动,你就可以使用大写float宏包的H参数,固定图片在当前位置,禁止浮动。
    \centering %使图片居中显示
    \includegraphics[width=0.9\textwidth]{FIGs/chapter6/peer.png} %中括号中的参数是设置图片充满文档的大小,你也可以使用小数来缩小图片的尺寸。
    \caption{创建Org1测试结果} %caption是用来给图片加上图题的
    \label{testcase1result} %这是添加标签,方便在文章中引用图片。
\end{figure}%figure环境

\textbf{步骤二:} 如表\ref{orderer_test}所示, HF网络管理员需要首先为OrdererMSP创建名为“ord-ca”的Ca节点; 其次, 在OrdererMSP中创建了名为“ord-node1”的Peer节点。如图\ref{testcase2result}所示为测试结果, 原型工具可以为HF网络管理员创建Orderer组织的Ca与Orderer节点, 其中包含但不限于Deployment、Service、Pod、Secret等。

{\footnotesize
\begin{longtable}[h]{m{45pt} m{45pt} m{180pt} m{50pt} m{20pt}}
    \caption[创建Orderer测试用例]{创建Orderer测试用例} \label{orderer_test}\\
        \toprule  
        \textbf{用例编号}&\textbf{测试编号}&\textbf{用户输入}&\textbf{预期结果}&\textbf{实际}\\
        \hline
        UC-Ca & TC2.1 & 创建名为ord-ca的Ca节点 & 成功启动Ca & 通过 \\
        \hline
        UC-Orderer & TC2.2 & 创建名为ord-node1的Orderer节点, 所属于OrdererMSP & 成功启动Orderer节点 & 通过 \\
        \bottomrule
    \end{longtable} 
}

\begin{figure}[h] %figure环境,h默认参数是可以浮动,不是固定在当前位置。如果要不浮动,你就可以使用大写float宏包的H参数,固定图片在当前位置,禁止浮动。
    \centering %使图片居中显示
    \includegraphics[width=0.9\textwidth]{FIGs/chapter6/orderer.png} %中括号中的参数是设置图片充满文档的大小,你也可以使用小数来缩小图片的尺寸。
    \caption{创建Orderer测试结果} %caption是用来给图片加上图题的
    \label{testcase2result} %这是添加标签,方便在文章中引用图片。
\end{figure}%figure环境

\textbf{步骤三:} 如表\ref{reg_enroll_test}所示, HF网络管理员需要为OrdererMSP以及Org1注册并登记若干用户, 并将用户的证书密钥信息导到Yaml文件中。表中仅展示在Orderer中注册用户, Org1中步骤相似, 注册名为peeruser的用户。测试结果表明, 可以成功将用户及OrdererMSP信息导出。

{\footnotesize
\begin{longtable}[h]{m{45pt} m{45pt} m{180pt} m{50pt} m{20pt}}
    \caption[注册、登记用户测试用例]{注册、登记用户测试用例} \label{reg_enroll_test}\\
        \toprule  
        \textbf{用例编号}&\textbf{测试编号}&\textbf{用户输入}&\textbf{预期结果}&\textbf{实际}\\
        \hline
        UC-Ca & TC3.1 & 为OrdererMSP注册名为ordereruser的用户, 其密码为ordererpw & 成功注册 & 通过 \\
        \hline
        UC-Ca & TC3.2 & 输入用户名密码, 为ordereruser用户登记 & 成功登记,输出证书文件 & 通过 \\
        \bottomrule
    \end{longtable} 
}

\textbf{步骤四:} 如表\ref{channel_test}所示, HF开发人员需要在新创建的HF网络上创建通道, 并初始化该通道内的创世区块。当通道被创建完成之后, 需要将在该通道内进行交易的Orderer、Peer加入该通道。最后, HF开发人员可以指定锚节点并查看通道的高度。如图\ref{channel_test_result}所示为测试结果, 展示了新创建的通道的高度。

{\footnotesize
\begin{longtable}[h]{m{45pt} m{45pt} m{180pt} m{50pt} m{20pt}}
    \caption[创建通道测试用例]{创建通道测试用例} \label{channel_test}\\
        \toprule  
        \textbf{用例编号}&\textbf{测试编号}&\textbf{用户输入}&\textbf{预期结果}&\textbf{实际}\\
        \hline
        UC-Chan & TC4.1 & 选择OrdererMSP为排序组织在Org1MSP上创建通道 & 成功创建通道 & 通过 \\
        \hline
        UC-Chan & TC4.2 & 在通道内创建创世节点 & 成功创建创世节点 & 通过 \\
        \hline
        UC-Orderer & TC4.2 & 将ord-node1加入通道 & 加入成功 & 通过 \\
        \hline
        UC-Peer & TC4.3 & 将org1-peer0加入通道 & 加入成功 & 通过 \\
        \hline
        UC-Chan & TC4.4 & 指定锚节点 & 指定成功 & 通过 \\
        \hline
        UC-Chan & TC4.5 & 查看通道高度 & 查看成功 & 通过 \\
        \bottomrule
    \end{longtable} 
}

\begin{figure}[h] %figure环境,h默认参数是可以浮动,不是固定在当前位置。如果要不浮动,你就可以使用大写float宏包的H参数,固定图片在当前位置,禁止浮动。
    \centering %使图片居中显示
    \includegraphics[width=0.6\textwidth]{FIGs/chapter6/channel.png} %中括号中的参数是设置图片充满文档的大小,你也可以使用小数来缩小图片的尺寸。
    \caption{查看通道高度测试结果} %caption是用来给图片加上图题的
    \label{channel_test_result} %这是添加标签,方便在文章中引用图片。
\end{figure}%figure环境

\textbf{步骤五:} 如表\ref{chaincode_test}所示, 为测试通道内所有的参与节点按照链码的合同规则执行, HF开发人员需要在新创建的通道内安装官方提供的asset\footnotemark[1]\footnotetext[1]{\href{https://github.com/hyperledger/fabric-samples/blob/main/asset-transfer-basic/chaincode-go/chaincode/smartcontract.go}{asset链码}}链码, 安装完链码之后, 需要对其进行审批、提交。最后, HF开发人员可以调用链码接口对其进行初始化、查询等一系列操作。如图\ref{chaincode_test_result}所示仅为调用测试结果, 第\ref{section: smart_contract_microservice_test}节已经证明了智能合约微服务化的可行性。

{\footnotesize
\begin{longtable}[h]{m{45pt} m{45pt} m{180pt} m{50pt} m{20pt}}
    \caption[创建通道测试用例]{创建通道测试用例} \label{chaincode_test}\\
        \toprule  
        \textbf{用例编号}&\textbf{测试编号}&\textbf{用户输入}&\textbf{预期结果}&\textbf{实际}\\
        \hline
        UC-CC & TC5.1 & 安装链码并指定语言、label & 安装成功 & 通过 \\
        \hline
        UC-CC & TC5.2 & 查询已经安装的链码 & 查询成功 & 通过 \\
        \hline
        UC-CC & TC5.3 & 审批链码, 并提供链码ID以及策略 & 审批成功 & 通过 \\
        \hline
        UC-CC & TC5.4 & 提交链码, 并提供链码ID以及策略 & 指定成功 & 通过 \\
        \hline
        UC-CC & TC5.5 & 调用链码, 并提供链码调用函数 & 调用成功 & 通过 \\
        \bottomrule
    \end{longtable} 
}

\begin{figure}[h] %figure环境,h默认参数是可以浮动,不是固定在当前位置。如果要不浮动,你就可以使用大写float宏包的H参数,固定图片在当前位置,禁止浮动。
    \centering %使图片居中显示
    \includegraphics[width=1.0\textwidth]{FIGs/chapter6/chaincode.png} %中括号中的参数是设置图片充满文档的大小,你也可以使用小数来缩小图片的尺寸。
    \caption{查询链码} %caption是用来给图片加上图题的
    \label{chaincode_test_result} %这是添加标签,方便在文章中引用图片。
\end{figure}%figure环境


经过上述步骤1-5, 最终的网络节点状态如图\ref{fabric_result}所示, 本文搭建了一个单Orderer以及单Org、单Peer的经典HF网络, 并分别为Orderer、以及Peer组织搭配Ca证书认证。在搭建了基本的网络之后, 分别为Orderer、Peer注册并登记用户, 最后在组织Org中搭建了一个通道并成功安装调用并链码。由案例分析的结果可知, 本文的区块链云化框架及原型工具能够成功搭建HF网络节点并能够正确地部署、运行链码。

\begin{figure}[h] %figure环境,h默认参数是可以浮动,不是固定在当前位置。如果要不浮动,你就可以使用大写float宏包的H参数,固定图片在当前位置,禁止浮动。
    \centering %使图片居中显示
    \includegraphics[width=1.0\textwidth]{FIGs/chapter6/fabric_result.png} %中括号中的参数是设置图片充满文档的大小,你也可以使用小数来缩小图片的尺寸。
    \caption{网络节点状态} %caption是用来给图片加上图题的
    \label{fabric_result} %这是添加标签,方便在文章中引用图片。
\end{figure}%figure环境


\subsection{架构评估}

在软件工程中, 常见的评估软件体系结构的方法有Software Architecture Analysis Method(SAAM)、Architecture Trade-off Analysis Method(ATAM), 以上都是基于场景的质量属性评估方法\cite{ionita2002scenario}。SAAM相较于ATAM而言其结构以及评估过程相对简单, 需要准备的工作较少并且SAAM适用于对软件最终版本的评估\cite{huhonglei2004}。本文首先对第\ref{section: framework_characteristics}节所涉及到的设计原则的描述映射到对应的质量属性, 然后采取SAAM方法对其进行评估。

在SAAM评估方法开始之前, 本文邀请了4位熟悉HF基本概念及网络搭建流程的软件开发工程师分别两两扮演HF网络管理员、HF开发人员, 1位熟悉Kubernetes操作的集群运维工程师扮演集群管理员, 以上5位分别为本次SAAM评估方法的风险承担者。本节实施SAAM评估方法涉及以下步骤:

\textbf{1. 场景开发: }所有的风险承担者通过头脑风暴的方式, 提出反应自己需求的场景, 产出结果见步骤3。

\textbf{2. 软件架构描述: }本文详细的为各位风险承担者介绍面向Hyperledger Fabric的区块链云化框架及其原型工具, 包括框架设计理念与原理、原型工具功能和工具实现细节。

\textbf{3. 场景分类: }在这个过程中, 如表\ref{saam_step3}所示, 本文对步骤1所产出的场景进行分类并设定优先级。同时, 在分析时根据场景是否需要修改特定的体系结构把场景分为直接场景与间接场景。

{\footnotesize
\begin{longtable}[h]{m{20pt} m{60pt} m{90pt} m{40pt} m{40pt} m{40pt} m{30pt}}
    \caption[场景分类结果]{场景分类结果} \label{saam_step3}\\
        \toprule  
        \textbf{编号}&\textbf{风险承担者}&\textbf{场景描述}&\textbf{设计原则}&\textbf{质量属性}&\textbf{场景分类}&\textbf{优先级}\\
        \hline
        T1&  HF网络管理员 & 命令式启停HF网络中的任意节点 & 简单易用  & 功能需求 & 直接需求 &  高 \\
        \hline
        T2&  HF网络管理员 & 命令创建HF网络用户 & 简单易用 & 功能需求 & 直接需求 &  高 \\
        \hline
        T3&  HF开发人员 & 命令式创建通道 & 简单易用 & 功能需求 & 直接需求 &  高 \\
        \hline
        T4&  HF开发人员 & 持续交付部署链码 & 简单易用 & 功能需求 & 直接需求 &  高 \\
        \hline
        T5&  HF开发人员 & 命令式操作链码 & 简单易用 & 功能需求 & 直接需求 &  高 \\
        \hline
        T6&  HF网络管理员 \newline HF开发人员 & 能在30min内启动完整网络 & 简单易用 & 易用性 & 间接需求 &  高 \\
        \hline
        T7&  HF网络管理员 & 扩容HF节点资源 & 灵活扩展 & 可扩展性 & 间接需求 &  中 \\
        \hline
        T8&  HF网络管理员 & 扩容账本存储单元 & 灵活扩展 & 可扩展性 & 间接需求 &  高 \\
        \hline
        T9&  HF开发人员 & 保存用户密钥 & 安全可靠 & 安全性 & 间接需求 &  高 \\
        \hline
        T10&  HF开发人员 & 操作网络时需要身份认证 & 安全可靠 & 安全性 & 间接需求 &  高 \\
        \hline    
        T11&  k8s管理员 & HF网络与其他集群程序进行隔离 & 安全可靠 & 安全性 & 间接需求 &  高 \\
        \hline
        T12&  k8s管理员 & Operator限定管理HF网络节点 & 安全可靠 & 安全性 & 间接需求 &  高 \\
        \hline
        T13&  HF网络管理员 \newline HF开发人员 & 良好异常反馈机制 & 安全可靠 & 可靠性 & 间接需求 &  高 \\
        \hline
        T14&  HF网络管理员 \newline HF开发人员 & HF网络节点的全面监控能力 & 可视运维 & 可靠性 & 间接需求 &  高 \\
        \hline
        T15&  HF网络管理员 \newline HF开发人员 & Operator在不同云平台之间迁移的便捷性 & 云链结合 & 可移植性 & 间接需求 &  中 \\
        \bottomrule
    \end{longtable} 
}

\textbf{4. 场景评估: }这个过程重点针对间接场景, 并对其进行评估。针对直接场景, 第\ref{section: tool_test}节已经对其进行了较为全面的功能性测试。

简单易用, 本文同时邀请了这5位风险承担者对本文的原型工具进行易用性测试。为了排除网络的影响,
本文提前下载好原型工具搭建HF网络所需要的Docker镜像。在介绍了本工具的基本原理以及命令行的基本使用方法后, 5位风险承担者分别独立自行搭建不同规格的HF网络, 最终5位风险承担者都能够在30分钟内学习并掌握原型工具的使用方法。


\begin{figure}[h] %figure环境,h默认参数是可以浮动,不是固定在当前位置。如果要不浮动,你就可以使用大写float宏包的H参数,固定图片在当前位置,禁止浮动。
    \centering %使图片居中显示
    \includegraphics[width=1.0\textwidth]{FIGs/chapter6/db.png} %中括号中的参数是设置图片充满文档的大小,你也可以使用小数来缩小图片的尺寸。
    \caption{网络存储状态} %caption是用来给图片加上图题的
    \label{db} %这是添加标签,方便在文章中引用图片。
\end{figure}%figure环境

灵活扩展, 在HF网络节点资源方面, 由于HF网络节点的CR中配置了HF网络节点运行态时所需的资源(CPU、内存)大小, 所以当对节点CR进行修改时, Controller会监听到CR的变化并更新运行中节点Deployment的资源大小。在数据存储的可扩展性方面,如图\ref{db}所示, 原型工具为每个HF网络中的工作节点设置了专属存储单元。每个存储单元配置专属的PVC管理, HF网络管理员可以不仅可以通过修改节点的CR进行扩容, 可以直接在Kubernetes中对PVC进行动态扩容。值得注意的是, 由于存储单元PVC扩容依赖于StorageClass, 当前只有AWS-EBS、GCE-PD、Azure磁盘、Azure文件、Glusterfs、Cinder、Portworx和Ceph RBD数据卷插件才能支撑数据扩容操作。


安全可靠, 原型工具通过可以以文件形式导出HF网络登记用户产生的密钥信息, 并可以通过Secret进行保存。当这些用户操作HF网络时, 原型工具会要求提供对应的密钥文件作为身份认证的方式。同时, 原型工具将所有的HF资源放在同一命名空间下, 并通过RBAC为该命名空间提供了两种类型的角色。当集群其他用户对已经部署的HF网络节点拥有超越权限的操作时, 原型工具会提示并不进行相关操作。在可靠性方面, 本文在使用原型工具搭建HF网络过程中有意输入错误的命令行参数, 输入不正确的密钥文件等, 原型工具都能在1s内给出参数的异常情况; 同时, 原型工具能够有效利用Prometheus监控体系对工具本身以及HF网络的所有节点、链码以及DB进行监控, 如图\ref{monitoring}展示了利用Prometheus与Grafana对Ca Resource进行监控的画面, 图中展示了Ca CPU的使用情况。在搭建完HF网络之后, 原型工具在两周内仅崩溃1次, 且能通过日志以及Prometheus的告警设置在5分钟内进行恢复。

\begin{figure}[h] %figure环境,h默认参数是可以浮动,不是固定在当前位置。如果要不浮动,你就可以使用大写float宏包的H参数,固定图片在当前位置,禁止浮动。
    \centering %使图片居中显示
    \includegraphics[width=0.9\textwidth]{FIGs/chapter6/ca_cpu.png} %中括号中的参数是设置图片充满文档的大小,你也可以使用小数来缩小图片的尺寸。
    \caption{Ca CPU监控图} %caption是用来给图片加上图题的
    \label{monitoring} %这是添加标签,方便在文章中引用图片。
\end{figure}%figure环境

云链结合, 重点是在可移植性下的场景。本文通过Dockerfile将本工具打包成Docker镜像\footnotemark[1]\footnotetext[1]{\href{https://hub.docker.com/repository/docker/zhangfuli/hfoperator}{HFOperator镜像}}, 使用预先打包并经过检验的镜像, 并为该原型工具配备专用的Helm chart, 其中包含有关原型工具构建的所有必要信息, 这样可以以最少的时间部署。结果表明, 原型工具可以通过Helm在支持Kubernetes 1.18版本以上的云平台上运行, 原型平台具备良好的可移植性。

\textbf{5. 总体评估: }本文的原型工具可以有效利用Kubernetes Operator云化策略来提升HF网络节点的易用性、可扩展性、安全性、以及可靠性。


\section{成熟度衡量}

\begin{figure}[h] %figure环境,h默认参数是可以浮动,不是固定在当前位置。如果要不浮动,你就可以使用大写float宏包的H参数,固定图片在当前位置,禁止浮动。
    \centering %使图片居中显示
    \includegraphics[width=0.95\textwidth]{FIGs/chapter6/maturity.pdf} %中括号中的参数是设置图片充满文档的大小,你也可以使用小数来缩小图片的尺寸。
    \caption{成熟度模型} %caption是用来给图片加上图题的
    \label{maturity} %这是添加标签,方便在文章中引用图片。
\end{figure}%figure环境


如图所示\ref{maturity}, Kubernetes Operator拥有5个成熟度级别的定义\cite{duan2021case}, 其通过定性的方式分析某个Operator应用是否达到了某个级别, 这5个成熟度级别定义如下:

\begin{itemize}[itemindent=2em]
    \item 第一级: Basic Install, 该级是Operator成熟度模型中最基本的级别。在该级中, 用户能够使用CRD对目标程序进行配置和安装;

    \item 第二级: Seamless Upgrades, 在该级别上的Operator能够不丢失数据的升级所管理的工作负载;

    \item 第三级: Full Lifecycle, 是否能达到该级别取决于Operator是否具备生命周期管理和的数据备份、恢复能力。在该级别上的Operator能够备份数据, 并在发生任何数据灾难时从备份数据中恢复数据;

    \item 第四级: Deep Insights, Operator提供监控和报警等功能。在该级别, Operator能够包含所有组件的运行状况指标, 并根据指标配备报警功能;

    \item 第五级: Auto Pilot, 最终级别拥有许多高级功能, 如自动伸缩。Operator可以根据收集到的指标来扩展工作负载。

\end{itemize}

\textbf{Basic Install}

借助于原型工具, 用户能够通过命令行方式一键启停HF网络中任意节点, 而且不需要进行复杂的证书管理。一旦原型工具被部署在Kubernetes网络中, 原型工具就可以对CR以及Helm进行管理, HF网络管理人员可以借助命令行参数的方式进行创建、更新CR以此来创建特定规格的HF网络。 一旦CR进行更新, 原型工具就会将当前状态调整为与指定状态一致。

\textbf{Seamless Upgrades}

在原型工具中, HF网络管理员可以通过修改CR的内容对包括HF网络节点镜像、端口、Host、版本等进行无缝修改。此外, 由于原型工具依赖于链外存储的CouchDB以及外部的Prometheus监控体系, 这两者的升级并不会对原型工具产生严重的负面影响。 

\textbf{Full Lifecycle}

虽然原型工具能够在Mangager中对结合Helm对HF网络进行全生命周期管理, 但目前原型工具目前仍缺少数据备份和恢复的能力。要达到这一能力需要在CR中指定远程备份的数据存储平台, 并在CR中指定数据备份的凭证以及远程备份的链接。同时, 原型工具需要在一定的时间周期内将数据备份到远程的数据存储平台上。

\textbf{Deep Insights}

原型工具利用Prometheus监控体系实现监控与报警的功能, 与此同时, 除了利用Prometheus抓取Pod的基本指标外, CRD中还设定了PodMonitor、ServiceMonitor、CouchDBExporter接口, 可以让Prometheus更全面的抓取HF网络的监控指标。HF网络管理员可以通过Grafna可视化图表查看整体HF网络运行状态, 并且能够根据监控指标创建自定义的告警规则。

\textbf{Auto Pilot}

在Kubernetes中存在两种自动伸缩的插件, 即HPA、VPA。当负载超过一定的阈值时, 就会对其进行伸缩或配置更多的资源。然而在HF网络中, 每个Peer都有记录全部账本的职责, 并且只需要超过51\%的Peer节点保持一致即可, 所以针对于Peer并不需要根据监控进行自我伸缩的能力。在数据存储方面, 随着账本的膨胀, 原型工具可以针对链外存储进行扩容。

综上, 通过定性分析, 本文原型工具利用Operator管理HF网络能够完美的具备第一级、第二级的能力, 在第三级上能够支持全生命周期管理但是对于数据的备份能力依旧是存在欠缺, 借助Prometheus实现了第四级的监控, 对于自动伸缩方面支持资源及数据层面的扩展, 所以本文原型工具基本满足5层成熟度模型的功能。

\section{对比分析} \label{section: tool_comparison}

除上述通过定性分析的手段对原型工具进行评估外, 本文选取了Hyperledger官方推出的BaaS平台Cello进行定量的对比分析。本文重点关注的是对HF网络节点的云化问题, 所以需要在网络部署时间与链码交付时间上对Cello以及原型工具进行对比。

本文在云主机环境下拉取Cello的release-0.9.0-h3c并进行打包构建以及运行。Cello通过图形化的Cello Operator进行主机绑定、组织管理、网络管理以及用户管理。操作Cello Operator的就是HF网络管理员, 管理员首先在主机管理中添加主机, 主机就是Cello将要部署的目标Docker或者Kubernetes环境, 然后再依次创建组织并启动网络, 整个过程全都通过图形化界面的方式完成。

% 如图\ref{cello}所示,
% \begin{figure}[h] %figure环境,h默认参数是可以浮动,不是固定在当前位置。如果要不浮动,你就可以使用大写float宏包的H参数,固定图片在当前位置,禁止浮动。
%     \centering %使图片居中显示
%     \includegraphics[width=0.95\textwidth]{FIGs/chapter6/cello.png} %中括号中的参数是设置图片充满文档的大小,你也可以使用小数来缩小图片的尺寸。
%     \caption{图形化Cello Operator} %caption是用来给图片加上图题的
%     \label{cello} %这是添加标签,方便在文章中引用图片。
% \end{figure}%figure环境


{\footnotesize
\begin{longtable}[h]{m{100pt} m{100pt} m{100pt}}
    \caption[Hyperledger Fabric版本信息]{Hyperledger Fabric版本信息} \label{test_fabric_net} \\
        \hline  
        \textbf{镜像}&\textbf{Cello版本}&\textbf{原型工具版本}\\
        \hline
        Fabric-Ca & 1.4.2 & 1.4.9 \\
        Fabric-Peer & 1.4.2 & 2.4.1 \\
        Fabric-Orderer & 1.4.2 & 2.4.1 \\
        \hline
    \end{longtable}
}

由于现在在Docker环境中部署HF网络仍旧是主流, 所以本文利用Cello在Docker中部署作为测试基准。由于Cello的局限性, 仅支持在部署HF网络的1.4版本, 而原型工具能更优地支持HF网络2.X以上版本。
如表\ref{test_fabric_net}展示了本次工具对比所部署的HF网络各节点的版本信息, 测试基准为基于单组织单Peer的网络部署时间, 共识算法选择Solo, 数据存储选择LevelDB。


为了避免人为的手工干扰, 获得更加准确的网络部署时间。本文提前在Cello Operator中创建好Orderer以及org1组织并为每个组织配置一个对应的节点。当点击提交网络时开始计时, 刚开始创建时, 网络节点的状态是“故障”, 当网络节点的状态从“故障”变成“正常”时停止计时, 随后删除该网络。重复10次上述操作, 且为避免后端镜像遗留干扰, 每次操作间隔3min~5min。在原型工具中, 为避免手工输入命令而带来的人为误差, 本文预先编写好创建网络的脚本, 脚本中创建10次网络, 创建完成后删除该网络并在删除后休眠30s, 如此循环10次共得到10次网络启动时间如表\ref{net_deployment_time}所示。

{\footnotesize
\begin{longtable}[h]{m{35pt}|m{40pt}|m{15pt} m{15pt} m{15pt} m{15pt} m{15pt} m{15pt} m{15pt} m{15pt} m{15pt} m{15pt}|m{20pt}}
    \caption[网络部署时间(单位: 秒(s))]{网络部署时间(单位: 秒(s))} \label{net_deployment_time}\\
        \hline
        \multirow{2}*{工具类型}
        & \multirow{2}*{\parbox[c]{40pt}{节点类型}}
        & \multicolumn{10}{c|}{序号}
        
        & \multirow{2}*{\parbox[c]{20pt}{平均}}\\
        \cline{3-12}
        & & 1 & 2 & 3 & 4 & 5 & 6 & 7 & 8 & 9 & 10 & \\
        \hline
        Cello & 整体网络 & 73 & 119 & 115 & 90 & 73 & 89 & 137 & 85 & 103 & 127 & 101.1\\
        \hline  
        \multirow{5}*{\parbox[c]{40pt}{原型工具}}
        & ca(peer) & 13 & 12 & 13 & 13 & 13 & 13 & 13 & 12 & 12 & 12 & 12.6 \\
        & peer & 20 & 21 & 29 & 16 & 23 & 27 & 20 & 20 & 16 & 22 &  21.4 \\
        & ca(ord) & 13 & 13 & 13 & 13 & 13 & 13 & 14 & 13 & 15 & 15 & 13.5 \\
        & orderer & 27 & 36 & 34 & 24 & 33 & 24 & 27 & 25 & 32 & 25 & 28.7 \\
        \cline{2-13}
        & 整体网络 & 73 & 82 & 89 & 66 & 82 & 77 & 74 & 70 & 75 & 74 & 76.2\\
        \hline
    \end{longtable} 
}

\begin{figure}[h] %figure环境,h默认参数是可以浮动,不是固定在当前位置。如果要不浮动,你就可以使用大写float宏包的H参数,固定图片在当前位置,禁止浮动。
    \centering %使图片居中显示
    \includegraphics[width=1.0\textwidth]{FIGs/chapter6/plt_deployment.png} %中括号中的参数是设置图片充满文档的大小,你也可以使用小数来缩小图片的尺寸。
    \caption{网络部署时间对比图} %caption是用来给图片加上图题的
    \label{plt_deployment} %这是添加标签,方便在文章中引用图片。
\end{figure}%figure环境

如图\ref{plt_deployment}展示了原型工具和Cello分别部署HF网络的对比图。值得注意的是, 由于Cello采用图形化界面无法配置单个Ca的启停, 所以采用Cello部署的单Orderer单组织的HF网络仅在单个组织内部启动了一个Ca节点, 共计3个网络节点。 而采用原型工具部署的单Orderer单组织的HF网络分别为Orderer以及单组织部署了一个Ca节点, 共计4个网络节点。图\ref{plt_deployment}左侧展示了10次部署排序后的时间曲线图, 图中可以较为直观的看到原型工具比Cello部署的时间要少, 原型工具部署整体网络的平均时间为76.2秒, 而Cello部署整体网络的平均时间为101.1秒; 经计算, 原型工具的总体标准偏差约为6.29, Cello的总体标准偏差约为21.41。图\ref{plt_deployment}右侧分别展示了两者的箱线图, 结合标准偏差可知原型工具部署时间相对而言更为稳定, 尤其在部署Ca节点时, 稳定在13秒左右。

通过分析源码\footnotemark[1]\footnotetext[1]{\href{https://github.com/hyperledger/cello/blob/release-0.9.0-h3c/src/modules/blockchain_network.py}{cello create network}}可知, 当在Cello Operator前端中提交网络之后, Cello对应的Docker Agent会循环解析传入的HF网络节点信息及数量的数据结构, 解析完成后串行构建HF网络中的各类节点。构建过程中, 生成Docker Compose文件并启动对应的网络节点。同样, 当Cello选择在Kubernetes部署时根据Template模板在代码中硬编码生成Yaml文件。由此可得, Cello网络部署时间会随着HF网络中节点数量的增加而线性递增。本文的原型工具, 因需要手动以命令方式部署对应的网络节点, 其部署时间也会随着节点数量增加而递增。因此, 只对单Orderer单Peer进行对比测试即可预估出不同网络规模下的网络部署时间。

在链码部署方面, Cello与原型工具都需要经过创建通道、将账本节点加入至通道内、安装链码等过程。在Cello中上述过程均在图行化界面中配置完成, 原型工具通过命令行方式构建。这两者在安装链码这个环节均在1s内完成, 这对于HF开发人员而言都是能够接受的, 所以这两者仅在安装链码这个环节不需要进行对比。智能合约微服务化的优势在于将智能合约进行解耦, 将智能合约单独作为可交付物。同时开发人员可以利用传统成熟的工具链扩充智能合约开发运维领域的工具空白, 利用已有成熟工具对智能合约进行持续测试、持续交付与持续监控。

{\footnotesize
\begin{longtable}[h]{m{35pt}|m{15pt} m{15pt} m{15pt} m{15pt} m{15pt} m{15pt} m{15pt} m{15pt} m{15pt} m{15pt}|m{20pt}}
    \caption[链码交付时间(单位: 秒(s))]{链码交付时间(单位: 秒(s))} \label{cc_deployment_time}\\
        \hline
        \multirow{2}*{工具类型}
        & \multicolumn{10}{c|}{序号}
        & \multirow{2}*{\parbox[c]{20pt}{平均}}\\
        \cline{2-11}
        & 1 & 2 & 3 & 4 & 5 & 6 & 7 & 8 & 9 & 10 & \\
        \hline
        Cello  & 141 & 128 & 185 & 153 & 166 & 105 & 137 & 160 & 186 & 117 & 147.7\\
        \hline  
        \multirow{1}*{\parbox[c]{40pt}{原型工具}}
        & 119 & 89 & 100 & 76 & 104 & 140 & 136 & 125 & 83 & 164 &  113.6 \\
        \hline
    \end{longtable} 
}

本文对比链码部署过程中所有消耗的时间, 即HF开发人员编写完链码后到将链码成功部署于HF网络节点上所用的时间。如表\ref{cc_deployment_time}展示了原型工具和Cello分别部署asset链码所用的时间对比表。Cello部署链码依次为: 创建通道、加入节点、代码仓库下载链码、压缩链码、计算压缩包MD5、上传链码、安装链码; 原型工具在持续交付流水线中的流程为: 创建通道、加入节点、代码仓库下载链码、打包链码镜像、安装链码。值得注意的是, 本次由于人工操作以及网络波动等情况, 数据可能会存在偏差。最终, Cello的部署链码的平均时间为147.7秒, 原型工具部署链码的平均时间为113.6秒。

综上, 本节对Cello与原型工具进行了定量的对比分析。在网络节点部署方面, 本文原型工具相较于Cello在具有更优的节点部署时间, 并且每次节点部署时间更加稳定。虽然在链码的部署时间方面原型工具表现的较为良好, 但智能合约微服务化的开发运维流程相较于Cello固定式的部署流程能够在流水线中纳入更多的环节如智能合约代码质量、安全等, 以促进智能合约更全面的发展。

\section{本章小结}

本章介绍了对原型工具的测试与评估。首先介绍了本文涉及到的两个测试环境, 并进行了三个维度的评估工作: (1)基本能力自证, 利用典型案例研究的方式进行功能可行性测试, 利用SAAM架构评估方式进行质量属性验证; (2)成熟度衡量, 利用五层成熟度模型的定性评估;(3)对比分析, 与Cello对比的定量分析。


\chapter{总结与展望}

\section{总结}

领域驱动设计作为一种针对复杂业务流程的分析与建模方法,
正在成为大型分布式系统设计与实现的最佳解决方案,
其中的战术建模层次可以帮助更快速地进行建模设计与实现代码的落地。
然而,战术建模在应用时仍面临着许多挑战,
由于对战术模式的规范和约束不明确,往往会导致战术建模结果不够标准和规范;
开发人员和架构师对战术建模理解程度不同,无法统一和复用建模结果;
领域建模过程缺少相应的支持平台和工具。
为了解决这些挑战,
对该领域的研究进展与工业界实践经验进行了调查与总结,
提出了一套战术建模支持方法及工具。

具体来说,本文所提出的战术建模支持方法及工具包括以下三个具体贡献。
其一,通过对领域驱动设计战术建模过程的理论调研和对工业界从业人员的访谈,
本文总结得出一套战术建模指南,
包括八种战术模式、这些模式的重要属性、使用时机以及实现技术,
该建模指南经过工业界实践经验认证,可以作为战术建模实践时的指导。
其二,通过开展焦点小组讨论,本文构建了一种战术建模语言,
该战术建模语言作为使用战术建模支持方法的支撑,
提升了建模的效率与准确性。
以上两点构成了战术建模支持方法的主要内容。
其三,以所提出的战术建模语言为基础,
本文还实现了一个建模支持工具,
实现了灵活易用的可视化建模过程,
保证了扩展性与通用性,并避免了对特定平台的依赖。

本文提出的战术建模支持方法和工具,支持简单了解领域驱动设计战术建模基本概念,
通过灵活的可视化建模来实现建模过程,并校验建模结果的标准性与规范性,
为战术建模提供了一套可靠的流程。此外,还支持将建模结果以多种格式导出和存储,
还能为实现阶段构建框架项目,提高了战术建模结果的复用程度。
总体而言,
本文提出的工具也降低了使用战术建模的最低要求,
对领域驱动设计在实践中发展具有积极意义。

\section{展望}

本文提出的领域驱动设计战术建模支持方法及工具可以支持战术建模实践,
但仍有很多方面可以进一步提升。首先,
战术建模的属性和实现技术在实践时还应该做到根据业务特性进行变通,
对不同开发团队的不同业务,战术建模工具应该有不同的个性化配置;
其次,战术建模工具对于战术建模的结果利用程度还不够高,
除了生成框架项目之外,
未来该工具还将支持更多种形式与更细粒度的扩展结果;
最后,战术建模工具在多人协同建模方面支持度不够,仅能通过共享建模结果来进行协同,
未来该工具还将支持通过在线协作实现即时协同建模,提升建模过程的体验并提高建模效率。



% \chapter{论文引用}

\section{引文相关}
此处的论文引用采用的是类似于IEEE的按出现位置的数字编号格式。建议将被引用的论文全名放入dblp网站(必应谷歌搜索dblp)搜索,之后进入该论文详细信息,如图~\ref{fig_dblpForBibtexCH7} 所示。

\begin{figure}[htb]
  \centering
  \includegraphics[width=5in]{figure/chapter7/dblpForBibtex.pdf}
  \caption{在dblp上下载Bibtex}\label{fig_dblpForBibtexCH7}
\end{figure}

点击图~\ref{fig_dblpForBibtexCH7}中所示的链接之后将得到Bibtex信息,如图~\ref{fig_bibtexDetailCH7}所示。
打开本地文件夹下的sample.bib文件,完整添加该信息。
并在需要引用的位置添加这一引用~\cite{DBLP:journals/computer/EgyedZHD18}。
格式为bibtex信息中的开头,\emph{例如图中的“DBLP:journals/computer/EgyedZHD18”。(此处是一个典型的因为长字符串导致的bad box,请参考上述章节的内容手动进行软换行)}。

\textbf{注意:在修改并保存sample.bib文件后,先编译tex文件,再编译参考文献,之后再编译两次,此时引用位置的方括号内将出现具体的编号而不是问号,且编号对应的引文已经出现在参考文献内。本模板点击引用编号后可以跳转到对应的参考文献处。}

在bib文件中出现,但并未在论文中被引用的论文不会出现在最后的参考文献中。如果dblp中并未包含你需要的论文,则可以尝试谷歌或百度学术的搜索结果,一般也包含bibtex信息,但可能不完整或不规范。

引用网站链接可以考虑这一格式~\cite{GanttSystemWeb}(不推荐,网站链接使用脚注更规范些)。

引用书籍可以考虑这一格式~\cite{Pohl2010Requirements}。

中文文献请参考这一格式~\cite{cyg2006}(引用标记不要使用中文,否则容易出现编译错误)。

以下英文引用用来测试引文排序是否按照插入顺序,以及多引文是否合并~\cite{DBLP:journals/computer/EgyedZHD18, DBLP:journals/ml/TingZCZWZ19}

\begin{figure}[htb]
  \centering
  \includegraphics[width=5in]{figure/chapter7/bibtexDetail.pdf}
  \caption{Bibtex详细信息}\label{fig_bibtexDetailCH7}
\end{figure}


% \chapter{标题}

这是章节标题。
注:一般而言,标题不要比小节标题更小,即不要出现1.2.3.4这种标题(本模板支持此类标题,即Subsubsection)。

\section{这是节标题}

每个章节标题下面都可以插入文字,一般可用于概述下面章节的内容

\subsection{这是小节标题}

此外本模板中的每个章节先保存在独立的tex文件里(本章节文件名为Title.tex,位于本地的chapter文件夹),再通过input命令引入主文件(例如,sample.tex)。
这样做的好处是减少每个文件的行数,便于浏览和维护。
缺点在于有些编辑器编译的时候要求回到主文件进行编译(如WinEdt),或在pdf文件向tex文件跳转的时候定位不准(如TeXStudio)。
如果不能接受上述问题,也可以删除input命令,并将引用文件的全部内容逐一放入主文件。
TeXStudio本身提供对章节索引的管理,可以缓解文件过长的问题。


% \chapter{正文}

需要指出的是本模板使用XeLaTeX编译,要求每个文件都是utf-8编码。
如有使用过CTeX经验的同学,需注意CTeX是使用pdfLaTeX进行编译,并使用GBK编码处理汉字以及CJK(中日韩)字符。
因此,如果要从CTeX源文件复制内容到本模板,必须做编码转化,否则会出现乱码及各种问题。

\section{正文书写的小技巧}
主流的LaTeX编辑器一般都自带一个输出pdf文件查看的功能,并支持在选中文字的区域后跳转到相应的pdf文件或tex文件(所谓的“反复橫跳”),从而尽可能的实现编译后即可得。
以TexStudio为例,在任何一个文件的文字区域点击鼠标右键,即可发现“跳转到源”或者“跳转到pdf”的提示。

只有间隔一个明显的换行才会自然段分段(参见源文件)。

因此,建议把一个自然段中的每句话都单独作为一行。
这样的好处是,每次双击一句话,都可以回到编辑器中具体的一行,方便定位(参见源文件)。

如果觉得TeXStudio自带的pdf查看器不好用,也可以外挂著名的SumatraPDF,并设置正向和反向搜索,能够在论文分章节文件的情况下也做到精准定位(需保持源文件及时更新),具体配置见如下链接 \footnote{https://blog.csdn.net/lizuoxin/article/details/48173907},亲测可行。

注意,配置SumatraPDF时要打开编译过的pdf文件(例如,sample.pdf,会识别出该文件背后存在一个gz文件),才能弹出反向搜索框。而TeXStudio自带的正向搜索命令已失效,要用默认快捷键调出配置中的用户自定义命令。

\section{一些正文中的标记}
\emph{斜体} 与 \textbf{加粗},以及代码格式\texttt{Source Code Pattern}。

\begin{center}
居中,左右对齐同理。
\end{center}

这里再次展示脚注。\footnote{数字列举和圆点列举见摘要部分}

一个小建议,中文后直接跟上述格式标记(包含各种引用)可能会出现一些问题。
因此,在中文字和格式标记的斜杠之间加入~\emph{一个波浪号}是一个常用的习惯。
双~~波~~浪~~线等价于一个强制空格,有时比键盘输入的空格要好用。


\section{注意软换行的使用}
论文一般会引用代码,本模板建议将代码声明为~\texttt{class.this()}格式。
在引用代码时,较长的函数名有时会导致函数名超出文本边界的情况,此时可以考虑手动进行软换行,请参考以下例子。

“图XX 展示了从AquaLush 系统中抽取的函数调用依赖示例,其中~\texttt{UICon-} \linebreak \texttt{troller.buildLogScrn()} 是为了实现新功能“the control panel shows log message”而在新版本中添加的函数。”


% \input{chapter/Table.tex}

% \chapter{图形}

\section{基本图形}
相对于表格而言,LaTeX中的图形就简单多了,需要注意的是本模板推荐将所有图形都转化为pdf,具体内容参见图~\ref{fig_errorExpCH4}。
该图形放在本模板的本地文件夹figure中。
图~\ref{fig_errorExpCH4}是将Excel的五个子图形排布在一个ppt页面上,之后保存为pdf文件,最终得到的图形可以保证是矢量图。


\begin{figure}[htb]
  \centering
  \includegraphics[width=5in]{figure/chapter4/errorExpCH4.pdf}
  \caption{以含错误的RTM为输入的五个系统上三个实验(Call,Data,Call+Data)的错误率(Incorrectness)}\label{fig_errorExpCH4}
\end{figure}

\textbf{注意:不要删除项目下面的njulogo、njuname和reviewPlaceholder这三个文件,分别是论文封面的校徽、手写体南大校名以及盲审时的空白占位符。}

\section{引用代码}

\begin{figure}[htb]
  \centering
  \includegraphics[width=\linewidth]{figure/chapter4/VoDCodeSample.pdf}
  \caption{VoD系统中的代码片段}\label{fig_VoDCodeSample}
\end{figure}

这里给出一个代码引用的推荐实践。
引用代码时先将代码放入word的文本框中,调整结束后,将该文本框页面另存为pdf文件,之后再作为图形来引用,如图~\ref{fig_VoDCodeSample}所示。

\section{其它图引用}

这里给出原模板提供的插图例子,请注意多行多图的设置方式。

一行一图,如图\ref{fig:line}。
\begin{figure}[htbp]
	\centering
	\includegraphics[width=0.7\textwidth]{figure/line.png} % requires the graphicx package
	\caption{待分行文本}
	\label{fig:line}
	%\vspace{0.8cm} % 用来调整和下方文字的间距
\end{figure}


一行两个图,如图\ref{fig:lstm}。
\begin{figure}[ht!]
	\centering
	\begin{subfigure}{.5\textwidth}
		\centering
		\includegraphics[width=0.9\textwidth]{figure/lstm1.png}
		\caption{长短时记忆单元模块}
	\end{subfigure}
	\begin{subfigure}{.4\textwidth}
		\centering
		\includegraphics[width=0.8\textwidth]{figure/lstm2.png}
		\caption{深双向长短时记忆}
		\label{fig:lstm2}
	\end{subfigure}
	\caption{(a)一个长短时记忆单元模块。(b)深度双向长短时记忆的结构。}
	\label{fig:lstm}
\end{figure}

多行多图,如图\ref{fig:multi}。注意源文件中的双空行起到了子图换行的作用。
子图中大小不一是有意为之,请留意源码中subfigure和includegraphics后面的命令与四个子图大小之间的关系。

\textbf{注意:后续连续出现图形是最终文档中需要避免的情况,一般而言出现这种情况都是图贴的太多,文字写的太少导致的。建议针对每个图或表都采用“三段论”,即给出图表之前先介绍图表的大致情况与理由,然后给出图表,在图表展示之后再对图表中的内容进行讨论。}

\begin{figure}[ht!]
	\centering
	\begin{subfigure}{.69\textwidth}
		\centering
		\includegraphics[width=1.0\textwidth]{figure/line1.png}
		\caption{全局损失切割第一行}
		\label{fig:line1}
	\end{subfigure}
	\begin{subfigure}{.3\textwidth}
		\centering
		\includegraphics[width=1.0\textwidth]{figure/line2.png}
		\caption{局部损失切割第一行}
		\label{fig:line2}
	\end{subfigure}
	
	
	\begin{subfigure}{.49\textwidth}
		\centering
		\includegraphics[width=0.5\textwidth]{figure/line1.png}
		\caption{全局损失切割第二行}
		\label{fig:line1}
	\end{subfigure}
	\begin{subfigure}{.49\textwidth}
		\centering
		\includegraphics[width=0.8\textwidth]{figure/line2.png}
		\caption{局部损失切割第二行}
		\label{fig:line2}
	\end{subfigure}
	\caption{分行结果比较。(a)全局损失切割;(b)局部损失切割;(c)缩放的全局损失切割;(d)缩放的局部损失切割}
	\label{fig:multi}
\end{figure}

\newpage %为了将图片实例放在一起,另起一页,使用时请删掉




% \chapter{公式}

这里直接给出几个较为复杂的公式的例子,可一一进行参照。
若有未包含的数学符号或公式格式,请参阅本模板所包含的手册(本地manual文件夹)或百度必应谷歌。
介绍公式时不妨也采用下面的方式,即先介绍公式的目的,给出公式,并逐一介绍公式中的变量。

\section{公式5.1与论证}
“从直接代码依赖的角度出发,从一个初始域外的类$C_{out}$ 出发我们尝试找到一个通往初始域内的类$C_{in}$ 的路径。一条合法的路径需要满足以下两点要求:(1)这一路径是单向的,即$C_{out}$ 传递性地到达$C_{in}$ 或$C_{in}$ 传递性地到达$C_{out}$;(2)路径中只能包含一个$C_{in}$ (为了避免重复路径的出现)。为了恰当的估计一条合法路径所代表的交互程度,我们计算路径上所有直接代码依赖的紧密度值的几何平均。我们用如下公式来重新计算给定$C_{out}$ 的IR 值($IR_{DC}$):”

\begin{align}
IR_{DC}=IR_{origin}+(IR_{top}-IR_{origin})^{\left| PATH\right|}\sqrt {\prod _{x \in PATH}Closeness_{DC}(x)} \end{align}

“其中$IR_{origin}$ 代表$C_{out}$ 的初始IR值,$IR_{top}$ 代表$C_{in}$ 被提升过的IR值,\emph{PATH} 代表$C_{out}$ 与$C_{in}$ 之间的路径内所有的直接代码依赖,而$Closeness_{DC}(x)$ 则代表每一条直接代码依赖关系的紧密度值。在同一对$C_{out}$ 和$C_{in}$ 之间可能存在多条合法路径,我们只保留其中能使$IR_{DC}$ 值最大的那条路径。”

\section{公式5.2与论证}

“由于IR 方法返回的是一个按照IR 值大小倒序排列的候选线索列表,因此一种常用的比较IR 方法的方式是在不同的查全率水平上比较不同方法之间的精确度,通常用$Precision-Recall$ 曲线表示。为了进一步衡量IR 方法返回结果的整体质量,我们选用了另外两个常用的实验度量:$AP$(Average Precision)与$MAP$ (Mean Average Precision)。其中,$AP$ 用于度量全部查询(需求)所检索的相关文档的排序质量,计算方式如下:”
\begin{align}
AP=\dfrac {\sum _{r=1}^{N}\left( Precision\left( r\right) \times isRelevant\left( r\right) \right) } {\left| RelevantDocuments\right| }
\end{align}

“其中,$r$ 表示被查询对象(类)在列表中的排序,$Precision(r)$ 表示前$r$ 个类的准确率。$isRelevant()$ 为一个二值函数,如果文档是相关的,则返回1,若无关,则返回0。”

\section{公式5.3与论证}

“由此,我们为类数据依赖定义紧密度$Closeness_{CD}$ 如下:”

\begin{align}Closeness_{CD}=\frac {\sum _{x \in \{DT_{i}\cap DT_{j}\}}idtf(x)} {\sum _{y \in \{DT_{i}\cup DT_{j}\}}idtf(y)}\end{align}

“其中$idtf(x)$ 代表共享数据类型的idtf值,$DT_i$ 与$DT_j$ 的交集代表该数据依赖上的共享数据类型,而$DT_i$ 与$DT_j$ 的并集则代表$C_i$ 和$C_j$ 在全部代码上所访问的数据类型。$Closeness_{CD}$ 的取值范围是0到1之间。”

\section{原模板中的其它公式}

\begin{equation}
\frac{\partial L}{\partial a_{k}^t} = {d(s)}^2 (y_{k}^t - \frac{\sum_{lab(\mathbf{l},k)} \alpha_t(s)\beta_t(s) }{y_{k}^t} )
\end{equation}

\begin{equation}
\begin{aligned}
d_{{0j}}&=\sum _{{k=1}}^{{j}}w_{{\mathrm  {ins}}}(a_{{k}}),\quad &{\text{for}}\;1\leq j\leq n\\
d_{{ij}}&={\begin{cases}d_{{i-1,j-1}}&{\text{for}}\;a_{{j}}=b_{{i}}\\\min {\begin{cases}d_{{i-1,j}}+w_{{\mathrm  {del}}}(b_{{i}})\\d_{{i,j-1}}+w_{{\mathrm  {ins}}}(a_{{j}})\\d_{{i-1,j-1}}+w_{{\mathrm  {sub}}}(a_{{j}},b_{{i}})\end{cases}}&{\text{for}}\;a_{{j}}\neq b_{{i}}\end{cases}}\quad &{\text{for}}\;1\leq i\leq m,1\leq j\leq n.
\end{aligned}
\end{equation}

\begin{equation}
\begin{aligned}
&\beta_T(|l{}'|)=y_{b}^{T}\\
&\beta_T(|l{}'|-1)=y_{l_|l|}^{T} \\
&\beta_T(s)=0, \forall s < |l{}'|-1
\end{aligned}
\end{equation}

递归公式举例(出现了公式过长的问题,实践中最好适当控制长度,给公式编号在同一行上留下位置)。
\begin{equation}
\beta_t(s)=\left\{
\begin{aligned}
& (\beta_{t+1}(s) d(s)+\beta_{t+1}(s+1))d(s+1)\,  y_{\l_s{}'}^t, \: \: if \:  l_s{}'=b \:  or \:  l_{s+2}{}'=l_s{}'\\
& (\beta_{t+1}(s) d(s)+\beta_{t+1}(s+1)d(s+1)+\beta_{t+1}(s+2)d(s+2))\,  y_{\l_s{}'}^t,\: \:   otherwise
\end{aligned}
\right.
\end{equation}



% \chapter{论文引用}

\section{引文相关}
此处的论文引用采用的是类似于IEEE的按出现位置的数字编号格式。建议将被引用的论文全名放入dblp网站(必应谷歌搜索dblp)搜索,之后进入该论文详细信息,如图~\ref{fig_dblpForBibtexCH7} 所示。

\begin{figure}[htb]
  \centering
  \includegraphics[width=5in]{figure/chapter7/dblpForBibtex.pdf}
  \caption{在dblp上下载Bibtex}\label{fig_dblpForBibtexCH7}
\end{figure}

点击图~\ref{fig_dblpForBibtexCH7}中所示的链接之后将得到Bibtex信息,如图~\ref{fig_bibtexDetailCH7}所示。
打开本地文件夹下的sample.bib文件,完整添加该信息。
并在需要引用的位置添加这一引用~\cite{DBLP:journals/computer/EgyedZHD18}。
格式为bibtex信息中的开头,\emph{例如图中的“DBLP:journals/computer/EgyedZHD18”。(此处是一个典型的因为长字符串导致的bad box,请参考上述章节的内容手动进行软换行)}。

\textbf{注意:在修改并保存sample.bib文件后,先编译tex文件,再编译参考文献,之后再编译两次,此时引用位置的方括号内将出现具体的编号而不是问号,且编号对应的引文已经出现在参考文献内。本模板点击引用编号后可以跳转到对应的参考文献处。}

在bib文件中出现,但并未在论文中被引用的论文不会出现在最后的参考文献中。如果dblp中并未包含你需要的论文,则可以尝试谷歌或百度学术的搜索结果,一般也包含bibtex信息,但可能不完整或不规范。

引用网站链接可以考虑这一格式~\cite{GanttSystemWeb}(不推荐,网站链接使用脚注更规范些)。

引用书籍可以考虑这一格式~\cite{Pohl2010Requirements}。

中文文献请参考这一格式~\cite{cyg2006}(引用标记不要使用中文,否则容易出现编译错误)。

以下英文引用用来测试引文排序是否按照插入顺序,以及多引文是否合并~\cite{DBLP:journals/computer/EgyedZHD18, DBLP:journals/ml/TingZCZWZ19}

\begin{figure}[htb]
  \centering
  \includegraphics[width=5in]{figure/chapter7/bibtexDetail.pdf}
  \caption{Bibtex详细信息}\label{fig_bibtexDetailCH7}
\end{figure}


%%%%%%%%%%%%%%%%%%%%%%%%%%%%%%%%%%%%%%%%%%%%%%%%%%%%%%%%%%%%%%%%%%%%%%%%%%%%%%%
% 致谢,应放在结论之后
\begin{acknowledgement}
在本篇论文最后,我想真诚地感谢在我撰写论文过程中指导和帮助我的老师、同学们,
还有支持我、鼓励我的家人和朋友们。

首先,我要感谢在我毕设研究工作中,张贺教授对我的悉心指导和帮助,
在跟随您的两年研究生学习工作中,您总是教导我们要多去实践,
要以严谨认真的态度踏踏实实地完成科研学习工作,
还要努力阅读大量学术文献,提高自己看问题的层次,
还为我们提供了很多与工业界交流的机会,这些都为我的毕设工作和日常工作学习带来了很大的好处。

然后,我要感谢实验室的师兄师姐和同学们,
感谢你们这两年中在学习、工作和生活中带给我的帮助与鼓励,
特别是指导我毕设工作的师姐,对我的问题一一解答,对我的论文工作耐心指导,
启发我进行深度的思考,反复与我评估和讨论论文研究的问题,
得益于她的帮助,我的论文才能保质保量地顺利完成。

最后,还要感谢我的父母和家人,感谢你们在背后默默支持和陪伴我。
很幸运,在研究生阶段遇到了相互扶持,互相鼓励的伴侣,
在生活和学习上我们互相帮助,给予彼此很大的动力与希望。
也要感谢我的舍友们,让我的生活中充满了乐趣。

在论文工作中,访谈和焦点小组中付出了大量的精力时间,
实现工具时也遇到过困难与问题,但我依然坚持尽力做到最好,
马上就要从学校步入社会,我将坚持“诚朴雄伟,励学敦行”的精神,
不畏艰难,努力奋斗出属于自己的一片天地。
  
\end{acknowledgement}


% 参考文献。应放在\backmatter之前。
% 推荐使用BibTeX,若不使用BibTeX时注释掉下面一句。
%\nocite{*}
\bibliography{sample}


% 附录,必须放在参考文献后,backmatter前
% \appendix
% \chapter{访谈问题}\label{app:1}
% \section{战术建模访谈问题}

%%%%%%%%%%%%%%%%%%%%%%%%%%%%%%%%%%%%%%%%%%%%%%%%%%%%%%%%%%%%%%%%%%%%%%%%%%%%%%%
% 书籍附件
\backmatter
%%%%%%%%%%%%%%%%%%%%%%%%%%%%%%%%%%%%%%%%%%%%%%%%%%%%%%%%%%%%%%%%%%%%%%%%%%%%%%%
% 作者简历与科研成果页,应放在backmatter之后
\begin{resume}
% 论文作者身份简介,一句话即可。
  \begin{authorinfo}
    \noindent 张富利,男,汉族,1997年8月出生,山东省临沂人。
  \end{authorinfo}
  % 论文作者教育经历列表,按日期从近到远排列,不包括将要申请的学位。
  \begin{education}
    \item[2015年9月 --- 2019年6月] 中国石油大学(华东)计算机与通信工程学院 \hfill 本科
  \end{education}
  % 论文作者在攻读学位期间所发表的文章的列表,按发表日期从近到远排列。
  \begin{publications}
    \item Fuli Zhang, Peiyu Hou,  ``一种智能合约微服务化框架''
    3篇论文!! 
    in \textsl{Proc. IEEE International Conference on Communications (ICC) 2010}, May. 2010.
    \item  论文作者在攻读学位期间参与的科研课题的列表,按照日期从近到远排列。
  \end{publications}
 

  % \begin{projects}
  % % \item 国家自然科学基金面上项目``问题研究''
  % % (课题年限~2010年1月 --- 2012年12月),负责相关问题的研究。
  % \end{projects}
  \end{resume}

%%%%%%%%%%%%%%%%%%%%%%%%%%%%%%%%%%%%%%%%%%%%%%%%%%%%%%%%%%%%%%%%%%%%%%%%%%%%%%%
% 生成《学位论文出版授权书》页面,应放在最后一页
%\makelicense

%%%%%%%%%%%%%%%%%%%%%%%%%%%%%%%%%%%%%%%%%%%%%%%%%%%%%%%%%%%%%%%%%%%%%%%%%%%%%%%
\end{document}
